\section{Hybrid Recommender System: Combination of Content-Based Method and Matrix Factorization}
In the above paper, matrix factorization has been discussed. For the content-based method, the concept is straightforward, which is to recommend items that are similar to what the target user liked before. Now, the intuition is whether we can combine the two approaches into a hybrid recommender system. Recall that in matrix factorization, by the stochastic gradient descent algorithm, we train two latent vector matrices: user matrix \textit{p} and item matrix \textit{q}. Through the inner product of user matrix \textit{p} and item matrix \textit{q}, we can have the predicted user-item matrix. In the predicted user-item matrix, row vectors represent the preferences of each user, and column vectors represent the attributes of each item, which means the user similarity can be obtained by calculating the similarity between row vectors, and similarly, the item similarity can be obtained by calculating the similarity of column vectors.

For the proposed hybrid system, the content-based approach is performed based on the matrix factorization (i.e., the content-based approach calculates the similarity through the user-item rating matrix derived by the result of matrix factorization). Consider that the detailed methodology of matrix factorization is already stated in section 3.1, so there we will illustrate the methodology after obtaining the user-item rating matrix, which derives from the result of matrix factorization.

Assume the predicted user-item rating matrix is available to recommend items to the target user. We have the following steps:
\begin{itemize}
    \item[1. ]  List all the item indexes rated by the target user.
    \item[2. ] Sort the items that have not yet been rated by the target user (i.e., the item indexes are not listed in the \textit{step1}) through our predicted rating score.
    \item[3. ] Get the top ranked item of the list created by \textit{step2}.
\end{itemize}
Then we use the \textit{cosine similarity} for similarity measurement to measure how similar the items are in attractiveness to the target user, where:
\begin{equation}
\operatorname{similar}(i, j)=\cos (i, j)=\frac{i \cdot j}{\|i\| \cdot\|j\|}
\end{equation}
The formula uses the angle between two vectors to calculate the cosine value. The angle between the vectors $i$ and $j$ is the thing this formula cares about, instead of the magnitude. So that the value of similarity is within the interval [-1,1], and 1 means “exactly the same”. After calculating the cosine similarity, we sort the items by the value of similarity, and then we can get the items that are most similar to the target user's favorite items. The implementation of this hybrid recommender system with real-world data will be presented in the next chapter.

It is foreseeable that if it combines memory-based collaborative filtering with a content-based approach, it can solve the problems faced by memory-based collaborative filtering, such as sparsity and loss of information. Similarly, as can be seen from our method description, hybrid systems increase computational steps and complexity.