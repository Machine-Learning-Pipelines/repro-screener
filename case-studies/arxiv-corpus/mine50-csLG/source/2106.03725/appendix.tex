%!TEX root = stability_manifold_TSP.tex


%%%%%%%%%%%%%%%%%%%%%%%%%%%%%%%%%%%%%%%%%%%%%%%%
%%%%%%%%%%%%%%%%%% SUBSECTION %%%%%%%%%%%%%%%%%% 
%%%%%%%%%%%%%%%%%%%%%%%%%%%%%%%%%%%%%%%%%%%%%%%%
\subsection{Definition \ref{def:manifold-convolution} and Convolutional Filters in Continuous Time}
\label{app:rem_convolution} 
The manifold convolution in Definition \ref{def:manifold-convolution} can also be motivated with a connection to linear time invariant filters. This requires that we consider the differential equation
%
\begin{equation} \label{eqn:wave}
   \frac{\partial u(x,t)}{\partial t}=  \frac{\partial}{\partial x} u(x,t) \text{.}
\end{equation}
%
This is a one-sided wave equation and it is therefore not an exact analogous of the diffusion equation in \eqref{eqn:heat} -- this would require that the second derivative be used in the right of \eqref{eqn:wave}. The important observation to make here is that the exponential of the derivative operator is a time shift so that we can write $u(x,t) = e^{t\partial/\partial x}f(x) = f(x-t)$. This is true because $e^{t\partial/\partial x}f(x)$ and $f(x-t)$ are both solutions of \eqref{eqn:wave}. It then follows that Definition \ref{def:manifold-convolution} particularized to \eqref{eqn:wave} yields the convolution definition
%
\begin{equation}\label{eqn:conv-1d-1}
    g(x) = \int_{0}^\infty \tdh(t) e^{t\partial/\partial x}f(x)\, \text{d}t.
         = \int_{0}^\infty \tdh(t) f(x-t) \,\text{d}t.
\end{equation}
%
This is the standard definition of time convolutions. 

The frequency representation result in Proposition \ref{prop:filter-spectral} holds for \eqref{eqn:conv-1d-1} and it implies that standard convolutional filters in continuous time are completely characterized by the frequency response in Definition \ref{def:frequency-response}. The more standard definition of a filter's frequency response as the Fourier transform of the impulse response $\tdh(t)$ -- as opposed to the Laplace transform we use in Definition \ref{def:frequency-response} -- suffices because complex exponentials $e^{jw}$ are an orthonormal basis of eigenfunctions of the derivative operator with associated eigenvalues $j\omega$. 

%%%%%%%%%%%%%%%%%%%%%%%%%%%%%%%%%%%%%%%%%%%%%%%%
%%%%%%%%%%%%%%%%%% SUBSECTION %%%%%%%%%%%%%%%%%% 
%%%%%%%%%%%%%%%%%%%%%%%%%%%%%%%%%%%%%%%%%%%%%%%%
\subsection{Proof of Theorem \ref{thm:perturb}}
 \label{app:perturb}
 Based on equation \eqref{eqn:deform} and the definition in \eqref{eqn:Laplacian}, the operation carried out on the deformed manifold data $f$ can be written as
\begin{align}
   - \ccalL' f(x)&= (\nabla\cdot \nabla)  f(\tau(x))
   \\ \label{eqn:changevariable} &=   (J(\tau_*)^T \nabla_\tau \cdot J(\tau_*)^T \nabla_\tau)  f(\tau(x)).
\end{align}
The equality in \eqref{eqn:changevariable} results from the chain rule of gradient operator where $\nabla_\tau$ is denoted as the intrinsic gradient around $\tau(x)$ in the tangent space $T_{\tau(x)}\ccalM$.
%\red{Explain the steps in the equation above. E.g., that the first equality comes from the definition of the Laplace-Beltrami operator (add citation if needed), the second from the chain rule, and the inequality from Cauchy-Schwarz. Also, here you are using the notation $\cdot$ for the inner product, while before you used $\langle\cdot,\cdot\rangle$. Try to be consistent with the notation.} 
By replacing $J(\tau_*) = I + \Delta$ the inner product term,  \eqref{eqn:changevariable} can be rewritten as
\begin{align}
    \nonumber J(\tau_*)^T\nabla_\tau\cdot  J(\tau_*)^T \nabla_\tau  =  \nabla_\tau \cdot & \nabla_\tau  + 2(\Delta^T \nabla_\tau \cdot \nabla_\tau ) \\
    &+  \Delta^T\nabla_\tau \cdot \Delta^T\nabla_\tau.
\end{align}
%with $\bbA_2$ contains $\nabla_\tau-\nabla$ term. Due to the smoothness of the gradient field, $\|\nabla_\tau-\nabla\|\leq \|\tau(x)\|$ with $\upsilon(x)=x+\tau(x)$. This leads to the norm of $\bbA_2$ bounded as $\|\bbA_2\|=O(\|\tau(x)\|)$.
%\red{For the equation above, same comment as before: explain the non-obvious steps.}
With $\ccalL=- \nabla_\tau \cdot \nabla_\tau$, the perturbed operator is
\begin{align}
   \label{eqn:E1} \ccalL-\ccalL'&= 2(\Delta^T \nabla_\tau \cdot \nabla_\tau )  + \Delta^T\nabla_\tau \cdot \Delta^T\nabla_\tau \\
   \label{eqn:E2} & = 2\|\Delta\|_F (\nabla_\tau \cdot \nabla_\tau) + \|\Delta\|_F^2 (\nabla_\tau \cdot \nabla_\tau)+\bbA.
\end{align}
From \eqref{eqn:E1} to \eqref{eqn:E2}, we extract the relative term and use $\bbA$ to represent the compliment terms.
This leads to $\ccalE  = \|\Delta\|_F^2 + 2\|\Delta\|_F$,
as the relative perturbation term, the norm of which is bounded by the leading term as $O(\epsilon)$. The norm of the compliment term therefore can be written as
\vspace{-1mm}
    \begin{align}
    \|\bbA\|& = \|\ccalE (\nabla_\tau \cdot \nabla_\tau)- 2(\Delta^T \nabla_\tau \cdot \nabla_\tau )  - \Delta^T\nabla_\tau \cdot \Delta^T\nabla_\tau \|\\
   \nonumber &\leq \left\lVert2\|\Delta\|_F (\nabla_\tau \cdot \nabla_\tau) -  2(\Delta^T \nabla_\tau \cdot \nabla_\tau )\right\rVert\\
    &\qquad \quad +\left\lVert \|\Delta\|_F^2 (\nabla_\tau \cdot \nabla_\tau) - \Delta^T\nabla_\tau \cdot \Delta^T\nabla_\tau\right\rVert,
\end{align}
which can be also bounded by the leading terms as $O(\epsilon)$ combining with the boundedness of the gradient field.



%%%%%%%%%%%%%%%%%%%%%%%%%%%%%%%%%%%%%%%%%%%%%%%%
%%%%%%%%%%%%%%%%%% SUBSECTION %%%%%%%%%%%%%%%%%% 
%%%%%%%%%%%%%%%%%%%%%%%%%%%%%%%%%%%%%%%%%%%%%%%%

\subsection{Proof of Theorem \ref{thm:stability_abs_filter}}
\label{app:stability_abs_filter}
In the following, we denote $\langle\cdot,\cdot \rangle_{L^2(\ccalM)}$ as $\langle\cdot,\cdot \rangle$ and $\|\cdot\|_{L^2(\ccalM)}$ as $\|\cdot\|$ for simplicity. We start by bounding the norm difference between two outputs of filter functions on operators $\ccalL$ and $\ccalL'$ defined in \eqref{eqn:convolution-general} as
\begin{align}
 \nonumber  &\left\| \bbh(\ccalL) f - \bbh(\ccalL')f\right\| 
  = \\
  &\qquad \qquad\left\| \sum_{i=1}^\infty \hat{h}(\lambda_{i}) \langle f, \bm\phi_i \rangle \bm\phi_i - \sum_{i=1}^\infty \hat{h}(\lambda'_{i}) \langle f, \bm\phi'_i \rangle \bm\phi'_i \right\|. \label{eqn:diff}
\end{align}
We denote the index of partitions that contain a single eigenvalue as a set $\ccalK_s$ and the rest as a set $\ccalK_m$. We can decompose the filter function as $\hat{h}(\lambda)=h^{(0)}(\lambda)+\sum_{l\in\ccalK_m}h^{(l)}(\lambda)$ with
\begin{align}
\label{eqn:h0}& h^{(0)}(\lambda) = \left\{ 
\begin{array}{cc} 
              \hat{h}(\lambda)-\sum\limits_{l\in\ccalK_m}\hat{h}(C_l)  &  \lambda\in[\Lambda_k(\alpha)]_{k\in\ccalK_s} \\
                0& \text{otherwise}  \\
                \end{array} \right. \\
\label{eqn:hl}& h^{(l)}(\lambda) = \left\{ 
\begin{array}{cc} 
                \hat{h}(C_l) &  \lambda\in[\Lambda_k(\alpha)]_{k\in\ccalK_s} \\
                \hat{h}(\lambda) & 
                \lambda\in\Lambda_l(\alpha)\\
                0 &
                \text{otherwise}  \\
                \end{array} \right.             
\end{align}
where $C_l$ is some constant in $\Lambda_l(\alpha)$. We can start by analyzing the output difference of $h^{(0)}(\lambda)$. With the triangle inequality, the norm difference can then be written as
\begin{align}
 & \nonumber \left\| \sum_{i=1}^\infty h^{(0)}(\lambda_{i}) \langle f, \bm\phi_i \rangle \bm\phi_i  -  h^{(0)}(\lambda'_i )  \langle f, \bm\phi'_i \rangle \bm\phi'_i \right\|  \\
 &\nonumber =\Bigg\|\sum_{i=1}^\infty  h^{(0)}(\lambda_{i}) \langle f, \bm\phi_i \rangle \bm\phi_i -  h^{(0)}(\lambda_{i}) \langle f, \bm\phi'_i \rangle \bm\phi'_i  + \\
 &\qquad \qquad \qquad h^{(0)}(\lambda_{i}) \langle f, \bm\phi'_i \rangle \bm\phi'_i- h^{(0)}(\lambda'_{i}) \langle f, \bm\phi'_i \rangle \bm\phi'_i \Bigg\| \\
  & \nonumber \leq \left\|\sum_{i=1}^\infty  h^{(0)}(\lambda_i)  \langle f, \bm\phi_i \rangle \bm\phi_i -  h^{(0)}(\lambda_i ) \langle f, \bm\phi'_i \rangle \bm\phi'_i\right\|  +\\
  &\qquad\quad  \left\|\sum_{i =1}^\infty h^{(0)}(\lambda_{i}) \langle f, \bm\phi'_i \rangle \bm\phi'_i -  h^{(0)}(\lambda'_{i}) \langle f, \bm\phi'_i \rangle \bm\phi'_i \right\| \\
    &\nonumber \leq \Bigg\| \sum_{i=1}^\infty  h^{(0)}(\lambda_i)\Bigg( \langle f, \bm\phi_i \rangle\bm\phi_i-\langle f, \bm\phi_i \rangle\bm\phi'_i +\langle f, \bm\phi_i \rangle\bm\phi'_i - \\
    &\quad \langle f, \bm\phi'_i \rangle \bm\phi'_i \Bigg) \Bigg\| +\left\|\sum_{i =1}^\infty  (h^{(0)}(\lambda_i ) -h^{(0)}(\lambda'_i) ) \langle f, \bm\phi'_i \rangle \bm\phi'_i  \right\| \\
    & \nonumber \leq \left\| \sum_{i=1}^\infty h^{(0)}(\lambda_i )\langle f, \bm\phi_i \rangle (\bm\phi_i - \bm\phi'_i ) \right\| \\\nonumber &\qquad \qquad  +\left\|  \sum_{i =1}^\infty  h^{(0)}(\lambda_i )\langle f, \bm\phi_i - \bm\phi'_i  \rangle \bm\phi'_i \right\|\\
      &\qquad \qquad \quad 
     + \left\|\sum_{i=1}^\infty  (h^{(0)}(\lambda_i ) -h^{(0)}(\lambda'_i) ) \langle f, \bm\phi'_i \rangle \bm\phi'_i  \right\| \label{eqn:3}
\end{align}
%From equation \eqref{eqn:3} to equation \eqref{eqn:4}, we use the orthonormal property that $\|\bm\phi_i\|=1$ for all $i$. 

For the first term in \eqref{eqn:3}, we employ Lemma \ref{lem:davis-kahan} and therefore we have $\sigma=\lambda_i$ and $\omega=\lambda'_i$, for $\lambda_i\in [\Lambda_k(\alpha)]_{k\in\ccalK_s}$ we can have
\begin{align}
\left\| \bm\phi_i -\bm\phi'_i \right\| \leq \frac{\pi}{2} \frac{\|\bbA\|}{\alpha-\epsilon}= \frac{\pi}{2} \frac{\epsilon}{\alpha-\epsilon}.
\end{align}
Here $d$ can be seen as $d=\min_{\lambda_i\in\Lambda_k(\alpha),\lambda_j\in\Lambda_l(\alpha),k\neq l}|\lambda_i-\lambda_j'|$. Combined with the fact that $|\lambda_i-\lambda_j|>\alpha$ and $|\lambda_i-\lambda_i'|\leq \epsilon$ for all $\lambda_i\in\Lambda_k(\alpha),\lambda_j\in\Lambda_l(\alpha),k\neq l$, we have $d\geq \alpha-\epsilon$. With Cauchy-Schwartz inequality, we have the first term in \eqref{eqn:3} bounded as
\begin{align}
&\nonumber\left\| \sum_{i=1}^\infty h^{(0)}(\lambda_i )\langle f, \bm\phi_i \rangle (\bm\phi_i - \bm\phi'_i ) \right\|\\
& \leq \sum_{i=1}^\infty |h^{(0)}(\lambda_i)| | \langle f, \bm\phi_i \rangle | \left\|\bm\phi_i-\bm\phi'_i \right\| \leq  \frac{N_s\pi\epsilon}{2(\alpha-\epsilon)}  \|f\|.
\end{align}

The second term in \eqref{eqn:3} is bounded as
\begin{align}
 &\nonumber \left\|  \sum_{i =1}^\infty  h^{(0)}(\lambda_i )\langle f, \bm\phi_i - \bm\phi'_i  \rangle \bm\phi'_i \right\| \\
 &\leq   \sum_{i =1}^\infty |h^{(0)}(\lambda_i)| \|\bm\phi_i - \bm\phi'_i \| \|f\| \leq   \frac{N_s \pi\epsilon}{2(\alpha-\epsilon)}  \|f\|.
\end{align}
These two bounds are obtained by noting that $|h^{(0)}(\lambda)|<1$ and $h^{(0)}(\lambda)=0$ for $\lambda\in[\Lambda_k(\alpha)]_{k\in\ccalK_m}$. The number of eigenvalues within $[\Lambda_k(\alpha)]_{k\in\ccalK_s}$ is denoted as $N_s$. The third term in \eqref{eqn:3} can be bounded by the Lipschitz continuity of $h$ combined with Lemma \ref{lem:eigenvalue_absolute}.
\begin{align}
\nonumber  \Bigg\|\sum_{i=1}^\infty  &(h^{(0)}(\lambda_i ) -h^{(0)}(\lambda'_i) ) \langle f, \bm\phi'_i \rangle \bm\phi'_i  \Bigg\|^2 
  \\ \nonumber & \leq \sum_{i=1}^\infty | h^{(0)}(\lambda_{i}) -h^{(0)}(\lambda'_i) |^2 |\langle f, \bm\phi'_i \rangle|^2 \\
  &\leq \sum_{i =1}^\infty A_h^2 |\lambda_i - \lambda'_i |^2 |\langle f, \bm\phi'_i \rangle|^2 \leq  A_h^2 \epsilon^2 \|f\|^2.
\end{align}

Then we need to analyze the output difference of $h^{(l)}(\lambda)$, we can bound this as
\begin{align}
    \nonumber &\left\| \bbh^{(l)}(\ccalL)f -\bbh^{(l)}(\ccalL')f \right\| 
    \\& \leq \left\| (\hat{h}(C_l)+\delta)f -(\hat{h}(C_l)-\delta)f\right\| \leq 2\delta\|f\|,
\end{align}
where $\bbh^{(l)}(\ccalL)$ and $\bbh^{(l)}(\ccalL')$ are manifold filters with filter function $h^{(l)}(\lambda)$ on the LB operators $\ccalL$ and $\ccalL'$ respectively.
Combining the filter functions, we can write
\begin{align}
   \nonumber &\|\bbh(\ccalL)f-\bbh(\ccalL')f\|=\\&
    \left\|\bbh^{(0)}(\ccalL)f +\sum_{l\in\ccalK_m}\bbh^{(l)}(\ccalL)f - \bbh^{(0)}(\ccalL')f - \sum_{l\in\ccalK_m} \bbh^{(l)}(\ccalL')f \right\|\\
    &\leq \|\bbh^{(0)}(\ccalL)f-\bbh^{(0)}(\ccalL')f\|+\sum_{l\in\ccalK_m}\|\bbh^{(l)}(\ccalL)f-\bbh^{(l)}(\ccalL')f\|\\
    &\label{eqn:sta-filter-alpha}\leq \frac{N_s\pi\epsilon}{\alpha-\epsilon}\|f\| + A_h\epsilon\|f\| +2(N-N_s)\delta\|f\|,
\end{align}
which concludes the proof.


