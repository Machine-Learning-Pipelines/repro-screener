%!TEX root = stability_manifold_TSP.tex


%%%%%%%%%%%%%%%%%%%%%%%%%%%%%%%%%%%%%%%%%%%%%%%%
%%%%%%%%%%%%%%%%%% SUBSECTION %%%%%%%%%%%%%%%%%% 
%%%%%%%%%%%%%%%%%%%%%%%%%%%%%%%%%%%%%%%%%%%%%%%%
%\red{L: The way this paragraph is written now, it sounds like the stability-discriminability tradeoff only stems from the term that depends on the Lipschitz constant. However, the tradeoff comes from both terms: from the first because a larger frequency threshold leads to filters that give similar response to a large group of eigenvalues; and from the second because smaller Lipschitz constants lead to less discriminative filters. My suggestion is that you don't break up the analysis in the first and second terms. Just discuss the effect of $\alpha,\gamma$ and $A_h,B_h$ on both stability and discriminability, and explain how these effects are opposites and thus lead to a stability-discriminability tradeoff.}
{\myparagraph{Stability vs. discriminability tradeoff}}
In both stability theorems for manifold filters (Theorems \ref{thm:stability_abs_filter}, \ref{thm:stability_rela_filter}) and in the stability theorem for MNNs (Theorem \ref{thm:stability_nn}), the stability  bounds depend on the frequency partition threshold ($\alpha$ or $\gamma$), the number of total partitions ($N$ or $M$) and the Lipschitz continuity constant ($A_h$ or $B_h$). 
%In the case of the frequency threshold ($\alpha$ or $\gamma$), 
The frequency partition threshold and the number of partitions have a combined effect on stability. As indicated by Definitions \ref{def:alpha-spectrum} and \ref{def:frt-spectrum}, a larger frequency threshold leads to a smaller number of singletons, as eigenvalues that would otherwise be separated for small thresholds end up being grouped when the threshold is large. While a large frequency threshold results in a larger number of partitions that contain more than one eigenvalue, the total number of partitions ($N$ or $M$) either stays the same or decreases because the number of eigenvalues does not exceed the number of partitions [cf. Proposition \ref{prop:finite_num} or \ref{prop:finite_num_rela}]. Thus, a larger frequency threshold and a smaller number of partitions both lead to a smaller stability bound. Simultaneously, a large frequency threshold makes the spectrum separated more sparsely. Therefore, a large number of eigenvalues are amplified in a similar manner, which makes the filter function less discriminative. The Lipschitz constant ($A_h$ or $B_h$) affects stability and discriminability in similar ways. Smaller Lipschitz constants decrease the stability bound, but lead to smoother filter functions which give similar frequency responses to different eigenvalues.
%A smaller Lipschitz constant leads to a s This also leads to an increasing of the stability while decreasing the discriminability with all eigenvalues treated similarly. In conclusion, larger values of frequency threshold ($\alpha$ or $\gamma$), smaller values of total partitions ($N$ or $M$) and smaller values of Lipschitz constant ($A_h$ or $B_h$) improve stability of manifold filters and MNNs, but worsen spectral discriminability. 
Hence, in both manifold filters and MNNs we observe a trade-off between stability and discriminability. Nevertheless, in MNNs this trade-off is alleviated due to the presence of nonlinearities as discussed below.

% The frequency threshold decides the partition criterion on the infinite spectrum. As Definition \ref{def:alpha-spectrum} and \ref{def:frt-spectrum} indicate, we can see that when the frequency thresholds  are larger, more eigenvalues will be grouped in the same partition which also leads to the decreasing of the total number of partitions ($N$ or $M$). These both lead to a smaller stability bound. On the other hand, a smaller Lipschitz continuity constant indicates a smoother filter function and gives more similar responses to different frequencies over the spectrum. However, considering a smooth filter function with few fluctuations over a large part of the spectrum, the filter would have problem in discriminating different frequencies. 
 
% In conclusion, smaller values of $A_h$ and larger values of $\alpha$ improve filter stability, but worsen spectral discriminability.
%As filters give similar responses to all frequency components and separate the spectrum more sparsely  by treating more eigenvalues with little difference, the filter function becomes less discriminative. 

%%%%%%%%%%%%%%%%%%%%%%%%%%%%%%%%%%%%%%%%%%%%%%%%
%%%%%%%%%%%%%%%%%% SUBSECTION %%%%%%%%%%%%%%%%%% 
%%%%%%%%%%%%%%%%%%%%%%%%%%%%%%%%%%%%%%%%%%%%%%%%
\myparagraph{Pointwise nonlinearity} 
%Based on the conclusions derived for the stability of manifold filters and MNNs under both absolute and relative perturbations, we can observe a trade-off between the stability and discriminability. 
%While increasing the frequency difference or frequency ratio thresholds (i.e. $\alpha$ or $\gamma$) and decreasing the Lipschitz or integral Lipschitz constants (i.e. $A_h$ or $B_h$) helps improve the stability bounds. This comes with a price of less discriminative responses as more eigenvalues especially high-eigenvalue frequencies are grouped in the same partition. 
%Nonlinearities have the effect of scattering the spectral components associated with the eigenvalues that are grouped together by the filters to other parts of the spectrum, where they can then be discriminated by the manifold filters in the following layer. 
As demonstrated by Propositions \ref{prop:finite_num} and \ref{prop:finite_num_rela}, large eigenvalues of LB operator tend to be grouped together in one large group and share similar frequency responses. This is part of the reason why manifold filters have a stability-discriminability tradeoff, which implies that they cannot be stable and discriminative at the same time. However, in MNNs this problem is circumvented with the addition of nonlinearities. Nonlinearities have the effect of scattering the spectral components all over the eigenvalue spectrum. In the MNN, they mix the frequency components by spilling spectral components associated with the large eigenvalues onto the smaller eigenvalues, where they can then be discriminated by the manifold filters in the following layer. This is consistent with the role of nonlinear activation functions in graph neural networks (GNNs) n\cite{gama2020stability}, which can be see as instantiations of MNNs on discrete samples of the manifold as further discussed in Section \ref{sec:discre_nn}.

%%%%%%%%%%%%%%%%%%%%%%%%%%%%%%%%%%%%%%%%%%%%%%%%
%%%%%%%%%%%%%%%%%% SUBSECTION %%%%%%%%%%%%%%%%%% 
%%%%%%%%%%%%%%%%%%%%%%%%%%%%%%%%%%%%%%%%%%%%%%%%
\myparagraph{Comparison with graphons} The graphon is another infinite-dimensional model that can represent the limit of convergent sequences of graphs, and a series of works have proved stability of graphon neural networks and the transferability of GNNs sampled from them \cite{ruiz2020graphon, ruiz2021transferability, ruiz2021graph, maskey2021transferability, keriven2020convergence}. Manifolds are however more powerful, because they can represent the limit of graphs with both bounded and unbounded degrees \cite{belkin2008towards}---the graphon is only the limit of sequences of dense graphs \cite{lovasz2012large}. Moreover, embedded manifolds in high-dimensional spaces are more realistic geometric models in a number of application scenarios, such as point clouds, 3D shape segmentation and classification. Other important differences are that (i) the stability analysis on graphon models in \cite{ruiz2021graphon,ruiz2020graphon} focuses on deformations to the adjacency matrix of the graph, which can be translated directly as perturbations of the graphon operator, and that (ii) in the case of graphons, only an absolute perturbation model makes sense since given that the graphon spectrum is bounded a relative perturbation can always be bounded by an absolute perturbation. Meanwhile, deformations to the manifold domain translate into a combination of absolute and relative perturbations of the LB operator, and the fact that the LB operator spectrum is unbounded makes the effects of absolute and relative perturbations distinct, especially in the high-frequency domain.

% \red{(L: Rewrite the last couple of sentences. It is not very clear what you mean by regulation. Also, I think it is worth mentioning that on graphons the deformation is to ``the adjacency matrix'', and translates directly into a perturbation of the operator; whereas on manifolds, the deformation is to the manifold, and translates into an absolute and a relative perturbation of the LB operator. Another important difference is that while on graphons the spectrum is bounded, on manifolds it is unbounded, which makes the effect of absolute and relative perturbations very distinct, especially in the high frequencies.)}
% The infinite spectrum of the manifold LB operator is also different from the bounded spectrum of the graphon operator. Therefore, we make regulations on the manifold depending on the spectral frequency. Furthermore, the unboundedness of the LB operator spectrum also makes it reasonable to consider the relative perturbation term, which allows for a more general perturbation model and a relation to the manifold deformation.



