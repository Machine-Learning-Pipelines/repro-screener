\section{Related Work}
\label{related_work}

\subsection{Soft underwater robots}
Soft robots are difficult to optimally design and control when compared to their rigid counterparts due to the infinite dimensionality of their compliant structures.
Due to this modeling complexity, an experienced designer must hand craft each design guided by intuition, experiments, and approximate models.
Marchese et al. offer approaches to designing and fabricating soft fluidic elastomer robots, the type of robot we are also using in this work~\cite{marchese2015recipe}.
%
Katzschmann et al. present the design, fabrication, control, and testing of a soft robotic fish with interior cavities that is hydraulically actuated. Their manually designed robot can swim at multiple depths and record aquatic life in the ocean~\cite{katzschmann2016hydraulic,katzschmann2018exploration}. 
Zhu et al. manually optimize the swimming performance of their robotic fish, Tunabot~\cite{zhu2019tuna}. The authors measured kinematics, speed, and power at increasing flapping frequencies to quantify swimming performance and find agreement in performance between real fish and their Tunabot over a wide range of frequencies.
Zheng et al. propose to design soft robots by pre-checking controllability during the numerical design phase~\cite{zheng2019controllability}. FEM is used to model the dynamics of cable-driven parallel soft robot and a differential geometric method is applied to analyze the controllability of the points of interest.
Katzschmann et al.~\cite{katzschmann2019dynamically} manually tweak the material parameters of their reduced-order FEM~\cite{thieffry2018control} with an experimental soft robotic arm to perform dynamic closed-loop control.
Van et al. present a DC motor driven soft robotic fish which is optimized for speed and efficiency based on experimental, numerical and theoretical investigation into oscillating propulsion~\cite{van2020biomimetic}.
Wolf et al. use a pneumatically-actuated fish-like stationary model to investigate how parameters like stiffness, strength, and frequency affect thrust force generation~\cite{wolf2020fish}. Wolf et al. measure thrust, side forces, and torques generated during propulsion and use a statistical linear model to examine the effects of parameter combinations on thrust generation; they show that both stiffness and frequency substantially affect swimming kinematics.
We are not aware of any work that uses a fast differentiable FEM simulation environment to learn material parameters for soft robotic fish using a bollard-pull style experimental setup.

\subsection{Differentiable soft-body simulators}
Our work is also relevant to the recent developments of robotic simulators, particularly for soft robots.
Geilinger et al. \cite{geilinger2020add} present a differentiable multi-body dynamics solver that is able to handle frictional contact for rigid and deformable objects.
Coevoet et al. \cite{coevoet2017software} notably present a non-differentiable framework for modeling, simulation, and control of soft-bodied robots using continuum mechanics for modeling the robotic components and using Lagrange multipliers for boundary conditions like actuators and contacts.
Most related to our work are the recent works on differentiable soft-body and fluid simulators~\cite{du2020stokes,du2021diffpd,hahn2019real2sim,hu2019difftaichi,hu2019chainqueen,huang2021plasticine,ma2021diffaqua}. 
These papers develop numerical methods for computing gradients in a traditional simulators. Furthermore, they demonstrate the power of gradient information in robotics applications, e.g., system identification or trajectory optimization. Most of the works present simulation results only, with ChainQueen~\cite{hu2019chainqueen} and Real2Sim~\cite{hahn2019real2sim} being two notable exceptions that discuss real-world soft-robot applications.
Notably, \cite{hahn2019real2sim} optimizes visco-elastic material parameters of a finite element simulation to approximate the dynamic deformations of real-world soft objects, such as an open-loop controlled tendon-driven crawling robot.
Bern et al. \cite{bern2020soft} have also demonstrated the use of differentiable simulation to learn from a quasi-static data set for the purpose of optimizing open-loop control inputs.
Dubied et al. \cite{dubied2022sim} is the most recent example that demonstrates sim2real matching for a soft robotic fish tail, shows system identification on a passive structure for just the Young's modulus, and investigates the mismatch in damping between reality and simulation. In this previous work, the fish tail actuation is simulated using a simplified muscle model and only one design is shown whereas in this paper, the pressure boundary condition is simulated exactly as fabricated for each pneumatic chamber geometry for three different designs. Simulating the pneumatic chambers improves accuracy and allows for physically-plausible Young's moduli and Poisson ratios to be identified. In this current work, we further demonstrate that the gradient-based optimization can be carried out to higher dimensional design spaces that include more than one material parameter.

\subsection{Hydrodynamic Surrogates}
For underwater soft robots, the challenge of simulation is exacerbated by the hydrodynamic interaction with the soft body.
Several previous works tackle the fluid-structure interaction problem through different methods, including heuristic hydrodynamics~\cite{du2021underwater,ma2021diffaqua,min2019softcon}, physically-informed neural network approaches~\cite{wandel_learning_2021}, and data-driven learning approaches~\cite{chen2018neural}.

Compared to these previous methods that simulate underwater soft robots such as~\cite{du2021underwater}, our work models pneumatic actuation using the exact chamber geometry rather than artificial muscles facilitating greater accuracy at large deformations (see \Cref{fig:pneumatic_fish_tail}), uses a neural network thrust predictor rather than approximate analytical or heuristic hydrodynamics, and presents a more sophisticated hardware pipeline that can be used to validate simulation. 