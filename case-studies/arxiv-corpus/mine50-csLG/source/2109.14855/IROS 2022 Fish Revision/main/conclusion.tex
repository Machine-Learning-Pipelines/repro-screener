\section{Conclusion}
\label{conclusion}
We present an experimentally-verified simulation framework that can be used to accurately predict the deformations of a pneumatically actuated fish tail with a flexible spine.
Our pipeline can accurately learn material parameters from a quasi-static data sets without having to do expensive and time-consuming material testing. It also eliminates the need to do manual tuning of material constants to get accurate simulation results. The parameters we found are not only within typical range of measured material parameters for our materials, but can be used to successfully predict the behavior of dynamic experiments for different pressure actuation amplitudes and frequencies to within $3\%$ positional error normalized to a actuator length of \SI{10}{cm}. Although we use an isotropic corotated material, which is linear elastic, we find that this model is more sufficient to model large deformations on average giving acceptable displacement results for our engineering application. In these experiments, the damping of the material and the hydrodynamic effects are found to be negligible. This is because the actuation pressures used dominate the deformation compared to losses and hydrodynamic pressure. 

We show a data-driven approach can be used to do simple prediction on a useful performance metric such as thrust force given a suitable hardware setup. However, more work is needed to produce a more robust thrust predictor if the morphology of the actuator changes substantially. We claim that for small design changes such as the choice of silicone or the number of internal chambers this framework can be used to quickly assess the relative merits of each design with a relatively sparse data set of approximately 30 types of different actuation signals.

Our aim is to further progress towards a systematic method by which soft roboticists can simulate and optimize their designs and controllers, whether they be soft fish, manipulators, or other flavors of soft robots. A fast and physically-verified co-optimization method of design and control is the goal.