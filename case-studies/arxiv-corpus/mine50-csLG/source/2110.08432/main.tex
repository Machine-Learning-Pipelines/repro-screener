\documentclass[twoside,11pt]{article}

% Any additional packages needed should be included after jmlr2e.
% Note that jmlr2e.sty includes epsfig, amssymb, natbib and graphicx,
% and defines many common macros, such as 'proof' and 'example'.
%
% It also sets the bibliographystyle to plainnat; for more information on
% natbib citation styles, see the natbib documentation, a copy of which
% is archived at http://www.jmlr.org/format/natbib.pdf

\usepackage{jmlr2e}

\usepackage{amsmath,amssymb,xspace}

\usepackage[utf8]{inputenc} % allow utf-8 input
\usepackage[T1]{fontenc}    % use 8-bit T1 fonts
\usepackage{hyperref}       % hyperlinks
\usepackage{url}            % simple URL typesetting
\usepackage{booktabs}       % professional-quality tables
\usepackage{amsfonts}       % blackboard math symbols
\usepackage{nicefrac}       % compact symbols for 1/2, etc.
\usepackage{microtype}      % microtypography
\usepackage{xcolor}         % colors

% Recommended, but optional, packages for figures and better typesetting:
\usepackage{microtype}
\usepackage{graphicx}
%\usepackage{subfigure}
\usepackage{booktabs} % for professional tables

% Recommended, but optional, packages for figures and better typesetting:
\usepackage{microtype}
\usepackage{graphicx}

% For citations
\usepackage{natbib}
% \usepackage{amsfonts}
% \usepackage{amssymb, amsmath,amsthm}

% For algorithms
\usepackage{algorithm}
\usepackage{algorithmic}
\usepackage{paralist}
\usepackage{multirow}
% Added by Author
% use Times
\usepackage{times}
% For figures
\usepackage{wrapfig}
%\usepackage[authoryear]{natbib}

% For algorithms
\usepackage{url,enumerate}
\usepackage{color,xcolor}
\usepackage{makeidx}  % allows for indexgeneration
% \usepackage{amsmath,amssymb}
\usepackage{mathtools}
% \usepackage[small, compact]{titlesec}
\usepackage{xspace}
\usepackage{epstopdf}
\usepackage{cite}

% For algorithms
\usepackage{mathrsfs}
\usepackage{times}
\usepackage{enumerate}
\usepackage{color}
\usepackage{graphicx,epsfig}
% \usepackage{amsmath,amssymb,xspace}
\usepackage{url}
%\usepackage{subfigure}
\usepackage{hyperref}
\usepackage{bm}
\usepackage{bbm}
\usepackage{upgreek}
\usepackage{cleveref}
\usepackage{multirow}
\usepackage{ulem}
\usepackage{cancel}
\usepackage{subcaption}
\usepackage{dsfont}
\usepackage{adjustbox}

% Math commands by Thomas Minka
\newcommand{\var}{{\rm var}}
\newcommand{\Tr}{^{\rm T}}
\newcommand{\vtrans}[2]{{#1}^{(#2)}}
\newcommand{\kron}{\otimes}
\newcommand{\schur}[2]{({#1} | {#2})}
\newcommand{\schurdet}[2]{\left| ({#1} | {#2}) \right|}
\newcommand{\had}{\circ}
\newcommand{\diag}{{\rm diag}}
\newcommand{\invdiag}{\diag^{-1}}
\newcommand{\rank}{{\rm rank}}
% careful: ``null'' is already a latex command
\newcommand{\nullsp}{{\rm null}}
\newcommand{\tr}{{\rm tr}}
\renewcommand{\vec}{{\rm vec}}
\newcommand{\vech}{{\rm vech}}
\renewcommand{\det}[1]{\left| #1 \right|}
\newcommand{\pdet}[1]{\left| #1 \right|_{+}}
\newcommand{\pinv}[1]{#1^{+}}
\newcommand{\erf}{{\rm erf}}
\newcommand{\hypergeom}[2]{{}_{#1}F_{#2}}

\newcommand{\RN}[1] {\MakeUppercase{\romannumeral #1}}

% boldface characters

\renewcommand{\a}{{\bf a}}
\renewcommand{\b}{{\bf b}}
\renewcommand{\c}{{\bf c}}
\renewcommand{\d}{{\rm d}}  % for derivatives
\newcommand{\e}{{\bf e}}
\newcommand{\f}{{\bf f}}
\newcommand{\g}{{\bf g}}
\newcommand{\h}{{\bf h}}
%\newcommand{\k}{{\bf k}}
% in Latex2e this must be renewcommand
\renewcommand{\k}{{\bf k}}
\newcommand{\m}{{\bf m}}
\newcommand{\mb}{{\bf m}}
\newcommand{\n}{{\bf n}}
\renewcommand{\o}{{\bf o}}
\newcommand{\p}{{\bf p}}
\newcommand{\q}{{\bf q}}
\renewcommand{\r}{{\bf r}}
\newcommand{\s}{{\bf s}}
\renewcommand{\t}{{\bf t}}
\renewcommand{\u}{{\bf u}}
\renewcommand{\v}{{\bf v}}
\newcommand{\w}{{\bf w}}
\newcommand{\x}{{\bf x}}
\newcommand{\y}{{\bf y}}
\newcommand{\z}{{\bf z}}
%s\newcommand{\l}{\boldsymbol{l}}
\newcommand{\A}{{\bf A}}
\newcommand{\B}{{\bf B}}
\newcommand{\C}{{\bf C}}
\newcommand{\D}{{\bf D}}
\newcommand{\E}{{\bf E}}
\newcommand{\F}{{\bf F}}
\newcommand{\G}{{\bf G}}
\renewcommand{\H}{{\bf H}}
\newcommand{\I}{{\bf I}}
\newcommand{\J}{{\bf J}}
\newcommand{\K}{{\bf K}}
\renewcommand{\L}{{\bf L}}
\newcommand{\M}{{\bf M}}
%\newcommand{\Mcal}{{\mathcal{M}}}
\newcommand{\N}{\mathcal{N}}  % for normal density
\newcommand{\MN}{\mathcal{MN}} 
\newcommand{\Acal}{\mathcal{A}}
\newcommand{\Bcal}{\mathcal{B}}
\newcommand{\Ccal}{\mathcal{C}}
\newcommand{\Dcal}{\mathcal{D}}
\newcommand{\Ocal}{\mathcal{O}}
\newcommand{\Qcal}{\mathcal{Q}}
\newcommand{\Ycal}{\mathcal{Y}}
\newcommand{\Zcal}{\mathcal{Z}}
\newcommand{\Fcal}{\mathcal{F}}
\newcommand{\Vcal}{\mathcal{V}}
\newcommand{\Lcal}{\mathcal{L}}
\newcommand{\Tcal}{\mathcal{T}}
\newcommand{\Gcal}{\mathcal{G}}
\newcommand{\Hcal}{\mathcal{H}}
\newcommand{\Scal}{\mathcal{S}}

%\newcommand{\N}{{\bf N}}
\renewcommand{\O}{{\bf O}}
\renewcommand{\P}{{\bf P}}
\newcommand{\Q}{{\bf Q}}
\newcommand{\R}{{\bf R}}
\renewcommand{\S}{{\bf S}}
\newcommand{\T}{{\bf T}}
\newcommand{\U}{{\bf U}}
\newcommand{\V}{{\bf V}}
\newcommand{\W}{{\bf W}}
\newcommand{\X}{{\bf X}}
\newcommand{\Y}{{\bf Y}}
\newcommand{\Z}{{\bf Z}}
\newcommand{\Mcal}{{\mathcal{M}}}
\newcommand{\Wcal}{{\mathcal{W}}}
\newcommand{\Ucal}{{\mathcal{U}}}


% this is for latex 2.09
% unfortunately, the result is slanted - use Latex2e instead
%\newcommand{\bfLambda}{\mbox{\boldmath$\Lambda$}}
% this is for Latex2e
\newcommand{\bfLambda}{\boldsymbol{\Lambda}}

% Yuan Qi's boldsymbol
\newcommand{\bsigma}{\boldsymbol{\sigma}}
\newcommand{\balpha}{\boldsymbol{\alpha}}
\newcommand{\bpsi}{\boldsymbol{\psi}}
\newcommand{\bphi}{\boldsymbol{\phi}}
\newcommand{\boldeta}{\boldsymbol{\eta}}
\newcommand{\bbeta}{\boldsymbol{\beta}}
\newcommand{\hatbbeta}{\widehat{\boldsymbol{\beta}}}
\newcommand{\btau}{\boldsymbol{\tau}}
\newcommand{\bvarphi}{\boldsymbol{\varphi}}
\newcommand{\bzeta}{\boldsymbol{\zeta}}
\newcommand{\bepi}{{\boldsymbol{\epsilon}}}

\newcommand{\blambda}{\boldsymbol{\lambda}}
\newcommand{\bLambda}{\mathbf{\Lambda}}
\newcommand{\bOmega}{\mathbf{\Omega}}
\newcommand{\bomega}{\mathbf{\omega}}
\newcommand{\bPi}{\mathbf{\Pi}}

\newcommand{\btheta}{\boldsymbol{\theta}}
\newcommand{\bTheta}{\boldsymbol{\Theta}}
\newcommand{\bpi}{\boldsymbol{\pi}}
\newcommand{\bxi}{\boldsymbol{\xi}}
\newcommand{\bSigma}{\boldsymbol{\Sigma}}
\newcommand{\bnu}{{\boldsymbol{\nu}}}

\newcommand{\bgamma}{\boldsymbol{\gamma}}
\newcommand{\bGamma}{\mathbf{\Gamma}}

\newcommand{\bmu}{\boldsymbol{\mu}}
\newcommand{\brho}{\boldsymbol{\rho}}
\newcommand{\1}{{\bf 1}}
\newcommand{\0}{{\bf 0}}

% \newcommand{\comment}[1]{}

\newcommand{\bs}{\backslash}
\newcommand{\ben}{\begin{enumerate}}
\newcommand{\een}{\end{enumerate}}

 \newcommand{\notS}{{\backslash S}}
 \newcommand{\nots}{{\backslash s}}
 \newcommand{\noti}{{\backslash i}}
 \newcommand{\notj}{{\backslash j}}
 \newcommand{\nott}{\backslash t}
 \newcommand{\notone}{{\backslash 1}}
 \newcommand{\nottp}{\backslash t+1}
% \newcommand{\notz}{\backslash z}

\newcommand{\notk}{{^{\backslash k}}}
%\newcommand{\noti}{{^{\backslash i}}}
\newcommand{\notij}{{^{\backslash i,j}}}
\newcommand{\notg}{{^{\backslash g}}}
\newcommand{\wnoti}{{_{\w}^{\backslash i}}}
\newcommand{\wnotg}{{_{\w}^{\backslash g}}}
\newcommand{\vnotij}{{_{\v}^{\backslash i,j}}}
\newcommand{\vnotg}{{_{\v}^{\backslash g}}}
\newcommand{\half}{\frac{1}{2}}
\newcommand{\msgb}{m_{t \leftarrow t+1}}
\newcommand{\msgf}{m_{t \rightarrow t+1}}
\newcommand{\msgfp}{m_{t-1 \rightarrow t}}

\newcommand{\proj}[1]{{\rm proj}\negmedspace\left[#1\right]}
\newcommand{\argmin}{\operatornamewithlimits{argmin}}
\newcommand{\argmax}{\operatornamewithlimits{argmax}}

\newcommand{\dif}{\mathrm{d}}
\newcommand{\abs}[1]{\lvert#1\rvert}
\newcommand{\norm}[1]{\lVert#1\rVert}
\newcommand{\whL}{{\widehat{\Lcal}}}
\newcommand{\whJ}{{\widehat{J}}}

%miscellaneous symbols
%\newcommand{\ie}{{{\em i.e.,}}\xspace}
%\newcommand{\ie}{{\textit{i.e.,}}\xspace}
%\newcommand{\eg}{{\textit{e.g.,}}\xspace}
%\newcommand{\etc}{{\textit{etc.}}\xspace}
\newcommand{\EE}{\mathbb{E}}
\newcommand{\expt}[2]{\EE_{#1}\big[#2\big]}
\newcommand{\dr}[1]{\nabla #1}
\newcommand{\VV}{\mathbb{V}}
\newcommand{\sbr}[1]{\left[#1\right]}
\newcommand{\rbr}[1]{\left(#1\right)}
%\newcommand{\cmt}[1]{}


\newcommand{\bi}{{\bf i}}
\newcommand{\bj}{{\bf j}}
\newcommand{\bK}{{\bf K}}
\newcommand{\Vtr}{\mathrm{Vec}}
\newcommand{\tlam}{{\tilde{\lambda}}}
\newcommand{\kl}{{\mathrm{KL} }}



\newcommand{\dataset}{{\cal D}}
\newcommand{\fracpartial}[2]{\frac{\partial #1}{\partial  #2}}

% \def\thefootnote{*}\footnotetext{equal contribution}

%-------------------------------------------------------------------------
\newcommand\blfootnote[1]{%
  \begingroup
  \renewcommand\thefootnote{}\footnote{#1}%
  \addtocounter{footnote}{-1}%
  \endgroup
}

% % Enter the paper's authors in order
% \addauthor{Author1$^\ast$}{mail1@mail.com}{1}
% \addauthor{Author2$^\ast$}{mail2@mail.com}{1} 
% \addauthor{Author3}{mail3@mail.com}{1}

% % Enter the institutions
% % \addinstitution{Name\\Address}
% \addinstitution{
% Institute\\
%  Address
% }

% \def\eg{\emph{e.g}\bmvaOneDot}
% \def\Eg{\emph{E.g}\bmvaOneDot}
% \def\etal{\emph{et al}\bmvaOneDot}

\newcommand{\ours}{{A-MAML}\xspace}
\newcommand{\bmaml}{{{B-MAML}}\xspace}
\newcommand{\pmaml}{{{P-MAML}}\xspace}
\newcommand{\emaml}{{{E-MAML}}\xspace}
\newcommand{\fomaml}{{{FOMAML}}\xspace}
\newcommand{\imaml}{{{iMAML}}\xspace}
\newcommand{\maml}{{{MAML}}\xspace}
\newcommand{\rap}{{{Reptile}}\xspace}
%\newcommand{\rap}{{\color{red}{{{Reptile}}\xspace}}}
\newcommand{\zhec}[1]{\textcolor{blue}{#1}}
%\newcommand{\zhec}[1]{#1}
\newcommand{\cmt}[1]{}
\newcommand{\eg}{{\textit{e.g.},}\xspace}
\newcommand{\ie}{{\textit{i.e.},}\xspace}
\newcommand{\etc}{{\textit{etc}.}\xspace}

\newcommand{\zsdc}[1]{{#1}}

\newcommand{\akil}[1]{{\leavevmode\color{red}{#1}}}
\renewcommand{\akil}[1]{}
% \newcommand{\cmt}[1]{}
% \newcommand{\eg}{{\textit{e.g.},}\xspace}
% \newcommand{\ie}{{\textit{i.e.},}\xspace}
% \newcommand{\etc}{{\textit{etc}.}\xspace}

% %-------------------------------------------------------------------------
% % Document starts here
% \begin{document}

% \maketitle
% % This creates the footnote text
% %\blfootnote{$^\ast$ Equal Contribution.}

% Heading arguments are {volume}{year}{pages}{submitted}{published}{author-full-names}

% \jmlrheading{1}{2000}{1-48}{4/00}{10/00}{Marina Meil\u{a} and Michael I. Jordan}

% Short headings should be running head and authors last names

% \ShortHeadings{Meta Learning of Interface Conditions for Multi-Domain Physics-Informed Neural Networks}
\firstpageno{1}

\begin{document}

\title{Meta-Learning with Adjoint Methods}

\author{\name Shibo Li \email shibo@cs.utah.edu \\
       \addr School of Computing\\
       University of Utah
       \AND
       \name Zheng Wang \email wzhut@cs.utah.edu\\
       \addr School of Computing\\
       University of Utah
       \AND
       \name Akil Narayan \email akil@sci.utah.edu\\
       \addr Department of Mathematics, Scientific Computing and Imaging Institute\\
       University of Utah
       \AND
       \name Robert M. Kirby \email kirby@cs.utah.edu \\
       \addr School of Computing, Scientific Computing and Imaging Institute\\
       University of Utah
       \AND
       \name Shandian Zhe \email zhe@cs.utah.edu \\
       \addr School of Computing\\
       University of Utah }

% \editor{Leslie Pack Kaelbling}


\maketitle

\chapter*{Abstract}
Multi-armed bandits (MAB) provide a principled online learning approach to attain the balance between exploration and exploitation. Generally speaking, in a multi-armed bandit problem, to obtain a higher reward, the agent must choose the optimal action in various states based on previous experience (\textit{exploit}) known actions to obtain a higher score; to discover these actions, the necessary discovery is required (\textit{exploration}). Due to the superior performance and low feedback learning without the learning to act in multiple situations, multi-armed bandits are drawing widespread attention in applications ranging from recommender systems. Likewise, within the recommender system, collaborative filtering (CF) is arguably the earliest and most influential method in the recommender system. The meaning of collaboration is to filter the information through the relationship between the users and the feedback of the user's rating of the items together to find the target users’ preferences. Crucially, new users and an ever-changing pool of recommended items are the challenges that recommender systems need to address. For collaborative filtering, the classical method is to train the model offline, then perform the online testing, but this approach can no longer handle the dynamic changes in user preferences, which is the so-called \textit{cold start}. So, how to effectively recommend items to users in the absence of effective information?

To address the aforementioned problems, a multi-armed bandit based collaborative filtering recommender system has been proposed, named BanditMF. BanditMF is designed to address two challenges in the multi-armed bandits algorithm and collaborative filtering: (1) how to solve the cold start problem for collaborative filtering under the condition of scarcity of valid information, (2) how to solve the sub-optimal problem of bandit algorithms in strong social relations domains caused by independently estimating unknown parameters associated with each user and ignoring correlations between users.

version https://git-lfs.github.com/spec/v1
oid sha256:f7f279fa0f93cb2842457a52f2e0361e29261eb78732a2e3ff4c6aebddfefb19
size 6684

version https://git-lfs.github.com/spec/v1
oid sha256:773f8cd61d0a6d6593bcf6a674cb40fe521472ae453fc90bb9522717b1539912
size 18018

version https://git-lfs.github.com/spec/v1
oid sha256:cb323d9fad3db6daf5d284e3e48c0f8f296e1bed741ca4d7f3258f66a71e2940
size 7395

version https://git-lfs.github.com/spec/v1
oid sha256:1bb624c6d850fa3fd5805fa6f9c266eaac79da2d816e27d80c36752dd05e0386
size 7783

\section{Conclusion}

Time series forecasting is an important business and research problem that has a broad impact in today's world. This paper proposes a novel Y-shaped architecture, specifically designed for the far horizon time series forecasting problem. The study shows the importance of direct connections from the multi-resolution encoder to the decoder and reconstruction loss for the task of time series forecasting. The Yformer couples the U-Net architecture from the image segmentation domain on a sparse transformer model and empirically demonstrates superior performance across multiple datasets for both univariate and multivariate settings. We believe that our work provides a base for future research in the direction of using efficient U-Net based skip connections and the use of reconstruction loss as an auxiliary loss within the time series forecasting community.

\subsubsection{Acknowledgements}: This work was supported by the Federal Ministry for Economic Affairs and Climate Action (BMWK), Germany, within the framework of the IIP-Ecosphere project (project number: 01MK20006D)

% % Acknowledgements should go at the end, before appendices and references

% \acks{We would like to acknowledge support for this project
% from the National Science Foundation (NSF grant IIS-9988642)
% and the Multidisciplinary Research Program of the Department
% of Defense (MURI N00014-00-1-0637). }

% % Manual newpage inserted to improve layout of sample file - not
% % needed in general before appendices/bibliography.

\newpage

% \input{supp}

% % Note: in this sample, the section number is hard-coded in. Following
% % proper LaTeX conventions, it should properly be coded as a reference:

% %In this appendix we prove the following theorem from
% %Section~\ref{sec:textree-generalization}:

% In this appendix we prove the following theorem from
% Section~6.2:

% \noindent
% {\bf Theorem} {\it Let $u,v,w$ be discrete variables such that $v, w$ do
% not co-occur with $u$ (i.e., $u\neq0\;\Rightarrow \;v=w=0$ in a given
% dataset $\dataset$). Let $N_{v0},N_{w0}$ be the number of data points for
% which $v=0, w=0$ respectively, and let $I_{uv},I_{uw}$ be the
% respective empirical mutual information values based on the sample
% $\dataset$. Then
% \[
% 	N_{v0} \;>\; N_{w0}\;\;\Rightarrow\;\;I_{uv} \;\leq\;I_{uw}
% \]
% with equality only if $u$ is identically 0.} \hfill\BlackBox

% \noindent
% {\bf Proof}. We use the notation:
% \[
% P_v(i) \;=\;\frac{N_v^i}{N},\;\;\;i \neq 0;\;\;\;
% P_{v0}\;\equiv\;P_v(0)\; = \;1 - \sum_{i\neq 0}P_v(i).
% \]
% These values represent the (empirical) probabilities of $v$
% taking value $i\neq 0$ and 0 respectively.  Entropies will be denoted
% by $H$. We aim to show that $\fracpartial{I_{uv}}{P_{v0}} < 0$....\\

% {\noindent \em Remainder omitted in this sample. See http://www.jmlr.org/papers/ for full paper.}


% \vskip 0.2in
% \bibliography{sample}

% \bibliographystyle{apalike}
\bibliography{AMAML}

\end{document}