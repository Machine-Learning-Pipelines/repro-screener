\section{The \algOnefullname Algorithm} \label{sec:atg}
\begin{algorithm}[t]
    \caption{The \adapst Algorithm}
    \label{alg:atg}
    \begin{algorithmic}[1]
      \Procedure{AST}{$f, \mathcal{N}, k, \epsi$}
      \State \textbf{Input:} evaluation oracle $f:2^{\mathcal N} \to \reals$, constraint $k$,
      accuracy parameter $\epsi > 0$
      \State Initialize $M \gets \max_{x \in \mathcal N} f(x)$; 
      $c \gets 4 + \alpha$, where $\alpha^{-1}$ is ratio of \unc;
      $\ell \gets \lceil\log_{1 - \epsi}(1/(ck)))\rceil$
      \For{ $i \gets 0$ to $\ell$ in parallel}\label{line:forconcurrent}
      \State $\tau_i \gets M \left( 1 - \epsi \right)^i$
      \State $A_i,A_i' \gets \threseq \left( f, k, \tau_i, \epsi, 1 / 2 \right) $
      \State $B_i,B_i' \gets \threseq \left( f\restriction_{\mathcal N\setminus A_i}, k, \tau_i, \epsi, 1 / 2 \right)$
      \State $A_i'' \gets \unc (A_i)$
      \State $C_i \gets \argmax \{ f(A_i'), f(B_i'), f(A_i'') \}$
      \EndFor
      \State \textbf{return} $C \gets \argmax_i \{ f(C_i) \}$
      \EndProcedure
\end{algorithmic}
\end{algorithm}
In this section, we present the simple algorithm \algOnefullname 
(\atg, Alg.~\ref{alg:atg}) and show it obtains a ratio of $1/6 - \epsi$ 
with nearly optimal query and adaptive complexity.
This algorithm relies on running \threseq for a suitably chosen
threshold value. A procedure for
unconstrained maximization is also required.
% which uses as subroutines \thresam and an $(1/\alpha )$-approximation 
%algorithm \unc for the unconstrained maximization problem.

\textbf{Overview of Algorithm.}
Algorithm \atg works as follows.
First, the \textbf{for} loop 
guesses a value of $\tau$ close to
$\frac{\opt}{(4 + \alpha)k}$, where 
$1/\alpha$ is the ratio of the algorithm used for the unconstrained maximization
subproblem. 
Next, \threseq is called with parameter $\tau$ to yield set $A$ and $A'$;
followed by a second call to \threseq with $f$ restricted to $\mathcal N \setminus A$
to yield set $B$ and $B'$. Next, an unconstrained
maximization is performed with $f$ restricted to $A$ to yield
set $A''$;
finally, the best of the three candidate sets $A',B',A''$ is returned.

% Our algorithm \atg may be regarded as a simplification
% of the algorithm \iter of \cite{Gupta2010}, with the
% standard greedy procedure of \cite{Gupta2010} replaced
% by an adaptive threshold procedure and $O(\log k)$ guesses
% for \opt. It is surprising
% that this simplification can achieve nearly the same
% ratio of $1/6$ as \iter.
%We remark that \thresam could be replaced with any
%procedure that guarantees the average marginal
%gain of each addition is at least $\tau$.
% \todo{finish editing}
% The \thresam algorithm %of \cite{Fahrbach2018} 
% is used
% to adaptively add elements with gain at least $\tau$.
% The $(1/\alpha )$-approximation algorithm for \unc is called upon $A$. 
% The best solution $C_i$ from the two threshold procedures and 
% the \unc algorithm is obtained, and the algorithm returns the
% best $C_i$ found in any iteration of the \textbf{for} loop.
% Pseudocode for \atg is provided in Algorithm~\ref{alg:atg}. 
We prove the following theorem concerning the performance of \atg.

\begin{theorem}
  \label{thm:atg}
  % \begin{theorem}
  Suppose there exists an $(1/\alpha )$-approximation for
  \unc with adaptivity $\Theta$ and query complexity
  $\Xi$, and let $\epsi > 0$, $c = 4+\alpha$.
  Then there exists an algorithm for \sm with
  expected approximation ratio $c^{-1}-\epsi$ with probability
  at least $1 - 1/n$, expected query complexity
  $\oh{ \log_{1 - \epsi}(1/(ck)) \cdot \left( n / \epsi + \Xi \right) }$,
  and adaptivity $\oh{\log( n) / \epsi + \Theta }$.
\end{theorem}
\noindent If the algorithm of \shortciteS{Chen2018b} is used for \unc,
\atg achieves ratio $1/6 - \epsi$
with adaptive complexity $O\left( \log( n) / \epsi + \log(1 / \epsi) / \epsi \right)$ 
and query complexity
$O \left( \log_{1 - \epsi}(1/(6k)) \cdot \left( n / \epsi + n \log^3(1 / \epsi) / \epsi^4 \right) \right)$.

\textbf{Overview of Proof.}
The proof uses the following strategy: either \threseq finds
a set $A'$ or $B'$ with value $\approx \tau k$,
which is sufficient to achieve the ratio, or we
have two disjoint sets $A$, $B$ of size less than
$k$, such that for any $x \not \in A \cup B$, $\marge{x}{A} < \tau$
and $\marge{x}{B} < \tau$. 
In this case, for any set $O$, we have
by submodularity, $f(O) \le f(O \cap A) + f(O \setminus A )$.
The first term is bounded by the unconstrained maximization, and
the second term is bounded by an application of submodularity and the
fact that the maximum marginal gain of adding an element into $A$ or $B$ is below $\tau$.
The choice of constant $c$ balances the trade-off between the two
cases of the proof. 

%$1/6 - \epsi$ if a $(1/2 - \epsi)$-approximation is used for \unc.
% $A$ and $B$, such that $A \cap B = \emptyset$. 
% First, the algorithm guesses the value $\tau = OPT / 6k$
% in parallel in logarithmically iterations. Within
% each iteration, the algorithm
% ensures that (i) elements added to $A$ or $B$ have 
% marginal gain at least $\tau$
% in expectation, and (ii) any element not added to $A$ or $B$
% has marginal gain less than $\tau$. To accomplish
% (i) and (ii), the algorithm employs \thresam. 
% Next, an approximation algorithm for \unc is called on $A$ to
% obtain $A'$,
% and a set in $\{A', A, B\}$ with maximum $f$
% value is returned. 

%Hence if $|A| = k$ or $|B| = k$, the solution has
%expected value at least $OPT / 6$. Moreover, if both
\begin{proof}[Proof of Theorem~\ref{thm:atg}]
  Let $(f,k)$ be an instance of \sm, and let $\epsi> 0$.
  Suppose algorithm \atg uses a procedure for \unc with
  expected ratio $1/\alpha$.
  We will show that the
  set $C$ returned by
  algorithm $\atg(f,k,\epsi)$ satisfies
  $\ex{f(C)} \ge \left( c^{-1}-\epsi \right) \opt$
  with probability at
  least $(1 - 1/n)$, where $\opt$
  is the optimal solution value on the instance $(f,k)$.

  Observe that $\tau_0 = M = \max_{x \in \mathcal N} f(x) \ge \opt / k$ 
  by submodularity of $f$;
  $\tau_\ell = M(1 - \epsi)^\ell \le \opt / (ck)$ 
  since $M \le \opt$. 
  Because $\tau$ decreases by a factor of $1-\epsi$, 
  there exists $i_0$ such that 
  $ \frac{(1 - \epsi)\opt}{ck} \le \tau_{i_0} \le \frac{\opt}{ck}$.
  Let $A,A',B,B',A''$ denote 
  $A_{i_0},A'_{i_0},B_{i_0},B'_{i_0},A''_{i_0}$, respectively.
  For the rest of the proof, we assume that
  the properties of Theorem~\ref{thm:threshold} hold for the calls to \threseq
  with threshold $\tau_{i_0}$, which happens with at least
  probability $1 - 1/(2n)$ by the union bound. 
  
  \textbf{Case $|A| = k$ or $|B| = k$}.
  %  Consider the case that there exists 
  Suppose that $|A|=k$ without loss of generality.
  By Theorem~\ref{thm:threshold} and the value of $\tau_{i_0}$,
  it holds that,
  \begin{align*}
    f(A') \ge (1-\epsi) \tau_{i_0} |A| \ge 
    \frac{(1-\epsi)^2\opt}{c} \ge (1/c-\epsi)\opt.
  \end{align*}
  Then $f(C) \ge f(A') \ge (1/c - \epsi) \opt$.

  \textbf{Case $|A| < k$ and $|B| < k$}.
  Let $O$ be a set such that $f(O) = \opt$ 
  and $|O| \le k$. 
  Since $|A| < k$, by Theorem~\ref{thm:threshold},
  it holds that for any $x \in \mathcal N$,
  $\marge{x}{A} < \tau_{i_0}$. 
  Similarly, for any $x \in \mathcal N \setminus A$,
  $\marge{x}{B} < \tau_{i_0}$. 
  Hence, by submodularity,
  \begin{equation}
    \numberthis
    \label{ineq:1}
    f(O \cup A) - f(A) \le \sum_{o \in O} \marge{o}{A}
    < k \tau_{i_0} \nonumber \le \opt / c, 
  \end{equation}
  \begin{equation} 
    \numberthis{}
    \label{ineq:2}
    f( (O \setminus A) \cup B ) - f(B) 
    \le \sum_{o \in O \setminus A} \marge{o}{B} < 
    k \tau_{i_0} \nonumber \le \opt / c. 
  \end{equation}
  Next, from (\ref{ineq:1}), (\ref{ineq:2}), submodularity, nonnegativity, 
  Theorem~\ref{thm:threshold}, and the fact that $A \cap B = \emptyset$, 
  it holds that,
  \begin{align} \label{ineq:3}
    f(A') + f(B') &\ge f(A) + f(B) \nonumber \\ 
    &\ge f(O \cup A) + f( (O \setminus A) \cup B ) - 2\opt/c \nonumber \\ 
    &\ge f(O \setminus A) + f( O \cup A \cup B)- 2\opt/c \nonumber \\ 
    &\ge f( O \setminus A )- 2\opt/c.
  \end{align}
  %From Theorem~\ref{lemm:unc}, we have
  % \begin{equation} \label{ineq:4}
  %   \frac{2}{1 - \epsi}\ex{f(A')} \ge f( O \cap A )
  % \end{equation}
Since \unc is an $\alpha$-approximation, we have
\begin{equation} \label{ineq:4}
  \alpha \ex{f(A'')} \ge f( O \cap A ).
\end{equation}
  
From Inequalities (\ref{ineq:3}), (\ref{ineq:4}),
and submodularity,
  we have
  \begin{align*} 
    \opt = f(O) &\le f( O \cap A ) + f( O \setminus A ) \\ 
    &\le \alpha \ex{f(C)} + 2f(C) + 2\opt / c,
  \end{align*}
  from which it follows that
  $\ex{ f(C) } \ge \opt / c.$

  \textbf{Adaptive and query complexities}.
  The adaptivity of \atg is twice the adaptivity of \threseq
  plus the adaptivity of \unc plus a constant. 
  Further, the total query complexity is $\log_{1 - \epsi}(1/(ck))$ times
  the sum of twice the query complexity of \threseq and the query complexity
  of \unc.
  % $O \left( \log_{1 - \epsi}(1/(6k)) \cdot \left( n / \epsi + n \log^3(1 / \epsi) / \epsi^4 \right) \right)$.
%  Thus, it is bounded by
%  $O\left( \log( n / \delta ) / \epsi + \log(1 / \epsi) / \epsi \right)$.
\end{proof}

%%% Local Variables:
%%% mode: latex
%%% TeX-master: "main"
%%% End:
