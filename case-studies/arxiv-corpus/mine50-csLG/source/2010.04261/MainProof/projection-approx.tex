\subsubsection{Projecting Hessians onto Finite Dimensions}
\label{sec:proof-project}
In this section we will develop some technical tools for analyzing the eigenvalues and eigenvectors of the output Hessians and the full layer-wise Hessians. In particular, we will project both infinite dimensional matrices to $c\times c$ matrices.

First, we prove a technical lemma that will be very useful when we bound the Frobenius norm of the difference between infinite size matrices.
\begin{lemma}
\label{lemma:polynomial}
Let $p(\rmA,\rmD,\rvx)$ be a homogeneous polynomial of $\rmA$, $\rmD$, and $\rvx$ and is degree 1 in $\rmA$, degree 2 in $\rmD$, and degree 2 in $\rvx$. Suppose the coefficients in $p$ are upper bounded in $\ell_1$-norm by an absolute constant $\mu$. Also let $\rmD'$ be an independent copy of $\rmD$ and $\rvx''$ be an independent copy of $\rvx$ independent to $\rmD$ and $\rmA$. Then with probability 1 over $\mW^{(1)}$ and $\mW^{(2)}$, we have
\begin{equation}
    \lim_{n\to\infty}\ex{p(\rmA, \rmD, \rvx)} = \ex{p(\rmA, \rmD', \rvx'')}
\end{equation}
\end{lemma}
\begin{proofof}{\lemmaref{lemma:polynomial}}
Fix any $\eps>0$.
Assume that the homogeneous polynomial is of the form
\begin{equation}
p(\rmA,\rmD,\rvx)=\sum_{i=1}^m c_i\rmA_{s(i),t(i)}\rmD_{u(i),u(i)}\rmD_{v(i),v(i)}\rvx_{p(i)}\rvx_{q(i)},
\end{equation}
for coefficients $c_i$, then from linearity of expectation we know
    
\begin{equation}
    \E[p(\rmA,\rmD,\rvx)] = \sum_{i=1}^m c_i\E[\rmA_{s(i),t(i)}\rmD_{u(i),u(i)}\rmD_{v(i),v(i)}\rvx_{p(i)}\rvx_{q(i)}].
\end{equation}
Hence \begin{equation}
\begin{split}
&|\ex{p(\rmA, \rmD, \rvx)} - \ex{p(\rmA, \rmD', \rvx'')}|\\
\leq&\sum_{i=1}^m c_i |\E[\rmA_{s(i),t(i)}\rmD_{u(i),u(i)}\rmD_{v(i),v(i)}\rvx_{p(i)}\rvx_{q(i)}] - \E[\rmA_{s(i),t(i)}\rmD_{u(i),u(i)}\rmD_{v(i),v(i)}\rvx_{p(i)}\rvx_{q(i)}]|
\end{split}
\end{equation}

Since the entries of $\rmD$ can only be $0$ or $1$, we have
\begin{equation}
\begin{split}
&\E[\rmA_{s(i),t(i)}\rmD_{u(i),u(i)}\rmD_{v(i),v(i)}\rvx_{p(i)}\rvx_{q(i)}]\\
=& \pr{\rmD_{u(i),u(i)}=\rmD_{v(i),v(i)}=1} \E[\rmA_{s(i),t(i)}\rvx_{p(i)}\rvx_{q(i)}|\rmD_{u(i),u(i)}=\rmD_{v(i),v(i)}=1]\\
=&\ \frac14\E[\rmA_{s(i),t(i)}\rvx_{p(i)}\rvx_{q(i)}|\rmD_{u(i),u(i)}=\rmD_{v(i),v(i)}=1].
\end{split}
\end{equation}
The last equality holds since $\rvu$ converges in distribution to a spherical Gaussian, and its entry-wise activations $\rmD$ follows a $p=\frac12$ Bernoulli distribution.
Assume $\sum_{i=1}^m|c_i|\geq \mu$, that the $\ell_1$ norm of the coefficients is upper bounded by some constant $\mu$. Set $\eps'=\frac{\eps}{\mu}$. To prove this lemma it is sufficient to prove that each term of the polynomial are sufficiently small, namely, for any index,
\begin{equation}
\begin{split}
&|\E[\rmA_{s(i),t(i)}\rvx_{p(i)}\rvx_{q(i)}|\rmD_{u(i),u(i)}=\rmD_{v(i),v(i)}=1]\\&\quad-\E[\rmA_{s(i),t(i)}\rvx''_{p(i)}\rvx''_{q(i)}|\rmD'_{u(i),u(i)}=\rmD'_{v(i),v(i)}=1]| 
\\
&|\E[\rmA_{s(i),t(i)}\rvx_{p(i)}\rvx_{q(i)}|\rmD_{u(i),u(i)}=\rmD_{v(i),v(i)}=1]-\E[\rmA_{s(i),t(i)}\rvx''_{p(i)}\rvx''_{q(i)}]|<\eps'.
\end{split}
\end{equation}

Fix a set of index $s,t,p,q,u,v$, for simplicity of notation, we use the abbreviation $\ex{\rmA_{st}\rvx_p\rvx_q|_\rmD}$ to denote $\E[\rmA_{s(i),t(i)}\rvx_{p(i)}\rvx_{q(i)}|\rmD_{u(i),u(i)}=\rmD_{v(i),v(i)}=1]$. Since $\rvx$ is of rectified Gaussian with the covariance of the initial Gaussian distribution being the identity, $\rvx_p$ and $\rvx_q$ shares the same density function when $x>0$, namely $f(x)=\frac{1}{\sqrt{2\pi}}\exp(-x^2/2)$.
Note that \begin{equation}
\label{eqn:proof-xxexp}
    \iint\limits_{\R^+\times\R^+}xy\ f(x)f(y)\ dx\ dy = \ex{\rvx_i\rvx_j}=\ex{\rvx_i}\ex{\rvx_j} = \frac{1}{2\pi}.
\end{equation}

Fix some $\beta < \frac\alpha2$, we have
\begin{equation}
\label{eqn:poly-bound}
\begin{split}
    &|\ex{\rmA_{st}\rvx_p\rvx_q|_\rmD}-\ex{\rmA_{st}\rvx''_p\rvx''_q}|\\
    =&\labs{\quad\iint\limits_{\R^+\times\R^+} \ex{\rmA_{st}|_{\rmD, \rvx_p=x,\rvx_q=y}}xy\ f(x)f(y)\ dx\ dy - \iint\limits_{\R^+\times\R^+} \ex{\rmA_{st}}xy\ f(x)f(y)\ dx\ dy}\\
    \leq& \iint\limits_{\R^+\times\R^+} |\ex{\rmA_{st}|_{\rmD, \rvx_p=x,\rvx_q=y}}-\ex{\rmA_{st}}|xy\ f(x)f(y)\ dx\ dy\\
    =&\iint\limits_{[0,n^\beta]\times [0,n^\beta]} |\ex{\rmA_{st}|_{\rmD, \rvx_p=x,\rvx_q=y}}-\ex{\rmA_{st}}|xy\ f(x)f(y)\ dx\ dy\\
    &+ \iint\limits_{\R^+\times\R^+\backslash\rbr{[0,n^\beta]\times [0,n^\beta]}} |\ex{\rmA_{st}|_{\rmD, \rvx_p=x,\rvx_q=y}}-\ex{\rmA_{st}}|xy\ f(x)f(y)\ dx\ dy.
\end{split}
\end{equation}
From \corollaryref{cor:A-invariant} we have, for any indices $s,t$, for sufficiently large $n$, for any $(x,y)\in[0,n^\beta]\times [0,n^\beta]$,
\begin{equation}
    |\ex{\rmA_{st}|_{\rmD, \rvx_p=x,\rvx_q=y}}-\ex{\rmA_{st}}| < \eps'
\end{equation}
Thus
\begin{equation}
\begin{split}
    &\iint\limits_{[0,n^\beta]\times [0,n^\beta]} |\ex{\rmA_{st}|_{\rmD, \rvx_p=x,\rvx_q=y}}-\ex{\rmA_{st}}|xy\ f(x)f(y)\ dx\ dy\\
    \leq &\ \eps'\iint\limits_{[0,n^\beta]\times [0,n^\beta]} xy\ f(x)f(y)\ dx\ dy = \frac{\eps'}{2\pi}.
\end{split}
\end{equation}
Now we consider the other integral. First note that since $\rmA_{st}$ is either $\rvp_i-\rvp_i^2$ or $-\rvp_i\rvp_j$ for some $i,j$, and $\rvp_i,\rvp_j, \rvp_i+\rvp_j\in(0,1)$ as it is the output of the softmax function, we have $\rmA_{st}\in(-\frac14, \frac14)$. It follows that $|\ex{\rmA_{st}|_{\rmD, \rvx_p=x,\rvx_q=y}}-\ex{\rmA_{st}}|\leq\frac12$. Therefore
\begin{equation}
\begin{split}
&\iint\limits_{\R^+\times\R^+\backslash\rbr{[0,n^\beta]\times [0,n^\beta]}} |\ex{\rmA_{st}|_{\rmD, \rvx_p=x,\rvx_q=y}}-\ex{\rmA_{st}}|xy\ f(x)f(y)\ dx\ dy\\
\leq &\ \frac12\iint\limits_{\R^+\times\R^+\backslash\rbr{[0,n^\beta]\times [0,n^\beta]}}xy\ \frac{e^{-x^2/2}}{\sqrt{2\pi}}\frac{e^{-y^2/2}}{\sqrt{2\pi}}\ dx\ dy\\
\leq&\ \frac12\cdot\frac{1}{2\pi}\int_{n^\beta}^\infty e^{-x^2/2}x\ dx\int_{\R^+}e^{-y^2/2}y\ dy + \frac12\cdot\frac{1}{2\pi}\int_{\R^+} e^{-x^2/2}x\ dx\int_{n^\beta}^{\infty}e^{-y^2/2}y\ dy\\
=&\ \frac{1}{2\pi}e^{-n^{2\beta}},
\end{split}
\end{equation}
which decreases below $\eps'/2$ for sufficiently large $n$. As both terms in \equationref{eqn:poly-bound} are less than $\eps'/2$ as $n\to\infty$, we have $|\ex{\rmA_{st}\rvx_p\rvx_q|_\rmD}-\ex{\rmA_{st}\rvx''_p\rvx''_q}|<\eps'$. Which completes the proof of this lemma.
\end{proofof}

We then generalize this lemma for a degree ten homogeneous polynomial, in which the monomials are roughly multiplied with an independent copy of itself (except for $\rmA$).
\begin{corollary}
\label{cor:polynomial}
Let $p(\rmA,\rmD,\rvx,\bar\rmA,\bar\rmD,\bar\rvx)$ be a homogeneous polynomial of $\rmA,\rmD,\rvx,\bar\rmA,\bar\rmD$, and $\bar\rvx$. Let it be degree 1 in $\rmA$, $\bar\rmA$, degree 2 in $\rmD$, $\bar\rmD$, and degree 2 in $\rvx$,$\bar\rvx$. Suppose the coefficients in $p$ are upper bounded in $\ell_1$-norm by an absolute constant $\mu$. Also let $\rmD'$ be an independent copy of $\rmD$ and $\rvx''$ be an independent copy of $\rvx$ independent to $\rmD$ and $\rmA$. Morever let $(\bar\rmA,\bar\rmD,\bar\rvx,\bar\rmD',\bar\rvx'')$ be an independent copy of $(\rmA,\rmD,\rvx,\rmD',\rvx'')$. Then with probability 1 over $\mW^{(1)}$ and $\mW^{(2)}$, we have
\begin{equation}
    \lim_{n\to\infty}\ex{p(\rmA, \rmD, \rvx, \bar\rmA, \bar\rmD, \bar\rvx)} = \ex{p(\rmA, \rmD', \rvx'', \bar\rmA, \bar\rmD', \bar\rvx'')}.
\end{equation}
\end{corollary}

\begin{proofof}{\corollaryref{cor:polynomial}}
For simplicity of notations, denote $\ervs_{ijuvrs} = \rmA_{ij}\rmD_{vv}\rmD_{ww}\rvx_r\rvx_s$, $\ervs'_{ijuvrs} = \rmA_{ij}\rmD'_{vv}\rmD'_{ww}\rvx''_r\rvx''_s$. Similarly, denote $\ervt_{klpqtu}=\bar\rmA_{kl}\bar\rmD_{pp}\bar\rmD_{qq}\bar\rvx_t\bar\rvx_u$ and $\ervt'_{klpqtu}=\bar\rmA_{kl}\bar\rmD'_{pp}\bar\rmD'_{qq}\bar\rvx''_t\bar\rvx''_u$. As there is no confusion on indexing, we will also omit the subscripts and use $\ervs, \ervt$.

Fix any $\eps>0$, Following the argument of the proof of \lemmaref{lemma:polynomial}, it is sufficient to prove this corollary by showing for any indexing, \begin{equation}
\begin{split}
&|\E[\rmA_{ij}\bar\rmA_{kl}\rmD_{vv}\rmD_{ww}\bar\rmD_{pp}\bar\rmD_{qq}\rvx_r\rvx_s\bar\rvx_t\bar\rvx_u] - \E[\rmA_{ij}\bar\rmA_{kl}\rmD_{vv}'\rmD_{ww}'\bar\rmD_{pp}'\bar\rmD_{qq}'\rvx_r''\rvx_s''\bar\rvx_t''\bar\rvx_u'']|\\
=&\ |\E[\ervs\ervt] - \E[\ervs'\ervt']|< \frac\eps\mu.
\end{split}
\end{equation}
First note that since $|\rmA_{ij}| < \frac14$ and $|\rmD_{ii}| \leq 1$ for all $i,j$, we have \begin{equation}
\begin{split}
|\E[\ervs]| &= |\E[\rmA_{ij}\rmD_{vv}\rmD_{ww}\rvx_r\rvx_s]|
\leq \frac14|\E[\rvx_r\rvx_s]| = \frac{1}{8\pi}.
\end{split}
\end{equation} The same argument also applies to $\ervs', \ervt$, and $\ervt'$.
Also, by \lemmaref{lemma:polynomial}, for sufficiently large $n$ we have $|\E[\ervs] - \E[\ervs']|<\eps'$ and $|\E[\ervt] - \E[\ervt']|<\eps'$. Since by construction $\ervs$ and $\ervt$ are independent, we have
\begin{equation}
\begin{split}
|\E[\ervs\ervt] - \E[\ervs'\ervt']|&=|\E[\ervs]\E[\ervt] - \E[\ervs']\E[\ervt']|\\ &=|\E[\ervs]\E[\ervt] - \E[\ervs]\E[\ervt'] + \E[\ervs]\E[\ervt'] - \E[\ervs']\E[\ervt']|\\
&\leq |\E[\ervs]||\E[\ervt] - \E[\ervt']| + |\E[\ervt']||\E[\ervs] - \E[\ervs']|\\
&\leq \frac{1}{8\pi}\eps' + \frac{1}{8\pi}\eps' < \eps',
\end{split}
\end{equation}
which completes the proof of \corollaryref{cor:polynomial}.
\end{proofof}

Now we formally begin our analysis. We will start from $\mM^{(1)}=\ex{\rmD\mW^{(2)\T}\rmA\mW^{(2)}\rmD}$, the output Hessian of the first layer. The output Hessian of the second layer is just $\ex{\rmA}$, which had been analyzed in \sectionref{sec:proof-A-structure}. In this section we will neglect the superscript for $\mM^{(1)}$ and use $\mM$ as there is no confusion. Also, we use $\mW$ to denote $\mW^{(2)}$ unless specified otherwise. We first state our main lemma of projecting $\mM$. 

\begin{lemma}
\label{lemma:M-proj-preserve-f-norm}
With probability 1 over $\mW^{(1)}$ and $\mW^{(2)}$, \begin{equation}
    \lim_{n\to\infty}\frac{\fns{\mW\mM\mW^\T}}{\fns{\mM}}=1.
\end{equation}
\end{lemma}
\begin{proofof}{\lemmaref{lemma:M-proj-preserve-f-norm}}
To prove the equivalence between $\fns{\mW\mM\mW^\T}$ and $\fns{\mM}$, we need to introduce a bridging term 
\begin{equation}
    \mM^*\triangleq\E[\rmD'\mW^{(2)\T}\rmA\mW^{(2)}\rmD']
\end{equation}
where $\rmD'$ is an independent copy of $\rmD$ and also independent of $\rmA$. Essentially $\mM^*$ is the matrix which has the same expression as $\mM$ except that we assume $\rmD$ is independent of $\rmA$ in $\mM^*$. Informally, the proof strategy of \lemmaref{lemma:M-proj-preserve-f-norm} is\begin{equation}
    \fns{\mW\mM\mW^\T}\approx \fns{\mW\mM^*\mW^\T}\approx \fns{\mM^*}\approx \fns{\mM}.
\end{equation}
We now formally establish this equivalence.

Then we look into the structures of the bridging matrix $\mM^*$. It is simple to analyze as we assumed the independence between $\rmA$ and $\rmD'$. Formally,
\begin{lemma}
\label{lemma:M-star}
With probability 1 over $\mW^{(1)}$ and $\mW^{(2)}$,
\begin{equation}
\label{eqn:proof-M-star}
\mM^*=\frac14\rbr{\mW^\T\E[\rmA]\mW + \emph{diag}(\mW^\T\E[\rmA]\mW)}.
\end{equation}
Moreover, $\norm{\mM^*}$ and $\fns{\mM^*}$ are bounded below by some nonzero constant and bounded above by some constant.
\end{lemma}
\begin{proofof}{\lemmaref{lemma:M-star}}
First note that since $\rmD'$ is the activation of $\rvu'$, which converges to a spherical Gaussian with probability 1 over $\mW^{(1)}$ and is independent with $\rmA$, each diagonal entry of $\rmD$ is a Bernoulli random variable with $p=\frac12$.
For $i,j\in [n]$, when $i\neq j$, we have\begin{equation}
\begin{split}
\mM^*_{ij} &= \E[\rmD'_{ii}(\mW^\T\rmA\mW)_{ij}\rmD'_{jj}]\\
&= \E[\rmD'_{ii}]\E[\rmD'_{jj}]\E[(\mW^\T\rmA\mW)_{ij}]\\
&= \frac14(\mW^\T\E[\rmA]\mW)_{ij}.
\end{split}
\end{equation}
When $i=j$,
\begin{equation}
\begin{split}
\mM^*_{i,i} &= \E[\rmD'_{ii}(\mW^\T\rmA\mW)_{ii}\rmD'_{ii}]\\
&= \E[\rmD'_{ii}]\E[(\mW^\T\rmA\mW)_{ii}]\\
&= \frac12(\mW^\T\E[\rmA]\mW)_{i,j}.
\end{split}
\end{equation}
Thus\begin{equation}
    \mM^*=\frac14\rbr{\mW^\T\E[\rmA]\mW + \diag(\mW^\T\E[\rmA]\mW)}.
\end{equation}
Now we show the lower bound and upper bound on norms of $\mM^*$.

Since $\langle\E[\mW^\T\rmA\mW],\text{diag}(\E[\mW^\T\rmA\mW])\rangle\geq0$, we have
\begin{equation}
    \fn{\mM^*}\geq\fn{\E[\mW^\T\rmA\mW]} = \fn{\mW^\T\tilde{\rmA}\mW}.
\end{equation}
Since $\mW\mW^\T$ converges to $\mI_c$ in spectral norm from \lemmaref{lemma:WW-identity}, we have for sufficiently large $n$, the smallest singular value of $\mW$ is larger than $\frac12$. Moreover, since $\E[\rmA]$ admits an eigenvalue that is bounded below by some constants $\eta\triangleq \xi^c\cdot\rbr{\frac{\exp(-\gamma)}{c\exp(\gamma)}}^2/2c$ where $\xi\approx 0.68$ is an absolute constant and $\gamma=\frac{(\pi-1)^2}{4\pi^2}$ as shown in \lemmaref{lemma:A-rank-c-1}, there exists an eigenvalue of $\mM^* = \mW^\T\E[\rmA]\mW$ that is larger than $\frac{\eta}{4}$. Hence for large $n$, $\norm{\mM^*}$ is bounded from below by $\frac{\eta}{4}$, and hence $\fns{\mM^*}$.

Besides, since $\rmD$ is a diagonal matrix with 0/1 entries, and the absolute value of each entry of $\rmA$ is bounded by 1, we have
\begin{equation}
    \fn{\mM}=\fn{\E[\rmD\mW^\T\rmA\mW \rmD]}\leq\fn{\E[\mW^\T\rmA\mW]}\leq\fns{\mW}\fn{\rmA}\leq c\fns{\mW}.
\end{equation}
From \lemmaref{lemma:W-norm}, we know that with probability 1, $\fns{\mW}\leq 2c$, therefore, $\norm{\mM}_F$ is upper bounded by $2c^2$, which is independent of $n$.
\end{proofof}

\begin{lemma}
\label{lemma:M-equivalence}
With probability 1 over $\mW^{(1)}$ and $\mW^{(2)}$,
\[
\lim_{n\to\infty}\frac{\fns{\mM}}{\fns{\mM^*}}=1.
\]\end{lemma}
\begin{proofof}{\lemmaref{lemma:M-equivalence}}

Recall that $\mM^*\triangleq\E[\rmD'\mW^{(2)\T}\rmA\mW^{(2)}\rmD']$ where $\rmD'$ is an independent copy of $\rmD$ and also independent of $\rmA$.
Since we will only explicitly use $\mW^{(2)}$ in this proof, for simplicity of notation, we will omit its superscript and use $\mW$.
Let $(\bar{\rmD},\bar{\rmA})$ be an independent copy of $(\rmD,\rmA)$, then
\begin{equation}
\begin{split}
    \fns{\mM} &=\fns{\E[\rmD\mW^\T\rmA\mW\rmD]}\\
    &=\E\left[\langle \rmD\mW^\T\rmA\mW\rmD,\bar{\rmD}\mW^\T\bar{\rmA}\mW\bar{\rmD}\rangle\right]\\
    &=\E\left[\tr\left(\rmD\mW^\T\rmA\mW\rmD\bar{\rmD}\mW^\T\bar{\rmA}\mW\bar{\rmD}\right)\right]\\
    &=\E\left[\tr\left(\mW\bar{\rmD}\rmD\mW^\T\rmA\mW\rmD\bar{\rmD}\mW^\T\bar{\rmA}\right)\right].
\end{split}
\end{equation}
Expressing the term inside the expectation as a polynomial of entries of $\rmA$, $\rmD$, $\bar{\rmA}$ and $\bar{\rmD}$, we get
\begin{equation}
\label{eqn:proof-Mfnorm-polyexpression}
\begin{split}
     &\tr\left(\mW\bar{\rmD}\rmD\mW^\T\rmA\mW\rmD\bar{\rmD}\mW^\T\bar{\rmA}\right) \\
    =&\sum_{i=1}^c\left(\mW\bar{\rmD}\rmD\mW^\T\rmA\mW\rmD\bar{\rmD}\mW^\T\bar{\rmA}\right)_{i,i}\\
    =&\sum_{i,j=1}^c\rbr{\mW\bar{\rmD}\rmD\mW^\T\rmA}_{i,j}\rbr{\mW\rmD\bar{\rmD}\mW^\T\bar{\rmA}}_{j,i}\\
    =&\sum_{i,j=1}^c\rbr{\sum_{k=1}^c\sum_{l=1}^n\mW_{i,l}\mW_{k,l}\rmD_{l,l}\rmD_{l,l}\rmA_{k,j}}\rbr{\sum_{s=1}^c\sum_{t=1}^n\mW_{j,t}\mW_{s,t}\bar{\rmD}_{t,t}\bar{\rmD}_{t,t}\rmA_{s,i}}\\
    =&\sum_{i,j,k,s=1}^c\sum_{l,t=1}^n \mW_{i,l}\mW_{k,l}\mW_{j,t}\mW_{s,t}\bar{\rmA}_{k,j}\rmA_{s,i}\bar{\rmD}_{l,l}\rmD_{l,l}\bar{\rmD}_{t,t}\rmD_{t,t}.
\end{split}
\end{equation}
The monomials are $\bar{\rmA}_{k,j}\rmA_{s,i}\bar{\rmD}_{l,l}\rmD_{l,l}\bar{\rmD}_{t,t}\rmD_{t,t}$, and the corresponding coefficients are $\mW_{i,l}\mW_{k,l}\mW_{j,t}\mW_{s,t}$.
Now we can bound the $\ell_1$ norm of the coefficient of this polynomial as follows:

\begin{equation}
\label{eqn:proof-Mfnorm-poly-l1bound}
\begin{split}
     &\lnorm{\sum_{i,j,k,s=1}^c\sum_{l,t=1}^n \mW_{i,l}\mW_{k,l}\mW_{j,t}\mW_{s,t}}_1\\
    \leq &\sum_{i,j,k,s=1}^c\sum_{l,t=1}^n |\mW_{i,l}|\cdot|\mW_{k,l}|\cdot|\mW_{j,t}|\cdot|\mW_{s,t}|\\
    =&\rbr{\sum_{i,k=1}^c\sum_{l=1}^n|\mW_{i,l}|\cdot|\mW_{k,l}|}\rbr{\sum_{j,s=1}^c\sum_{t=1}^n|\mW_{j,t}|\cdot|\mW_{s,t}|}\\
    \leq &\rbr{\sum_{i,k=1}^c\sum_{l=1}^n\frac{\mW_{i,l}^2+\mW_{k,l}^2}{2}}\rbr{\sum_{j,s=1}^c\sum_{t=1}^n\frac{\mW_{j,t}^2+\mW_{s,t}^2}{2}}\\
    =&\rbr{\sum_{i,k=1}^c\frac{\ns{\mW_i}+\ns{\mW_k}}{2}}\rbr{\sum_{j,s=1}^c\frac{\ns{\mW_j}+\ns{\mW_s}}{2}}\\
    =&(c\fns{\mW})^2=c^2\fn{\mW}^4.
\end{split}
\end{equation}
From \lemmaref{lemma:W-norm} we know that $\fn{\mW}^2=O(c)$ with probability 1 over $\mW$, so the coefficient of this polynomial is $\ell_1$-norm bounded.

For any $\eps>0$, fix $\eps$. Note that $\fns{\mM^*}$ is just substituting $\rmD,\bar\rmD$ by $\rmD',\bar\rmD'$ in the polynomial characterized by \equationref{eqn:proof-Mfnorm-polyexpression}. From \corollaryref{cor:polynomial} we have the convergence of the difference of the expectation of the two polynomials, namely $|\fns{\mM} - \fns{\mM^*}| < \eps$ for sufficiently large $n$.
Since the spectral norm of $\mM^*$ is on the order of constant from \lemmaref{lemma:M-star}, we have $\lim_{n\to\infty} \fns{\mM}/\fns{\mM^*}=1.$
\end{proofof}

\begin{lemma}
\label{lemma:WMW-equivalence}
For all $i,j\in[c], \lim_{n\to\infty}((\mW\mM\mW^\T)_{i,j}-(\mW\mM^*\mW^\T)_{i,j})=0$. Thus, \[\lim_{n\to\infty}\frac{\fns{\mW\mM\mW^\T}}{\fns{\mW\mM^*\mW^\T}}=1.\]
\end{lemma}
\begin{proofof}{\lemmaref{lemma:WMW-equivalence}}
This proof is very similar to that of \lemmaref{lemma:M-equivalence}. First, we focus on a single entry of the matrix $\mW\mM\mW^\T$ and express it as a polynomial of entries of $\rmA$ and $\rmD$:
\begin{equation}
\label{eqn:proof-WMW-equiv-poly}
\begin{split}
(\mW\mM\mW^\T)_{i,j} &= \E[(\mW\rmD\mW^\T\rmA\mW\rmD\mW^\T)_{i,j}]\\
              &= \E\left[\sum_{k=1}^c(\mW\rmD\mW^\T\rmA)_{i,k}(\mW\rmD\mW^\T)_{k,j}\right]\\
              &= \E\left[\sum_{k=1}^c\rbr{\sum_{s=1}^c\sum_{l=1}^n\mW_{i,l}\mW_{s,l}\rmD_{l,l}\rmA_{s,k}}\rbr{\sum_{t=1}^n\mW_{k,j}\mW_{j,t}\rmD_{t,t}}\right]\\
              &= \E\left[\sum_{k,s=1}^c\sum_{l,t=1}^n\mW_{i,l}\mW_{s,l}\mW_{k,t}\mW_{j,t}\rmA_{s,k}\rmD_{l,l}\rmD_{t,t}\right].
\end{split}
\end{equation}
Then we bound the $\ell_1$ norm of the coefficients of this polynomial as follows:
\begin{equation}
\label{eqn:proof-WMW-equiv-l1bound}
\begin{split}
    &\lnorm{\sum_{k,s=1}^c\sum_{l,t=1}^n\mW_{i,l}\mW_{s,l}\mW_{k,t}\mW_{j,t}}_1\\
    \leq &\sum_{k,s=1}^c\sum_{l,t=1}^n|\mW_{i,l}|\cdot|\mW_{s,l}|\cdot|\mW_{k,t}|\cdot|\mW_{j,t}|\\
    =    &\rbr{\sum_{s=1}^c\sum_{l=1}^n|\mW_{i,l}|\cdot|\mW_{s,l}|}\rbr{\sum_{k=1}^c\sum_{t=1}^n|\mW_{k,t}|\cdot|\mW_{j,t}|}\\
    \leq &\rbr{\sum_{s=1}^c\sum_{l=1}^n\frac{\mW_{i,l}^2+\mW_{s,l}^2}{2}}\rbr{\sum_{k=1}^c\sum_{t=1}^n\frac{\mW_{k,t}^2+\mW_{j,t}^2}{2}}\\
    =    &\rbr{c\ns{\mW_i}+\fns{\mW}}\rbr{c\ns{\mW_j}+\fns{\mW}}\\
    \leq &(2c\fns{\mW})^2=4c^2\fn{\mW}^4.
\end{split}
\end{equation}
Similar to \lemmaref{lemma:M-equivalence}, this coefficient is $\ell_1$-norm bounded. 
The expression of each entry of $\mW\mM^*\mW^\T$ is just substituting $\rmD,\bar\rmD$ by $\rmD',\bar\rmD'$ in the polynomial characterized by \equationref{eqn:proof-WMW-equiv-poly}.
Therefore, using \lemmaref{lemma:polynomial}, we have with probability 1 over $\mW$, for all $i,j\in[c]$, 
\begin{equation}
    \lim_{n\to\infty}((\mW\mM\mW^\T)_{i,j}-(\mW\mM^*\mW^\T)_{i,j})=0.
\end{equation}
This completes the proof of the lemma as $\mW\mM\mW^\T$ is of constant size.
\end{proofof}


\begin{lemma}
\label{lemma:F-norm-equal}
With probability 1 over $\rmW^{(1)}$ and $\rmW^{(2)}$,
\begin{equation}
\lim_{n\to\infty}\frac{\fns{\mW\mM^*\mW^\T}}{\fns{\mM^*}}=1.
\end{equation}

\end{lemma}
\begin{proofof}{\lemmaref{lemma:F-norm-equal}}
The proof of this lemma will be divided into two parts. In the first part, we will estimate the Frobenius norm of $\mM^*$, and in the second part we do the same thing for $\mW\mM^*\mW^\T$.

\textbf{Part 1:} From \lemmaref{lemma:M-star} we know that
\begin{equation}
\mM^*=\frac14\rbr{\mW^\T\E[\rmA]\mW + \emph{diag}(\mW^\T\E[\rmA]\mW)}.
\end{equation}
Denote $\tilde{\rmA}\triangleq\E[\rmA]$, then
\begin{equation}
\E[\mW^\T\rmA\mW] = \mW^\T\E[\rmA]\mW = \mW^\T\tilde{\rmA}\mW.
\end{equation}
From \lemmaref{lemma:WW-identity}, for all $\eps'>0$, with probability 1 over $\rmW$ we have $\norm{\mW\mW^\T-\mI_c}\leq\eps'$. Besides, from \cite{kleinman1968design} we know that for positive semi-definite matrices $\mA$ and $\mB$ we have $\lambda_{\min}(\mA)\tr(\mB)\leq \tr(\mA\mB)\leq \lambda_{\max}(\mA)\tr(\mB)$, so
\begin{equation}
\begin{split}
    \bigg|\fns{\mW^\T\tilde{\rmA}\mW} - \fns{\tilde{\rmA}}\bigg|
    &=\Big|\tr(\mW^\T\tilde{\rmA}\mW\mW^\T\tilde{\rmA}\mW)-\tr(\tilde{\rmA}\tilde{\rmA})\Big|\\
    &=\Big|\tr(\mW\mW^\T\tilde{\rmA}\mW\mW^\T\tilde{\rmA})-\tr(\tilde{\rmA}\tilde{\rmA})\Big|\\
    &\leq\Big|\rbr{\norml{\mW\mW^\T-\mI_c}+1}\tr(\tilde{\rmA}\mW\mW^\T\tilde{\rmA})-\tr(\tilde{\rmA}\tilde{\rmA})\Big|\\
    &=\Big|\rbr{\norml{\mW\mW^\T-\mI_c}+1}\tr(\mW\mW^\T\tilde{\rmA}\tilde{\rmA})-\tr(\tilde{\rmA}\tilde{\rmA})\Big|\\
    &\leq\Big|\rbr{\norml{\mW\mW^\T-\mI_c}+1}^2\tr(\tilde{\rmA}\tilde{\rmA})-\tr(\tilde{\rmA}\tilde{\rmA})\Big|\\
    &\leq\norml{\mW\mW^\T-\mI_c}^2\fns{\tilde{A}} + 2\norml{\mW\mW^\T-\mI_c}\fns{\tilde{A}}.
\end{split}
\end{equation}
For any $\eps>0$, set $\eps'=\min\{\frac{\eps}{4},\frac{\sqrt{\eps}}{2}\}$ gives us with probability 1,
\begin{equation}
    \lim_{n\to\infty}\frac{\bigg|\fns{\mW^\T\tilde{\rmA}\mW} - \fns{\tilde{\rmA}}\bigg|}{\fns{\tilde{\rmA}}}=0,
\end{equation}
i.e.,
\begin{equation}
    \lim_{n\to\infty}\frac{\fns{\mW^\T\tilde{\rmA}\mW}}{\fns{\tilde{\rmA}}}=1.
\end{equation}
Besides, if we denote the $i$-th column of $\mW$ by $\vw_i$, then
\begin{equation}
\begin{split}
\fns{\diag(\E[\mW^\T\rmA\mW])} &= \sum_{i=1}^n (\vw_i^\T\tilde{\rmA}\vw_i)^2\\
&\leq \sum_{i=1}^n \rbr{\ns{\vw_i}\cdot\norml{\tilde{\rmA}}}^2\\
&= \ns{\tilde{\rmA}}\sum_{i=1}^n \norml{\vw_i}^4.
\end{split}
\end{equation}
Since $\E[n^2\norml{\vw_i}^4]=c^2+2c$, by the additive form of Chernoff bound we get
\begin{equation}
\pr{\sum_{i=1}^n\norml{\vw_i}^4\geq \frac{c^2+3c}{n}}=\pr{\frac{\sum_{i=1}^nn^2\norml{\vw_i}^4}{n}-(c^2+2c)\geq c}\leq e^{-2nc^2}.
\end{equation}
Therefore, when $n\to\infty$, with probability 1 over $\rmW$ we have
\begin{equation}
\fns{\diag(\E[\mW^\T\rmA\mW])}\leq \ns{\tilde{\rmA}}\sum_{i=1}^n \norml{\vw_i}^4\leq \ns{\tilde{\rmA}}\cdot\frac{c^2+3c}{n}.
\end{equation}
Thus, with probability 1 over $\rmW$,
\begin{equation}
\label{lemma:W-diag-neg}
\lim_{n\to\infty}\frac{\fns{\diag\left(\E\left[\mW^\T\rmA\mW\right]\right)}}{\fns{\mW^\T\tilde{\rmA}\mW}} = 0,
\end{equation}
i.e.,
\begin{equation}
\lim_{n\to\infty}\frac{\frac{1}{16}\fns{\tilde{\rmA}}}{\fns{\mM^*}} = 1.
\end{equation}

\textbf{Part 2:} We now estimate the norm of $\mW\mM^*\mW$. Plug equation \equationref{eqn:proof-M-star} into $\mW\mM^*\mW$ and we get
\begin{equation}
\mW\mM^*\mW = \frac14\rbr{\E[\mW\mW^\T\rmA\mW\mW^\T]+\E[\mW \diag(\mW^\T\rmA\mW)\mW^\T]}.
\end{equation}
Similar to \textbf{Part 1}, when $n\to\infty$, with probability 1, we have
\begin{equation}
\lim_{n\to\infty}\frac{\fns{\E[\mW\mW^\T\rmA\mW\mW^\T]}}{\fns{\tilde{\rmA}}} = 1.
\end{equation}
Besides, when $n\to\infty$, with probability 1 we have
\begin{equation}
\fns{\mW\diag(\E[\mW^\T\rmA\mW])\mW^\T}\leq \fns{\mW}\ns{\tilde{\rmA}}\sum_{i=1}^n \norml{\vw_i}^4\leq \ns{\tilde{\rmA}}\cdot\frac{c^2+3c}{n}\fns{\mW}.
\end{equation}
As a result, with probability 1,
\begin{equation}
\lim_{n\to\infty}\frac{\fns{\mW\diag(\E[\mW^\T\rmA\mW])\mW^\T}}{\fns{\mW\mW^\T\tilde{\rmA}\mW\mW^\T}} = 0,
\end{equation}
i.e.,
\begin{equation}
\lim_{n\to\infty}\frac{\frac{1}{16}\fns{\tilde{\rmA}}}{\fns{\mW\mM^*\mW^\T}} = 1.
\end{equation}
Combining the results of \textbf{Part 1} and \textbf{Part 2} proves this lemma.

\end{proofof}
Combining \lemmaref{lemma:M-equivalence}, \lemmaref{lemma:WMW-equivalence}, and \lemmaref{lemma:F-norm-equal} directly finishes the proof of \lemmaref{lemma:M-proj-preserve-f-norm}.
\end{proofof}

After establishing the projection of $\mM$ onto a $c\times c$ matrix, we may project the full layer-wise Hessian of the first layer, namely $\mH^{(1)}=\exs{\rmD\mW^{(2)\T}\rmA\mW^{(2)}\rmD\otimes \rvx\rvx^\T}$ onto a $c\times c$ matrix using very similar techniques. For simplicity of notation, we will denote $\mW^{(2)}$ by $\mW$ and $\mH^{(1)}$ by $\mH$ unless explicitly stated otherwise.

Since the autocorrelation matrix $\rvx\rvx$
has unbounded Frobenious norm, we will consider a re-scaled version $\tmH\triangleq \mH/d^2$ for our analysis. Let $\mU\triangleq\frac{1}{\sqrt{d}}\1_d^\T\in\R^{1\times d}$ be an all-1 matrix scaled by $\frac{1}{\sqrt{d}}$, we have $\mU\mU^\T=1$. Let $\mV\triangleq \mW\otimes\mU$ be our projection matrix for $\tmH$, we may then state our main lemma for full layer-wise Hessian.
\begin{lemma}
\label{lemma:H-proj-preserve-f-norm}
With probability 1 over $\mW^{(1)}$ and $\mW^{(2)}$, \begin{equation}
    \lim_{n\to\infty}\frac{\fns{\mV\tmH\mV^\T}}{\fns{\tmH}}=1.
\end{equation}
\end{lemma}

\begin{proofof}{\lemmaref{lemma:H-proj-preserve-f-norm}}
Similar to the proof for the output Hessian, we will introduce a ``bridging term''\begin{equation}
    \tmH^*\triangleq\frac{1}{d}\E[\rmD'\mW^{(2)\T}\rmA\mW^{(2)}\rmD'\otimes\rvx''\rvx''^\T]
\end{equation}
where $\rmD'$ is an independent copy of $\rmD$ and also independent of $\rmA$, and $\rvx''$ is an independent copy of $\rvx$ which is independent to both $\rmD'$ and $\rmA$.
Informally, we will show \begin{equation}
    \fns{\mV\tmH\mV^\T}\approx \fns{\mV\tmH^*\mV^\T}\approx \fns{\tmH^*}\approx \fns{\tmH}.
\end{equation}
We first look into the structures of $\tmH^*$.
\begin{lemma}
\label{lemma:H-star}
With probability 1 over $\mW^{(1)}$ and $\mW^{(2)}$,
\begin{equation}
\label{eqn:proof-H-star}
\tmH^*=\frac{1}{4d}\rbr{\mW^\T\E[\rmA]\mW + \emph{diag}(\mW^\T\E[\rmA]\mW)}\otimes \rbr{\frac{1}{2\pi}\1_d\1_d^\T+\frac{\pi - 1}{2\pi}\mI_d}.
\end{equation}
Moreover, for large $n$, $\eta/32<\fn{\tmH^*}<2c^2$. 
\end{lemma}
\begin{proofof}{\lemmaref{lemma:H-star}}
By independence in construction, we have $\tmH^* = \mM^*\otimes(\frac1d\E[\rvx''\rvx''^\T])$. Thus we only need to look into $\E[\rvx''\rvx''^\T]$.
For $i=j$, we have $\E[\rvx''\rvx''^\T]_{ii}=\E[\rvx_i\rvx_i]=\frac{1}{2}$ while for $i\neq j$, $\E[\rvx''\rvx''^\T]_{ij}=\E[\rvx_i\rvx_j]=\frac{1}{2\pi}$. Thus
\begin{equation}
\label{eqn:xxT-structure}
\E[\rvx''\rvx''^\T]=\frac{1}{2\pi}\1_d\1_d^\T+\frac{\pi - 1}{2\pi}\mI_d.
\end{equation}
It follows that\begin{equation}
\begin{split}
    \lim_{d\to\infty}\frac1d\fn{\E[\rvx\rvx^\T]} &= \lim_{d\to\infty}\frac1d\sqrt{d^2\frac{1}{4\pi^2} + d\frac{(\pi-1)^2}{4\pi^2}} = \frac{1}{2\pi} > \frac{1}{8}.
\end{split}
\end{equation}
Thus for large $n$ we have $\frac18<\frac1d\fn{\E[\rvx\rvx^\T]}<1$.
Since $\fn{\tmH^*} = \frac1d\fn{\mM^*\otimes \E[\rvx\rvx^\T]} = \fn{\mM^*}\cdot\frac1d\fn{\E[\rvx\rvx^\T]}$ and we know that $\frac{\eta}{4}<\fn{\tmH^*}<2c^2$ from \lemmaref{lemma:M-star}. We can conclude that for large $n$, $\eta/32<\fn{\tmH^*}<2c^2$.
\end{proofof}

\begin{lemma}
\label{lemma:H-equivalence} With probability 1 over $\mW^{(1)}$ and $\mW^{(2)}$,
\[
\lim_{n\to\infty}\frac{\fns{\tmH}}{\fns{\tmH^*}}=1.
\]\end{lemma}
\begin{proofof}{\lemmaref{lemma:H-equivalence}}
Unsurprisingly, this proof will be very similar to the proof of \lemmaref{lemma:M-equivalence}.
Recall that $\tmH^*\triangleq\frac1d\E[\rmD'\mW^\T\rmA\mW\rmD'\otimes\rvx''\rvx''^\T]$. Let $(\bar{\rmD},\bar{\rmA},\bar{\rvx})$ be an independent copy of $(\rmD,\rmA,\rvx)$,
\begin{equation}
\begin{split}
    \fns{\tmH} &=\lnorm{\frac1d\E[\rmD\mW^\T\rmA\mW\rmD\otimes \rvx\rvx^\T]}_F^2\\
    &=\E\left[\frac{1}{d^2}\langle \rmD\mW^\T\rmA\mW\rmD\otimes \rvx\rvx^\T,\bar{\rmD}\mW^\T\bar{\rmA}\mW\bar{\rmD}\otimes \bar\rvx\bar\rvx^\T\rangle\right]\\
    &=\ex{\frac{1}{d^2}\tr\left(\rbr{\rmD\mW^\T\rmA\mW\rmD\otimes\rvx\rvx^\T}\rbr{\bar{\rmD}\mW^\T\bar{\rmA}\mW\bar{\rmD}\otimes\bar\rvx\bar\rvx^\T}\right)}\\
    &=\ex{\frac{1}{d^2}\tr\rbr{\rmD\mW^\T\rmA\mW\rmD\bar{\rmD}\mW^\T\bar{\rmA}\mW\bar{\rmD}}\tr\rbr{\rvx\rvx^\T\bar\rvx\bar\rvx^\T}}\\
    &=\ex{\frac{1}{d^2}(\rvx^\T\bar\rvx\bar\rvx^\T\rvx)\tr\left(\mW\bar{\rmD}\rmD\mW^\T\rmA\mW\rmD\bar{\rmD}\mW^\T\bar{\rmA}\right)}.
\end{split}
\end{equation}
Expressing the term inside the expectation as a polynomial of entries of $\rmA$, $\rmD$, $\bar{\rmA}$ and $\bar{\rmD}$, we get
\begin{equation}
\label{eqn:proof-Mfnorm-polyexpression}
\begin{split}
     &\frac{1}{d^2}(\rvx^\T\bar\rvx\bar\rvx^\T\rvx)\tr\left(\mW\bar{\rmD}\rmD\mW^\T\rmA\mW\rmD\bar{\rmD}\mW^\T\bar{\rmA}\right) \\
    =&\frac{1}{d^2}\sum_{p,q=1}^d\rvx_p\bar\rvx_p\rvx_q\bar\rvx_q\rbr{\sum_{i=1}^c\left(\mW\bar{\rmD}\rmD\mW^\T\rmA\mW\rmD\bar{\rmD}\mW^\T\bar{\rmA}\right)_{i,i}}\\
    =&\frac{1}{d^2}\sum_{p,q=1}^d\sum_{i,j,k,s=1}^c\sum_{l,t=1}^n \mW_{i,l}\mW_{k,l}\mW_{j,t}\mW_{s,t}\bar{\rmA}_{k,j}\rmA_{s,i}\bar{\rmD}_{l,l}\rmD_{l,l}\bar{\rmD}_{t,t}\rmD_{t,t}\rvx_p\bar\rvx_p\rvx_q\bar\rvx_q.
\end{split}
\end{equation}
We skipped some derivations as they are identical to \equationref{eqn:proof-Mfnorm-polyexpression}.
The monomials are\\ $\bar{\rmA}_{k,j}\rmA_{s,i}\bar{\rmD}_{l,l}\rmD_{l,l}\bar{\rmD}_{t,t}\rmD_{t,t}\rvx_p\bar\rvx_p\rvx_q\bar\rvx_q$, and the corresponding coefficients are $\mW_{i,l}\mW_{k,l}\mW_{j,t}\mW_{s,t}$.
The $\ell_1$ norm of the coefficients is
\begin{equation}
\begin{split}
    \lnorm{\frac{1}{d^2}\sum_{p,q=1}^d\sum_{i,j,k,s=1}^c\sum_{l,t=1}^n \mW_{i,l}\mW_{k,l}\mW_{j,t}\mW_{s,t}}_1 = \lnorm{\sum_{i,j,k,s=1}^c\sum_{l,t=1}^n \mW_{i,l}\mW_{k,l}\mW_{j,t}\mW_{s,t}}_1,
\end{split}
\end{equation}
which we know is upper bounded by some constant with probability 1 over $\mW$ from \equationref{eqn:proof-Mfnorm-poly-l1bound}.

For any $\eps>0$, fix $\eps$. Note that $\fns{\tmH^*}$ is just substituting $(\rmD,\bar\rmD,\rvx,\bar\rvx)$ by $(\rmD',\bar\rmD',\rvx'',\bar\rvx'')$ in the polynomial characterized by \equationref{eqn:proof-Mfnorm-polyexpression}. From \corollaryref{cor:polynomial} we have the convergence of the difference of the expectation of the two polynomials, namely $|\fns{\tmH} - \fns{\tmH^*}| < \eps$ for sufficiently large $n$.
Since the spectral norm of $\tmH^*$ is bounded below from 0 by \lemmaref{lemma:M-star}, we have $\lim_{n\to\infty} \fns{\tmH}/\fns{\tmH^*}=1.$
\end{proofof}

\begin{lemma}
\label{lemma:VHV-equivalence}
For all $i,j\in[c], \lim_{n\to\infty}((\mV\tmH\mV^\T)_{i,j}-(\mV\tmH^*\mV^\T)_{i,j})=0$. Thus,
\begin{equation}
\lim_{n\to\infty}\frac{\fns{\mV\tmH\mV^\T}}{\fns{\mV\tmH^*\mV^\T}}=1.
\end{equation}
\end{lemma}
\begin{proofof}{\lemmaref{lemma:VHV-equivalence}}
This proof is very similar to that of \lemmaref{lemma:H-equivalence}. First, we focus on a single entry of the matrix $\mV\tmH\mV^\T$ and express it as a polynomial of entries of $\rmA$ and $\rmD$:
\begin{equation}
\label{eqn:proof-VHV-equiv-poly}
\begin{split}
(\mV\tmH\mV^\T)_{i,j} &= \ex{\rbr{(\mW\otimes \mU)\frac1d(\rmD\mW^\T\rmA\mW\rmD\otimes \rvx\rvx^\T)(\mW\otimes \mU)^\T}_{i,j}}\\
&= \ex{\frac1d\rbr{(\mW\rmD\mW^\T\rmA\mW\rmD\mW^\T)\otimes(\mU\rvx\rvx^\T \mU^\T)}_{i,j}}\\
&= \ex{\frac1d\cdot \frac{1}{d}(\1_d^\T\rvx\rvx^\T \1_d)\rbr{\mW\rmD\mW^\T\rmA\mW\rmD\mW^\T}_{i,j}}\\
&= \ex{\frac{1}{d^2}\rbr{\sum_{p,q=1}^d\rvx_p\rvx_q}\rbr{\sum_{k=1}^c(\mW\rmD\mW^\T\rmA)_{i,k}(\mW\rmD\mW^\T)_{k,j}}}\\
&= \ex{\frac{1}{d^2}\sum_{p,q=1}^d\sum_{k,s=1}^c\sum_{l,t=1}^n\mW_{i,l}\mW_{s,l}\mW_{k,t}\mW_{j,t}\rmA_{s,k}\rmD_{l,l}\rmD_{t,t}\rvx_p\rvx_q}.
\end{split}
\end{equation}
We skipped some derivations as they are identical to \equationref{eqn:proof-WMW-equiv-poly}.
The monomials are $\rmA_{s,k}\rmD_{l,l}\rmD_{t,t}\rvx_p\rvx_q$, and the corresponding coefficients are $\mW_{i,l}\mW_{s,l}\mW_{k,t}\mW_{j,t}$. Observe that the $\ell_1$ norm of the coefficients satisfies
\begin{equation}
    \lnorm{\frac{1}{d^2}\sum_{p,q=1}^d\sum_{k,s=1}^c\sum_{l,t=1}^n\mW_{i,l}\mW_{s,l}\mW_{k,t}\mW_{j,t}}_1 = \lnorm{\sum_{k,s=1}^c\sum_{l,t=1}^n\mW_{i,l}\mW_{s,l}\mW_{k,t}\mW_{j,t}}_1,
\end{equation}
which we know is bounded above by some constant from \equationref{eqn:proof-WMW-equiv-l1bound}.
Note that the expression of each entry of $\mV\tmH^*\mW^\T$ is just substituting $(\rmD,\bar\rmD,\rvx,\bar\rvx)$ by $(\rmD',\bar\rmD',\rvx'',\bar\rvx'')$ in the polynomial characterized by \equationref{eqn:proof-VHV-equiv-poly}.
Therefore, using \lemmaref{lemma:polynomial}, we have with probability 1 over $\mW$, for all $i,j\in[c]$, 
\begin{equation}
    \lim_{n\to\infty}((\mV\tmH\mV^\T)_{i,j}-(\mV\tmH^*\mV^\T)_{i,j})=0.
\end{equation}
This completes the proof of the lemma as $\mV\tmH\mV^\T$ is of constant size.
\end{proofof}

\begin{lemma}
\label{lemma:F-norm-equal-H}
With probability 1 over $\rmW^{(1)}$ and $\rmW^{(2)}$,
\begin{equation}
\lim_{n\to\infty}\frac{\fns{\mV\tmH^*\mV^\T}}{\fns{\tmH^*}}=1.
\end{equation}
\end{lemma}

\begin{proofof}{\lemmaref{lemma:F-norm-equal-H}}
This lemma is a direct corollary of \lemmaref{lemma:F-norm-equal} for the output Hessian. Note that by the independence in construction,\begin{equation}
\begin{split}
\mV\tmH^*\mV^\T &= \frac{1}{d}\rbr{\mW\otimes\mU}\ex{\rmD'\mW^\T\rmA\mW\rmD\otimes\rvx''\rvx''^\T}(\mW^\T\otimes\mU^\T)\\
&= \frac{1}{d}\rbr{\mW\otimes\mU}\rbr{\mM^*\otimes\ex{\rvx''\rvx''^\T}}(\mW^\T\otimes\mU^\T)\\
&= \frac1d \rbr{\mW\mM^*\mW^\T}\otimes\rbr{\mU\ex{\rvx''\rvx''^\T}\mU^\T}\\
&= \rbr{\mW\mM^*\mW^\T}\otimes\rbr{\frac{1}{d^2}\1_d^\T\ex{\rvx''\rvx''^\T}\1_d}\\
&= \frac{1}{d^2}\1_d^\T\ex{\rvx''\rvx''^\T}\1_d\rbr{\mW\mM^*\mW^\T}.
\end{split}
\end{equation}
From \equationref{eqn:xxT-structure} we have
\begin{equation}
    \1_d^\T\ex{\rvx''\rvx''^\T}\1_d = \sum_{i,j=1}^d\ex{\rvx\rvx^\T}_{ij} = \frac{1}{2\pi}d^2 + \frac{\pi-1}{2\pi}d.
\end{equation}
Thus \begin{equation}
\begin{split}
    \lnorm{\mV\tmH^*\mV^\T}_F^2 &= \lnorm{\rbr{\frac{1}{2\pi} + \frac{\pi-1}{2\pi d}}\mW\mM^*\mW}_F^2\\
    &= \rbr{\frac{1}{4\pi^2} + \frac{\pi-1}{2\pi^2 d} + \frac{(\pi-1)^2}{4\pi^2d^2}}\fns{\mW\mM^*\mW}.
\end{split}
\end{equation}
Meanwhile note that \begin{equation}
    \fns{\tmH^*} = \frac{1}{d^2}\fns{\tmM^*\otimes\ex{\rvx''\rvx''^\T}} = \frac{1}{d^2}\fns{\tmM^*}\otimes\fns{\ex{\rvx''\rvx''^\T}},
\end{equation}
where \begin{equation}
    \fns{\ex{\rvx''\rvx''^\T}} = \sum_{i,j=1}^d\ex{\rvx\rvx^\T}_{ij}^2 = \frac{1}{4\pi^2}d^2 + \frac{\pi-1}{2\pi}d.
\end{equation}
Thus\begin{equation}
    \fns{\tmH^*} = \rbr{\frac{1}{4\pi^2} + \frac{\pi-1}{2\pi d}}\fns{\tmM^*}.
\end{equation}
Since $d=n^{1+\alpha}$ for some constant $\alpha>0$, we have 
\begin{equation}
    \lim_{n\to\infty}\frac{\frac{1}{4\pi^2} + \frac{\pi-1}{2\pi^2 d} + \frac{(\pi-1)^2}{4\pi^2d^2}}{\frac{1}{4\pi^2} + \frac{\pi-1}{2\pi d}} = 1.
\end{equation}
Thus combined with the result from \lemmaref{lemma:F-norm-equal}, we have\begin{equation}
    \lim_{n\to\infty}\frac{\fns{\mV\tmH^*\mV^\T}}{\fns{\tmH^*}} = \rbr{\lim_{n\to\infty}\frac{\frac{1}{4\pi^2} + \frac{\pi-1}{2\pi^2 d} + \frac{(\pi-1)^2}{4\pi^2d^2}}{\frac{1}{4\pi^2} + \frac{\pi-1}{2\pi d}}}\rbr{\lim_{n\to\infty}\frac{\fns{\mW\mM^*\mW^\T}}{\fns{\mM^*}}} = 1.
\end{equation}
\end{proofof}

Combining \lemmaref{lemma:H-equivalence}, \lemmaref{lemma:VHV-equivalence}, and \lemmaref{lemma:F-norm-equal-H} completes the proof of \lemmaref{lemma:H-proj-preserve-f-norm}.
\end{proofof}

Now we are done with the lemmas and will proceed to the proof of the main theorems.