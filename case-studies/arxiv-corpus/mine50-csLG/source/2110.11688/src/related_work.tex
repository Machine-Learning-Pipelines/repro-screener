
\section{Related Work}
\label{sec:related-works}



\paragraph{DP-ERM.}
Differentially Private Empirical Risk Minimization was first studied by
\citet{chaudhuri2011Differentially}, using output perturbation (adding noise
to the solution of the non-private ERM problem) and objective perturbation
(adding noise to the ERM objective itself).
\citet{bassily2014Private} then proposed DP-SGD and proved its
near-optimality. %
\citet{wang2017Differentially} obtained faster convergence rates using a
DP version of the SVRG algorithm
\citep{johnson2013Accelerating,xiao2014Proximal}.
DP-SGD has
become the standard approach to DP-ERM.
In our work, we show that coordinate-wise updates can have lower sensitivity
than DP-SGD updates and propose a DP-CD algorithm achieving competitive
results.
A private variant of the Frank-Wolfe algorithm (DP-FW) was also
proposed to solve \emph{constrained} DP-ERM problems
\citep{talwar2015Nearly}.  Although these algorithms achieve a good
privacy-utility trade-off in theory, we are not aware of any empirical
evaluation.
DP-FW algorithms access gradients indirectly through a linear
optimization oracle over a constrained set.
Restricting to a constrained set is not necessary in DP-CD, allowing its use for a different family of problems.

\paragraph{DP-SCO.} Recent work has also studied algorithms and
utility guarantees for
stochastic convex optimization under differential privacy constraints, a
problem very similar to DP-ERM. \citet{bassily2019Private} \citep[following work
from][]{hardt2016Train,bassily2020Stability} extended results known
for DP-ERM to this setting, showing that the population risk of DP-SCO
is asymptotically equivalent to the one of non-private SCO. Efficient
algorithms for
DP-SCO
were proposed by \citet{feldman2020Private,wang2022Differentially},
and \citet{asi2021Private,bassily2021NonEuclidean} studied stochastic
variants of DP-FW. As detailed by
\citet{dwork2015Preserving,bassily2016Algorithmic,jung2021New} results
from DP-ERM can be converted to DP-SCO.




\paragraph{Coordinate descent.}
Coordinate descent (CD) algorithms have a long history in optimization.
\citet{Luo_Tseng1992,Tseng01,Tseng_Yun09} have shown convergence results for
(block) CD algorithms for nonsmooth optimization.
\citet{Nesterov12} later proved a global non-asymptotic $1/k$ convergence
rate for CD with random choice of coordinates for a convex, smooth objective.
Parallel, proximal variants were developed by
\citet{richtarik2014Iteration,fercoq2014Accelerated}, while
\citet{hanzely2018SEGA} further considered non-separable non-smooth parts.
\citet{shalev-shwartz2013Stochastic} introduced Dual CD algorithms
for smooth ERM, showing performance similar to SVRG.
We refer to \citet{wright2015Coordinate} and \citet{shi2017Primer} for
detailed reviews on CD.
Inexact CD was studied by \citet{tappenden2016Inexact}, but their analysis
requires updates not to increase the objective, which is hardly compatible
with DP.
We obtain tighter results for inexact CD with noisy gradients
(see Remark~\ref{rmq:improvement-inexact-coordinate-descent}).



\paragraph{Private coordinate descent.}
\citet{damaskinos2021Differentially} introduced a CD method to privately solve
the dual problem associated with generalized linear models with $\ell_2$
regularization. Dual CD is tightly related to SGD, as each
coordinate in the dual is associated with one data point.
The authors briefly mention the possibility of performing primal coordinate
descent but discard it on account of the seemingly large sensitivity of its
updates.
We show that primal DP-CD is in fact quite effective, and can be used to solve more general problems than considered by \citet{damaskinos2021Differentially}.
Primal CD was successfully used by \citet{bellet2018Personalized} to privately learn personalized models from decentralized datasets.
For the smooth objective they consider,
each coordinate depends only on a subset of the full dataset, which directly
yields low coordinate-wise sensitivity updates.
In contrast, we introduce a general algorithm for composite DP-ERM, for which
a novel utility analysis was required.

