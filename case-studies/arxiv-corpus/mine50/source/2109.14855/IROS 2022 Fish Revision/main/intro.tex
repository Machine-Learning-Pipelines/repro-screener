\section{Introduction}

Soft robots are favored over traditional hard robots in a growing list of scenarios, including bio-inspired design, conformal gripping, co-bot situations, and more. While the design and simulation of hard robots are well understood and extensively shown thanks to decades of development, the same simulation task for compliant structures and their interaction with the environment is an ongoing challenge. 

Although many simulation schemes exist, none have demonstrated sufficient sim2real matching through verifiable experiments for soft robots in a hydrodynamic situation such as a robotic fish. The high dimensionality of soft mechanisms and the complex physics of fluid-structure interactions (FSI) make this problem particularly difficult. One critical challenge in producing high accuracy simulations is the precise measurement and manual tuning needed to dial-in the material parameters of soft mechanisms. The problem has recently become more tractable because of advances in physics-based simulation, such as differentiability and data-driven approaches. 

\begin{figure}[tb]
    \centering
    \includegraphics[width=\linewidth]{figures/CAD.png}
    \caption{\textit{Left:} A model of a pneumatic fish tail (\emph{Nemo}). \textit{Middle:} The same fish tail inflated to \SI{250}{mbar} in simulation. \textit{Right:} The actual hardware of the fish tail in an experimental setup for measuring the deformation and the thrust of the fish in air and underwater. Our framework is capable of reproducing the buckling effect in simulation, a valuable result.}
    \label{fig:pneumatic_fish_tail}
    \vspace{-8pt}
\end{figure}

We present a differentiable simulation framework that can be used to accurately predict the deformation of a soft pneumatic actuator and identify its material properties through gradient-based optimization. Three designs of a pneumatically-actuated fish tail are used as benchmark (see \Cref{tab:fish}) since composite bi-morph structures are difficult to simulate due to the disparate Young's moduli and aspect ratios of the deformable body and stiffer spine. The deformable body of the fish tail is cast out of silicone elastomer and has a flexible spine made of an acetal plastic sheet in the center (see \Cref{fig:pneumatic_fish_tail}). .

Our framework is capable of learning physically-plausible material parameters, i.e., Young's modulus and Poisson's ratio, using only a quasistatic data set without extensive material testing. We demonstrate that after optimization we can produce accurate simulation results compared to dynamic data collected in our hardware setup to within millimeter accuracy or within $3\%$ max error normalized to the actuator length. The fast system identification is achieved using our finite element method (FEM)~\cite{du2021diffpd,ma2021diffaqua}, which combines projective dynamics~\cite{bouaziz2014projective} and differentiable simulation~\cite{hu2019difftaichi}.

To verify our simulation results, we developed an experimental setup for collecting position, pressure, and force data using marker tracking, a pneumatic valve array, and a load cell. The raw data is synchronized and analyzed in MATLAB.

Furthermore, we are able to reproduce the measured thrust in our bollard-pull type experiment using a simple neural network predictor that can be integrated in our differentiable simulator. Our model is capable of predicting thrust performance in actuation signals not seen in the training set and preserves differentiability of the simulator. Our aim is to provide straightforward sim2real methods for roboticists that are within reach: such methods could be used confidently for design optimization if they are verifiable by real data.

In this paper, we contribute:
\begin{itemize}
    \item a simulation framework for a soft pneumatic bending actuator consisting of disparate materials and geometries;
    \item a system identification method that uses differentiable simulation and gradient-based optimization to accurately learn material parameters of two isotropic corotated materials;
    \item a data-driven hydrodynamics model using a neural network as a simple predictor of the thrust force generated by the fish; and
    \item experimental verification of our simulation results using a hardware setup suitable for data acquisition in both air and water.
\end{itemize}
