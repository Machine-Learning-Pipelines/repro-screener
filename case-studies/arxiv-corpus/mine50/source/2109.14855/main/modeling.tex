\section{Physical Modeling}
\label{physical_modeling}

\subsection{Solid mechanics}
\label{solid_mechanics}
The deformation of a soft material can be modeled by the equations of elasticity. For the constitutive model we choose an isotropic corotated material~\cite{sifakis2012fem} since it is straightforward to implement for projective dynamics and can predict bending to within sufficient accuracy for engineering purposes.

A corotated material is defined by its energy density function using the Frobenius Norm and the trace operator,
\begin{equation}
    \Psi = \mu \left\lVert\mathbf{U} - \mathbf{I}\right\rVert^2_F + \frac{\lambda}{2}\text{tr}^2(\mathbf{U}-\mathbf{I}),
\end{equation}
where $\mathbf{U}$ is the stretch tensor, $\mathbf{I}$ is the identity, and $\mu$ and $\lambda$ are the Lam\'e parameters, which have the following relation to the Young's modulus $E$ and Poisson's ratio $\nu$ for an isotropic material,
\begin{equation}
    \mu = \frac{E}{2(1+\nu)}, \quad \lambda = \frac{E\nu}{(1+\nu)(1-2\nu)}.
\end{equation} This material energy model is popular for physical simulation and animation, but it is not the most accurate choice for closing the sim2real gap for large deformations~\cite{sifakis2012fem, adeeb2011introduction}. 

To simulate a pneumatically actuated fish tail, we use a spatially-constant time-varying pressure boundary condition on the interior surface of each chamber of the tail.

\subsection{Hydrodynamics}
\label{hydrodynamics}
The hydrodynamic effects that produce thrust and drag on a swimmer are important, however, the physics governing the fluid can be complicated due to turbulent effects and two-way coupling between the solid and fluid domains. One may choose to model water using the Navier-Stokes equations, however, in practice simulating the full FSI problem is under time constraints computationally unfeasible. For this reason, many workers in the field of computer graphics and simulation choose to use simplified heuristic hydrodynamic models such as the one presented by Min et al.~\cite{min2019softcon}. Analytical models for fish propulsion have also been studied extensively in the literature~\cite{lighthill1971large, wu1961swimming, triantafyllou2000hydrodynamics}, but these approximations require severe simplifying assumptions necessary for analytical tractability. Using the elongated-body theory (EBT) described in \Cref{sec:large_amplitude_ebt}~\cite{lighthill1971large}, the average thrust produced by a fish is approximated by
$f_\textrm{EBT} = m \hat{g}(\dot{x},\dot{y},y)$,
where $m$ is the virtual mass and $\hat{g}$ is a function of the velocity $\dot{x},\dot{y}$ and deformation $y$ of the fish tail measured in millimeters. We use EBT for comparison to our learned hydrodynamic model.

\subsection{Simulation}
\label{simulation}

\subsubsection{Discretization} 
We use our finite element method with tetrahedra to discretize the continuous spatial domain occupied by the soft fish tail. Furthermore, we implement the implicit Euler time-stepping scheme for time discretization. Formally, let $\mathbf{x}_i$ and $\mathbf{v}_i$ be vectors of size $3N$ stacking up the nodal positions and velocities from the tetrahedra at the $i$-th time step. We then rewrite the governing equations in the following discretized form
\begin{align}\label{eqn:discretization}
    \mathbf{x}_{i+1}=&\mathbf{x}_{i}+h\mathbf{v}_{i+1},\\
    \mathbf{v}_{i+1}=&\mathbf{v}_i+h\mathbf{M}^{-1}[\mathbf{f}_{ela}(\mathbf{x}_{i+1})+\mathbf{f}_{act}].
\end{align}
Here, $h$ is the time step, $M$ is the mass matrix, and $\mathbf{f}_{ela}$ and $\mathbf{f}_{act}$ represent the elastic force and the actuator force (computed by the pressure from the pneumatic actuator and the surface area of the chambers), respectively. The time integration is implicit because we make the elastic force depend on the nodal position $\mathbf{x}_{i+1}$ at the beginning of the next time step instead of the current one.

\subsubsection{Solver} 
Solving \Cref{eqn:discretization} typically requires expensive numerical computation due to its implicit time-stepping scheme. However, mature numerical tools exist since this problem has been extensively studied in computer graphics and physics simulation. In this work, we choose to use DiffPD~\cite{du2021diffpd}, a recent differentiable simulator with projective dynamics that is suitable for simulating terrestrial and underwater soft robots. We choose differentiable projective dynamics over other numerical solvers, for example traditional Newton's methods, due to its simultaneous accommodation of speed, robustness, and differentiability.
