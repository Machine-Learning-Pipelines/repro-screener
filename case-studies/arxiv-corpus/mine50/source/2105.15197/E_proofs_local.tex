\section{Proofs of lemmas}

\subsection{Riesz representers}

\begin{proof}[of Lemma~\ref{prop:RR}]
As the operator norm of $\gamma\mapsto E\{m(W,\gamma)\}$,
$$
\bar{M}=\inf [c\geq 0: |E\{m(W,\gamma)\}|\leq c  \text{ for all } \gamma \text{ in }\Gamma \text{ such that } \|\gamma\|_{\text{\normalfont pr,2}}=1].
$$
By Jensen's inequality and Assumption~\ref{assumption:cont},
$$
|E\{m(W,\gamma)\}| \leq [E \{m(W,\gamma)^2\}]^{1/2} \leq \left(\bar{Q} [E \{\gamma(W)^2\} ]^q\right)^{1/2} = \bar{Q}^{1/2} \|\gamma\|^q_{\text{\normalfont pr,2}}.
$$
Taking the supremum of both sides over $\gamma$ in $\Gamma$ such that $\|\gamma\|_{\text{\normalfont pr,2}}=1$, we conclude that $\bar{M}\leq \bar{Q}^{1/2}<\infty$. The rest of the claim is shown in \cite[Lemma 2.1]{chernozhukov2018global}.
\end{proof}

\begin{proof}[of Lemma~\ref{lemma:RR_exists}]
For Examples~\ref{ex:CATE} and~\ref{ex:RDD}, the result is immediate from standard propensity score and regression arguments. For Example~\ref{ex:elasticity}. the result follows from \cite[Proposition 3 and Example 5]{ichimura2021influence}. For Example~\ref{ex:deriv}, the result follows from integration by parts.
\end{proof}

\subsection{Global functionals}

\begin{proof}[of Lemma~\ref{lemma:global}]
We extend \cite[Lemma 3.3]{chernozhukov2018global}. Write
$$
\psi_0(W)=U_1+\alpha_0^{\min}(W)U_2.
$$
By law of iterated expectations,
\begin{align*}
    \sigma^2
&=E(U_1^2)+2E\{U_1\alpha_0^{\min}(W)U_2\}+ E(\{\alpha_0^{\min}(W)U_2\}^2) \\
&=E\{E(U_1^2 \mid W)\}+2E\{U_1\alpha_0^{\min}(W) E(U_2 \mid W)\}+ E\{\alpha_0^{\min}(W)^2 E(U_2^2 \mid W)\}.
\end{align*}
$E(U_2 \mid W)=0$ by definition of $U_2$. Note that
$$
0 \leq E\{E(U_1^2 \mid W)\} \leq \bar{c}^2,
$$
and
$$
\tilde{c}^2 \bar{M}^2 \leq E\{\alpha_0^{\min}(W)^2 E(U_2^2 \mid W)\} \leq \bar{c}^2\bar{M}^2.
$$
In summary
$$
\tilde{c}^2 \bar{M}^2 \leq \sigma^2 \leq \bar{c}^2 (1+\bar{M}^2).
$$
By triangle inequality,
\begin{align*}
   \|\psi_0\|_{\text{ \normalfont pr},q} &
   \leq \|U_1\|_{\text{ \normalfont pr},q} + \|\alpha_0^{\min}(W)U_2\|_{\text{ \normalfont pr},q}  \\
    &=\|U_1\|_{\text{ \normalfont pr},q} + [E\{\alpha_0^{\min}(W)^q E( U_2^q \mid W)\}]^{1/q} \\
    &\leq \bar{c}+ \bar{c} \|\alpha^{\min}_0\|_{\text{ \normalfont pr},q}   \\
    &\leq \bar{c}(1+c(\bar{M}^2\vee 1)).
\end{align*}
\end{proof}



\begin{proof}[of Lemma~\ref{lemma:rp}]
To begin, observe that
\begin{align*}
    \partial_d \{ f(d\mid x) \partial_d \gamma(d,x)\}
    &= \{\partial_df(d\mid x)\}\{\partial_d \gamma(d,x)\}+f(d\mid x) \{\partial^2_d\gamma(d,x)\}\\
    &= [\{\partial_d \log f(d \mid x)\}\{\partial_d \gamma(d,x)\}+ \partial^2_d\gamma(d,x)]f(d \mid x) \\
    &=-k_{\gamma}(d,x)f(d \mid x).
\end{align*}
Using integration by parts and the boundary condition together with this result,
\begin{align*}
    E [\{\partial_d \gamma(D,X)\}^2]
    &=\int \{\partial_d \gamma(d,x)\}^2 f(d \mid x) f(x) \mathrm{d}dx \\
    &=\int \{\partial_d \gamma(d,x)\} \{ f(d\mid x) \partial_d \gamma(d,x)\} f(x) \mathrm{d}dx \\
    &=-\int \gamma(d,x) \partial_d \{ f(d\mid x) \partial_d \gamma(d,x)\} f(x) \mathrm{d}dx \\
    &=\int \gamma(d,x) k_{\gamma}(d,x) f(d\mid x)f(x) \mathrm{d}dx \\
    &=E \{\gamma(D,X) k_{\gamma}(D,X)\} \\
    &\leq \|\gamma\|_{\text{ \normalfont pr},2} \|k_{\gamma}\|_{\text{ \normalfont pr},2},
\end{align*}
where the inequality is Cauchy Schwarz. The final results immediately follow from the definition of $k_{\gamma}$ and triangle inequality.
\end{proof}


\begin{proof}[of Lemma~\ref{lemma:cont}]
For Example~\ref{ex:CATE}, write
\begin{align*}
    E [\{\ell_h(V)\gamma(1,V,X)-\ell_h(V)\gamma(0,V,X)\}^2 ]\leq 2E\{\ell_h(V)^2\gamma(1,V,X)^2\}+2E\{\ell_h(V)^2\gamma(0,V,X)^2\}.
\end{align*}
Invoking the bounded weighting and propensity score assumptions,
\begin{align*}
   E\{\ell_h(V)^2\gamma(1,V,X)^2\}&=E \left\{\frac{D}{\pi_0(V,X)}\ell_h(V)^2\gamma(D,V,X)^2\right\} \leq C E \left\{\gamma(D,V,X)^2\right\}; \\
    E\{\ell_h(V)^2\gamma(0,V,X)^2\}&=E \left\{\frac{1-D}{1-\pi_0(V,X)}\ell_h(V)^2\gamma(D,V,X)^2\right\} \leq C E \left\{\gamma(D,V,X)^2\right\}.
\end{align*}

For Example~\ref{ex:RDD}, write
$$
E[\{\ell^{+}_{h}(D)\gamma_0(D,X)-\ell^{-}_{h}(D)\gamma_0(D,X)\}^2]\leq 2 E[\{\ell^{+}_{h}(D)\gamma_0(D,X)\}^2]+2E[\{\ell^{-}_{h}(D)\gamma_0(D,X)\}^2].
$$
Invoking the bounded weighting assumption,
\begin{align*}
    E[\{\ell^{+}_{h}(D)\gamma_0(D,X)\}^2]&\leq CE\{\gamma_0(D,X)\}^2; \\
     E[\{\ell^{-}_{h}(D)\gamma_0(D,X)\}^2]&\leq CE\{\gamma_0(D,X)\}^2. 
\end{align*}

For Examples~\ref{ex:elasticity} and~\ref{ex:deriv}, appeal to Lemma~\ref{lemma:rp}.
\end{proof}

\begin{proof}[of Lemma~\ref{lemma:bounded_RR_global}]
The result is immediate from Lemma~\ref{lemma:RR_exists}.
\end{proof}

\subsection{Local functionals}

\begin{proof}[of Lemma~\ref{lemma:local}]
We extend \cite[Lemma 3.4]{chernozhukov2018global}. We proceed in steps.
\begin{enumerate}
    \item Moment bounds. 
    
    As in the of Lemma~\ref{lemma:global},
$$
\sigma^2=E\{E(U_1^2 \mid W)\}+ E\{\alpha_0^{\min,h}(W)^2 E(U_2^2 \mid W)\}.
$$
Note that
$$
0 \leq E\{E(U_1^2 \mid W)\} \leq \bar{c}^2 \|\ell\|^2_{\text{pr},2},
$$
and
$$
\tilde{c}^2 \|\alpha_0^{\min,h}\|^2_{\text{pr,2}} \leq E\{\alpha_0^{\min,h}(W)^2 E(U_2^2 \mid W)\} \leq \bar{c}^2\|\alpha_0^{\min,h}\|^2_{\text{pr,2}}.
$$
In summary
$$
\tilde{c}^2 \|\alpha_0^{\min,h}\|^2_{\text{pr,2}}  \leq \sigma^2 \leq \bar{c}^2 (\|\ell\|^2_{\text{pr},2}+\|\alpha_0^{\min,h}\|^2_{\text{pr,2}}).
$$
As in the proof of Lemma~\ref{lemma:global},
\begin{align*}
  \|\psi_0\|_{\text{pr},q} &\leq   \|U_1\|_{\text{ \normalfont pr},q} + [E\{\alpha_0^{\min,h}(W)^q E( U_2^q \mid W)\}]^{1/q}  
  \leq  \bar{c} (\|\ell \|_{\text{pr,q}}+\|\alpha_0^{\min,h}\|_{\text{pr,q}}).
\end{align*}
Next we characterize $\|\alpha_0^{\min,h}\|_{\text{pr,q}}$ in terms of $\|\ell \|_{\text{pr,q}}$. Since $\alpha_0^{\min,h}(w)=\ell_h(w_j)\alpha_0^{\min}$,
$$
\tilde{\alpha} \|\ell\|_{\text{pr},q}\leq \|\alpha_0^{\min,h}\|_{\text{pr},q} \leq \check{\alpha} \|\ell\|_{\text{pr},q},\quad \|\alpha_0^{\min,h}\|_{\text{pr},2}=\bar{M}.
$$
In summary,
$$
\tilde{c}  \tilde{\alpha} \| \ell\|_{\text{pr},2}  \leq \sigma \leq \bar c \sqrt{1+ \check{\alpha}^2} \| \ell\|_{\text{pr},2}, \quad  \tilde{\alpha} \| \ell\|_{\text{pr},2}  \leq \bar{M} \leq \check{\alpha} \| \ell\|_{\text{pr},2},  \quad
\|\psi_0\|_{\text{pr},q}
\leq  \bar c (1 + \check{\alpha} ) \| \ell \|_{\text{pr},q}.
$$
    
    \item Taylor expansion.
    
    Consider the change of variables $u=(v'-v)/h$ so that $\mathsf{d} u  = h^{-1} \mathsf{d} v'$. Hence
    \begin{align*}
        \| \ell\|^q_{\text{pr},q} \omega^q &=
        \| \ell  \omega\|^q_{\text{pr},q} \\
        &=  \left\| h^{-1} K \left(\frac{v-v'}{h}\right) \right\|^q_{\text{pr},q} \\
        &= \int h^{-q}\left|K \left(\frac{v'-v}{h}\right)\right|^q f_V(v') \mathsf{d} v' \\
        &=  \int h^{-(q-1) }|K (u)|^q f_V(v - u h)  \mathsf{d} u .
    \end{align*}
It follows that
$$
 h^{-(q - 1)/q}  \tilde{f}^{1/q}  \left(\int |K|^q\right)^{1/q} 
 \leq \| \ell\|_{\text{pr},q} \omega 
 \leq  h^{-(q - 1)/q}  \bar f^{1/q}  \left(\tiny{\int} |K|^q\right)^{1/q}.
$$ 
Further, we have that
$$
\omega  = \int  h^{-1} K\left(\frac{v'-v}{h}\right) f_V(v') \mathsf{d} v' = \int   K(u) f_V(v- u h) \mathsf{d} u.
$$
Note that
$$
\int   K(u) f_V(v-0u) \mathsf{d} u=\int   K(u) f_V(v) \mathsf{d} u=f_V(v).
$$
Using the Taylor expansion in $h$ around $h=0$ and the Holder inequality, there exist some $\tilde{h}$ in $[0,h]$ such that
$$
|\omega - f_V(v)| =  \left|    h \int   K(u) \partial_v f_V(v- u \tilde h) u \mathsf{d} u \right |  \leq   h \bar f' \int |u|| K(u)| du.
$$
Hence there exists some $h_1$ in $(h,h_0)$ depending only on $(K, \bar f', \tilde{f}, \bar f)$ such that
$$ \tilde{f}/2  \leq \omega \leq 2 \bar f.$$
In summary,
$$
 h^{- (q - 1)/q}  \tilde{f}^{1/q}  \left(\int |K|^q\right)^{1/q} \frac{1}{2 \bar f} \leq \| \ell\|_{\text{pr},q}  \leq  h^{-(q - 1)/q}  \bar f^{1/q}  \left(\tiny{\int} |K|^q\right)^{1/q} \frac{2}{\tilde{f}}.
$$     
    
    \item Collecting results.

In summary, for all $h < h_1$
$$
\tilde{c}  \tilde{\alpha} \| \ell\|_{\text{pr},2}  \leq \sigma \leq \bar c \sqrt{1+ \check{\alpha}^2} \| \ell\|_{\text{pr},2}, \quad  \tilde{\alpha} \| \ell\|_{\text{pr},2}  \leq \bar{M} \leq \check{\alpha} \| \ell\|_{\text{pr},2},  \quad
\|\psi_0\|_{\text{pr},q}
\leq  \bar c (1 + \check{\alpha} ) \| \ell \|_{\text{pr},q},
$$
where
$$
 h^{- (q - 1)/q}  \tilde{f}^{1/q}  \left(\int |K|^q\right)^{1/q} \frac{1}{2 \bar f} \leq \| \ell\|_{\text{pr},q}  \leq  h^{-(q - 1)/q}  \bar f^{1/q}  \left(\tiny{\int} |K|^q\right)^{1/q} \frac{2}{\tilde{f}},
$$  
so
$$
\sigma \asymp \bar{M}\asymp \| \ell\|_{\text{pr},2},\quad \|\psi_0\|_{\text{pr,q}}\lesssim \|\ell\|_{\text{pr},q},\quad \|\ell\|_{\text{pr},q}\asymp h^{-(q-1)/q}.
$$
\end{enumerate}
\end{proof}

\begin{proof}[of Lemma~\ref{lemma:cont_local}]
We prove the result for Example~\ref{ex:CATE}. The result for Example~\ref{ex:RDD} is similar.

Without loss of generality, let $\bar{Q}_h$ be the smallest finite constant for which Assumption~\ref{assumption:cont} holds, i.e.
$$
\bar{Q}_h=\inf [c\geq 0: E\{m_h(W,\gamma)^2\}\leq c \|\gamma\|^2_{\text{\normalfont pr,2}} \text{ for all } \gamma \text{ in }\Gamma ].
$$
To begin, write
\begin{align*}
    E\{m_h(W,\gamma)^2\}&=E[\ell_h(V)^2\{\gamma(1,V,X)-\gamma(0,V,X)\}^2] \\
    &\leq 2 E\{\ell_h(V)^2\gamma(1,V,X)^2\}+2E\{\ell_h(V)^2\gamma(0,V,X)^2\}.
\end{align*}
Since $\pi_0(v,x)$ is bounded away from zero and one,
$$
E\{\ell_h(V)^2\gamma(1,V,X)^2\}=E\left\{\ell_h(V)^2 \frac{D}{\pi_0(V,X)}\gamma(D,V,X)^2\right\} \leq C E\left\{\ell_h(V)^2 \gamma(D,V,X)^2\right\}.
$$
Likewise for $E\{\ell_h(V)^2\gamma(0,V,X)^2\}$. In summary,
$$
E\{m_h(W,\gamma)^2\} \leq 2C E\left\{\ell_h(V)^2 \gamma(D,V,X)^2\right\}.
$$
Viewing the latter expression as an inner product in $\mathbb{L}_2$, it is maximized by alignment, i.e. taking 
$
\gamma(D,V,X)^2=\ell_h(V)^2.
$
Therefore
$$
\frac{E\{m_h(W,\gamma)^2\}}{E\{\gamma(W)^2\}} \leq \frac{2C E\left\{\ell_h(V)^4\right\}}{E\left\{\ell_h(V)^2\right\}}=2C \frac{\|\ell\|^4_{\text{pr,4}}}{\|\ell\|^2_{\text{pr,2}}}.
$$
Appealing to $\|\ell\|_{\text{pr},q}\asymp h^{-(q-1)/q}$ from the proof of Lemma~\ref{lemma:local},
$$
\frac{E\{m_h(W,\gamma)^2\}}{E\{\gamma(W)^2\}}\lesssim \frac{h^{-3}}{h^{-1}}=h^{-2}.
$$
\end{proof}

\begin{proof}[of Lemma~\ref{lemma:bounded_RR_local}]
Write
$$
\|\alpha_0^{\min,h}\|_{\infty}\leq \check{\alpha}\|\ell\|_{\infty}.
$$
By the proof of Lemma~\ref{lemma:local},
$$
\|\ell\|_{\infty}=\left\| \frac{1}{h\omega}K\left(\frac{v'-v}{h}\right) \right\|_{\infty}\leq \bar{K}\frac{1}{h\omega}\leq \bar{K}\frac{2}{h \tilde{f}}.
$$
Therefore
$$
\|\alpha_0^{\min,h}\|_{\infty}\leq \check{\alpha}\bar{K}\frac{2}{h \tilde{f}}\lesssim h^{-1}.
$$
\end{proof}

\begin{proof}[of Lemma~\ref{lemma:translate_RR}]
Write
\begin{align*}
    \mathcal{R}(\hat{\alpha}^h_{\ell})
    &=E[\{\hat{\alpha}^h_{\ell}(W)-\alpha^{\min,h}_0(W)\}^2\mid I^c_{\ell}] \\
    &=E[\{\ell_h(W_i)\hat{\alpha}_{\ell}(W)-\ell_h(W_i)\alpha^{\min}_0(W)\}^2\mid I^c_{\ell}] \\
    &\leq \|\ell_h\|^2_{\infty} E[\{\hat{\alpha}_{\ell}(W)-\alpha^{\min}_0(W)\}^2\mid I^c_{\ell}] \\
    &=\|\ell_h\|^2_{\infty} \mathcal{R}(\hat{\alpha}_{\ell}).
\end{align*}
Finally recall from the proof of Lemma~\ref{lemma:bounded_RR_local} that $\|\ell_h\|_{\infty}\lesssim h^{-1}$. An identical argument holds for $ \mathcal{P}(\hat{\alpha}^h_{\ell})$.
\end{proof}

\subsection{Approximation error}

\begin{proof}[of Lemma~\ref{lemma:approx}]
For completeness, we quote the proof of \cite[Lemma 3.6]{chernozhukov2018global}. Define the quantities
\begin{align*}
    \vartheta_1(h) &=  \int m(v') h^{-1} K\left(\frac{v-v'}{h} \right) f_V(v') \mathsf{d} v'= \int m(v - h u) K(u) f_V(v - hu)\mathsf{d} u;\\
    \vartheta_2(h) &=  \int  h^{-1} K\left(\frac{v-v'}{h} \right) f_V(v') \mathsf{d} v' = \int   K(u) f_V(v- u h) \mathsf{d} u.
\end{align*}
By $\int K  =1 $,
$$
\vartheta_1(0) = m(v) f_V(v), \quad \vartheta_2(0) = f_V(v). 
$$
Hence$$
\theta^h_0 = \frac{\vartheta_1(h) }{\vartheta_2(h)},  \quad \theta_0^{\lim} = \frac{\vartheta_1(0) }{\vartheta_2(0)} = m(v). $$
The standard argument to control the bias of the higher order kernels employs the Taylor expansion of order $\mathsf{v}$ in $h$ around $h=0$; see e.g. \cite[Lemma B2]{newey1994kernel}. Such an argument implies there exists some constant $A_{\mathsf{v}}$ that depends only on  $\mathsf{v}$ such that 
$$
| \vartheta_1(h) - \vartheta_1(0)| \leq A_{\mathsf{v}} h^{ \mathsf{v}} \bar g_\mathsf{v} \int |  u|^{\mathsf{v}} | K(u)| du,
$$$$
| \vartheta_2(h) - \vartheta_2(0)| \leq A_{\mathsf{v}} h^{\mathsf{v}} \bar f_\mathsf{v} \int |  u|^{\mathsf{v}} | K(u)| du.
$$
Then using the relation
\begin{align*}
    &\frac{\vartheta_1(h) }{\vartheta_2(h)} - \frac{\vartheta_1(0) }{\vartheta_2(0)} \\ 
    &=
 \vartheta^{-1}_2(0) \{\vartheta_1(h)  - \vartheta_1(0)\}+  \vartheta_1(0) \{\vartheta_2^{-1}(h)-\vartheta_2^{-1}(0)\}+   \{\vartheta_1(h)  - \vartheta_1(0)\}\{\vartheta_2^{-1}(h)-\vartheta_2^{-1}(0)\},
\end{align*}
we deduce that for all $h< h_1\leq h_0$,
$$
|\theta_0^h  - \theta_0^{\lim}| \leq  \left | \frac{\vartheta_1(h) }{\vartheta_2(h)} - \frac{\vartheta_1(0) }{\vartheta_2(0)}  \right | \leq C h^{\mathsf{v}}, 
$$
where $C$ and $h_1$ depend  only on $(K, \mathsf{v}, \bar g_{\mathsf{v}}$,  $\bar f_{\mathsf{v}}$, $\tilde{f})$.
\end{proof}

\begin{proof}[of Corollary~\ref{cor:CI_local}]
By Lemma~\ref{lemma:local}, write the regularity condition on moments as
$$
\left\{\left(\kappa/\sigma\right)^3+\zeta^2\right\}n^{-1/2}\lesssim \left\{\left(h^{-1/6}\right)^3+(h^{-3/4})^2\right\}n^{-1/2}\lesssim h^{-3/2} n^{-1/2}.
$$
By Lemmas~\ref{lemma:local},~\ref{lemma:cont_local}, and~\ref{lemma:bounded_RR_local}, write the first learning rate condition as
$$
\left(\bar{Q}^{1/2}+\bar{\alpha}/\sigma+\bar{\alpha}'\right)\{\mathcal{R}(\hat{\gamma}_{\ell})\}^{1/2} 
\lesssim \left(h^{-1}+h^{-1}/h^{-1/2}+\bar{\alpha}'\right)\{\mathcal{R}(\hat{\gamma}_{\ell})\}^{1/2} 
\lesssim \left(h^{-1}+\bar{\alpha}'\right)\{\mathcal{R}(\hat{\gamma}_{\ell})\}^{1/2}.
$$
By Lemma~\ref{lemma:translate_RR}, write the second learning rate condition as
$$
\bar{\sigma}\{\mathcal{R}(\hat{\alpha}^h_{\ell})\}^{1/2} \lesssim \bar{\sigma}h^{-1}\{\mathcal{R}(\hat{\alpha}_{\ell})\}^{1/2}.
$$
By Lemmas~\ref{lemma:local} and~\ref{lemma:translate_RR}, write the initial term in the third learning rate condition as
$$
\{n \mathcal{R}(\hat{\gamma}_{\ell}) \mathcal{R}(\hat{\alpha}^h_{\ell})\}^{1/2}  /\sigma 
\lesssim \{n \mathcal{R}(\hat{\gamma}_{\ell}) \mathcal{R}(\hat{\alpha}_{\ell})\}^{1/2}  h^{-1}/h^{-1/2}
=h^{-1/2}\{n \mathcal{R}(\hat{\gamma}_{\ell}) \mathcal{R}(\hat{\alpha}_{\ell})\}^{1/2}.
$$
Likewise for the other terms. The approximation error condition is immediate from Lemma~\ref{lemma:approx}.
\end{proof}