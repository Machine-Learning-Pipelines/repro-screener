\section{Experimental details}
\label{sect:exp-setting-all}

\subsection{Settings for main experiment results}
\label{sect: exp-practical}
\smallsection{Dataset}
We include experiment results on CIFAR-10, CIFAR-100, Tiny-ImageNet and SVHN.

\smallsection{Training setting}
We employ SGD as the optimizer. The batch size is fixed to 128. The momentum and weight decay are set to $0.9$ and $0.0005$ respectively. Other settings are listed as follows.
\begin{itemize}[leftmargin=*,nosep]
    \item CIFAR-10/CIFAR-100: we conduct the adversarial training for $160$ epochs, with the learning rate starting at $0.1$ and reduced by a factor of $10$ at the $80$ and $120$ epochs.
    \item Tiny-ImageNet: we conduct the adversarial training for $80$ epochs, with the learning rate starting at $0.1$ and reduced by a factor of $10$ at the $40$ and $60$ epochs.
    \item SVHN: we conduct the adversarial training for $80$ epochs, with the learning rate starting at $0.01$ (as suggested by \citep{chen2021robust}) and reduced by a factor of $10$ at the $40$ and $60$ epochs.
\end{itemize}

\smallsection{Adversary setting}
We conduct adversarial training with $\ell_\infty$ norm-bounded perturbations. We employ adversarial training methods including PGD-AT, TRADES and FGSM. We set the perturbation radius to be $8/255$. For PGD-AT and TRADES, the step size is $2/255$ and the number of attack iterations is $10$.
% We employ AutoAttack~\citep{Croce2020ReliableEO} For reliable robustness evaluation.


\smallsection{Robustness evaluation}
We consider the robustness against $\ell_\infty$ norm-bounded adversarial attack with perturbation radius $8/255$. We employ AutoAttack for reliable evaluation. We also include the evaluation results again PGD-1000, Square Attack and RayS.


\smallsection{Neural architectures}
We include experiments results on pre-activation ResNet-18, WRN-28-5, WRN-34-10 and VGG-19. 

\smallsection{Hardware}
We conduct experiments on 
% NVIDIA GeForce GTX 1080 Ti and 
NVIDIA Quadro RTX A6000.





% ----------------------------------------
\subsection{Settings for analyzing double descent in adversarial training}
\label{sect: exp-double-descent}

\smallsection{Dataset}
We conduct experiments on the CIFAR-10 dataset, without additional data.

\smallsection{Training setting}
We conduct the adversarial training for $1000$ epochs unless otherwise noted.
By default we use SGD as the optimizer with a fixed learning rate $0.1$. When we experiment on a subset (see below) we use the Adam optimizer to improve training stability, where the learning rate is fixed as $0.0001$.
% , since it requires minimal hyper-parameter tuning. 
The batch size will be fixed to $128$, and the momentum will be set as $0.9$ wherever necessary. No regularization such as weight decay is used. These settings are mostly aligned with the empirical analyse of double descent under standard training~\citep{Nakkiran2020DeepDD}.

\smallsection{Sample size}
To reduce the computation load demanded by an exponential number of training epochs, we reduce the size of the training set by randomly sampled a subset of size $5000$ from the original training set without replacement. 
% This will linearly shift the double descent curve but will not significant distort its shape as shown in Appendix~\ref{sect:double-descent-reconcile}. 
We adopt this setting for extensive experiments for analyzing the dependence of epoch-wise double descent on the perturbation radius and data quality (i.e. Figure~\ref{fig:dependence-perturbation-quality})..
% and the number of attack iterations.
% Note that in the experiment associated with data quality, we randomly sampled the training subset from those examples with quality lower than a threshold. 
% The sampled subset is restricted to class-balanced.

\smallsection{Adversary setting}
We conduct adversarial training with $\ell_\infty$ norm-bounded perturbations. We employ standard PGD training with the perturbation radius set to $8/255$ unless otherwise noted. The number of attack iterations is fixed as $10$, and the perturbation step size is fixed as $2/255$.
 
\smallsection{Robustness evaluation}
We consider the robustness against $\ell_\infty$ norm-bounded adversarial attack with perturbation radius $8/255$. We use PGD attack with $10$ attack iterations and step size set to $2/255$.



\smallsection{Neural architecture}
By default we experiment on Wide ResNet~\citep{Zagoruyko2016WideRN} with depth $28$ and widening factor $5$ (WRN-28-5) to speed up training. 

\smallsection{Hardware}
We conduct experiments on 
% NVIDIA GeForce GTX 1080 Ti and 
NVIDIA Quadro RTX A6000.

  
version https://git-lfs.github.com/spec/v1
oid sha256:122051f2431a5b436b094ebddc345f4fd484efcc0a36baa1a161ba0cc24b0baf
size 1294



% \subsection{Settings for adversarial augmentation}
% \label{sect: exp-ad-augment}




% \subsection{Settings for Mixup augmentation}
% \label{sect: exp-mixup-augment}

% For every example in the clean training set, we generate a perturbed training example by interpolating it with another example from a different class. All those perturbed examples along with their original labels are then grouped into a new training set, where we conduct standard training for $1000$ epochs. Other settings are same as those listed in Appendix~\ref{sect: exp-double-descent} except no adversary is employed.

\subsection{Settings for standard training on fixed augmented training sets}

\subsubsection{General settings for both adversarial augmentation and Gaussian augmentation}

\smallsection{Dataset}
We conduct experiments on the CIFAR-10 dataset, without additional data.

\smallsection{Training setting}
We conduct the standard training for $1000$ epochs.
We use Adam as the optimizer with a fixed learning rate $0.0001$ to improve training stability with a small training set (see below). The batch size will be fixed to $128$, and the momentum will be set as $0.9$ wherever necessary. No regularization such as weight decay is used.

\smallsection{Sample size}
To reduce the computation load demanded by an exponential number of training epochs, we reduce the size of the training set by randomly sampled a subset of size $5000$ from the original training set without replacement.

\smallsection{Neural architecture}
By default we experiment on Wide ResNet~\citep{Zagoruyko2016WideRN} with depth $28$ and widening factor $5$ (WRN-28-5).

\smallsection{Hardware}
We conduct experiments on 
% NVIDIA GeForce GTX 1080 Ti and 
NVIDIA Quadro RTX A6000.

\subsubsection{Construction of the training set}

% \subsection{Static adversarial perturbation}
% \label{sect:exp-static}

\smallsection{Adversarial augmentation}
We first obtain a robust model by conduct PGD training with pre-activation ResNet-18 on CIFAR-10. We use early stopping to obtain the most robust model on a validation set. The specific settings are aligned with Section~\ref{sect: exp-practical}.

Using this model, we then generate adversarial examples with PGD attack on the $5000$ examples randomly sampled from CIFAR-10 training set. The number of attack iterations is fixed as $10$ and the step size is fixed as $2/255$. The adversarial examples along with their original labels are then grouped into a training set for adversarial augmentation experiments.




version https://git-lfs.github.com/spec/v1
oid sha256:d7370b8d68c42cd3b7e2217d91e192b30ebf952537885002ad4bb239e60e833d
size 2901



