% \subsection{Real-world Experiments}
\subsection{Rectified model probability mitigates robust overfitting}
\label{sect:exp-practical-adversarial-training}

% \chengyu{This section validates the effectiveness of the confidence calibration. Therefore right after the analysis}







We now work on a realistic dataset (CIFAR-10) to demonstrate the rectified model probability 
% proposed in Equation~(\ref{eq:approximate-label-distribution})
can effectively mitigate the robust overfitting, or equivalently the epoch-wise double descent in adversarial training. The outer minimization of adversarial training (Equation~(\ref{eq:outer-minimization})) now becomes 
    % \begin{equation}
    %     \theta^* = \argmin_\theta \mathbbm{E}_\mathcal{D_\delta}~ \ell\left(f_\theta(x_\delta), P_{\theta^{\text{Trad}}}^{T, \lambda}(Y_\delta | x_\delta)\right),
    % \end{equation}
    \begin{equation}
        % \theta^* = \argmin_\theta \mathbbm{E}_\mathcal{D_\delta}~ \ell\left(f_\theta(x_\delta), P_{\theta^{\text{Trad}}}^{T, \lambda}(Y_\delta | x_\delta)\right),
        \theta^* = \argmin_\theta \mathbbm{E}_\mathcal{D'}~ \ell\left(f_\theta(x'), f_{\hat{\theta}}(x'; T, \lambda)_{y'}\right),
    \end{equation}
    where $\hat{\theta}$ denotes the parameters of a classifier adversarially trained beforehand.
    The details of the experimental setting are available in the Appendix. 
    % ~\ref{sect: exp-practical}.

% \todo{Rephrase our training framework: 1) get model probability by one training 2) find optimal $T$ and $\lambda$ 3) Adversarial training on the rectified model probability.}

% \jingbo{move Figure~\ref{fig:method-grid-search} to this page? Simplify the caption a bit by moving some of the sentences to the paragraph here.} 
As shown in Figure~\ref{fig:method-grid-search}, adversarial training on rectified model probability can mitigate the robust overfitting when the temperature $T$ and interpolation ratio $\lambda$ are optimal. Such optimal hyperparameters perfectly aligns with the ones automatically determined by Equation~(\ref{eq:calibration}).






