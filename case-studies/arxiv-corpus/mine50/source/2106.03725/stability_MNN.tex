%!TEX root = stability_manifold_TSP.tex

Manifold neural networks (MNNs) are deep convolutional architectures comprised of $L$ layers, where each layer consists of two components: a convolutional filterbank and a pointwise nonlinearity. 
At each layer $l=1,2,\hdots, L$, the convolutional filters map the incoming $F_{l-1}$ features from layer $l-1$ into $F_l$ intermediate linear features given by
\begin{equation}\label{eqn:mnn}
y_l^p(x) =  \sum_{q=1}^{F_{l-1}} \bbh_l^{pq}(\ccalL) f_{l-1}^q(x),
\end{equation}
where $\bbh_l^{pq}(\ccalL)$ is the filter mapping the $q$-th feature from layer $l-1$ to the $p$-th feature of layer $l$ as in \eqref{eqn:convolution-general}, for $1\leq q\leq F_{l-1}$ and $1\leq p\leq F_{l}$. The intermediate features are then processed by a pointwise nonlinearity $\sigma: \reals\rightarrow \reals$ as
\begin{equation}\label{eqn:mnn}
f_l^p(x) = \sigma\left(y_l^p(x) \right).
\end{equation}
The nonlinearity $\sigma$ processes each feature individually and we further make an assumption on its continuity as follows. 

%%%%%%%%%%%%%%%%%%%%%%%%%%%%%%%%%%%%%%%%%%%%%%%%
%%%%%%%%%%%%%%%%%% ASSUMPTION %%%%%%%%%%%%%%%%%% 
%%%%%%%%%%%%%%%%%%%%%%%%%%%%%%%%%%%%%%%%%%%%%%%%
\begin{assumption}[Normalized Lipschitz activation functions]\label{ass:activation}
 The activation function $\sigma$ is normalized Lipschitz continous, i.e., $|\sigma(a)-\sigma(b)|\leq |a-b|$, with $\sigma(0)=0$.
\end{assumption}
 
Note that this assumption is rather reasonable, since most common activation functions (e.g., the ReLU, the modulus and the sigmoid) are normalized Lipschitz by design.

At the first layer of the MNN, the input features are the input data $f^q$ for $1\leq q\leq F_0$. At the output of the MNN, the output features are given by the outputs of the $L$-th layer, i.e., $f_L^p$ for $1 \leq p \leq F_L$. To represent the MNN more succinctly, we may gather the impulse responses of the manifold convolutional filters $\bbh_l^{pq}$ across all layers in a function set $\bbH$, and define the MNN map $\bbPhi(\bbH,\ccalL, f)$. This map emphasizes that the MNN is parameterized by both the filter functions and the LB operator $\ccalL$. We next will analyze the stability of $\bbPhi(\bbH,\ccalL,f)$ with respect to perturbations on the underlying manifold.

%\red{(L: Recall that at this stage our filters are not yet parametrized by coefficients $h_k$. So you have to change the definition of this map a bit.)}

%%%%%%%%%%%%%%%%%%%%%%%%%%%%%%%%%%%%%%%%%%%%%%%%
%%%%%%%%%%%%%%%%%% REMARK %%%%%%%%%%%%%%%%%%%%%% 
%%%%%%%%%%%%%%%%%%%%%%%%%%%%%%%%%%%%%%%%%%%%%%%%
% \red{\begin{remark}
% \label{rem:parametrization}
% \normalfont
% The mapping function $\bbPhi(\bbH,\ccalL, f)$ elucidates the parametrization of the MNN depending on both the filter coefficients $\bbH$ and the LB operator $\ccalL$. The number of filter coefficients is independent of the dimensionality of the manifold due to the convolutional parametrization as shown in \eqref{eqn:convolution-general}. The fact that LB operator $\ccalL$ decides the MNN mapping when the filter coefficients are fixed is especially important. What we analyze in this paper is precisely how the performance of the MNN is affected by perturbations of $\ccalL$ caused by perturbations of the manifold.
% \end{remark}
% }
% \red{L: Now that we have the section on the implementation of MNNs, I am not sure we need this remark. And same comment as above: at this point we haven't parametrized the filter with coefficients $h_k$ yet.}


%%%%%%%%%%%%%%%%%%%%%%%%%%%%%%%%%%%%%%%%%%%%%%%%
%%%%%%%%%%%%%%%%%% SUBSECTION %%%%%%%%%%%%%%%%%% 
%%%%%%%%%%%%%%%%%%%%%%%%%%%%%%%%%%%%%%%%%%%%%%%%

\subsection{Stability of MNNs to LB Operator Perturbations}
\label{subsec:stability_mnn_abs}

MNNs inherit stability to perturbations of the LB operator from the manifold filters that compose the filterbanks in each one of their layers. This result is stated in general form---encompassing both absolute and relative perturbations---in the following theorem.

%%%%%%%%%%%%%%%%%%%%%%%%%%%%%%%%%%%%%%%%%%%%%%%%
%%%%%%%%%%%%%%%%%% THEOREM %%%%%%%%%%%%%%%%%%%%% 
%%%%%%%%%%%%%%%%%%%%%%%%%%%%%%%%%%%%%%%%%%%%%%%%
\begin{theorem}[MNN stability]\label{thm:stability_nn}
Consider a compact embedded manifold $\ccalM$ with LB operator $\ccalL$. Let $\bm\Phi(\bbH,\ccalL,f)$ be an $L$-layer MNN on $\ccalM$ \eqref{eqn:mnn} with $F_0=F_L=1$ input and output features and $F_l=F,l=1,2,\hdots,L-1$ features per layer. 
The filters $\bbh(\ccalL)$ and nonlinearity functions satisfy  Assumptions \ref{ass:filter_function} and \ref{ass:activation} respectively. Let $\ccalL'$ be the perturbed LB operator [cf. Definition \ref{def:abso-perturb} or Definition \ref{def:rela-perturb}] with $\max\{\alpha, 2, |\gamma/1-\gamma|\}\gg \epsilon$. If the manifold filters satisfy $\| \bbh(\ccalL)f -\bbh(\ccalL')f \|_{L^2(\ccalM)}\leq C_{per} \epsilon \|f\|_{L^2(\ccalM)}$, it holds that
\begin{align}\label{eqn:stability_nn}
   \nonumber \|\bm\Phi(\bbH,\ccalL,f)-\bm\Phi(\bbH,\ccalL',f)\|_{L^2(\ccalM)} \leq LF^{L-1}C_{per}\epsilon \|f\|_{L^2(\ccalM)}.
 \end{align}
\end{theorem}
\begin{proof}
See Section \ref{app:stability_nn} in supplementary material.
\end{proof}

Theorem \ref{thm:stability_nn} reflects that the stability of the MNN is affected by the hyperparameters of the MNN architecture and the stability constant of the manifold filters $C_{per}$. More explicitly, the stability bound grows linearly with the number of layers $L$ and exponentially with the number of features $F$ where the rate is determined by $L$. This stability result also shows that there is a linear dependence on the stability constant $C_{per}$ of manifold filters $\bbh(\ccalL)$ and the perturbation size $\epsilon$. As we have shown in Section \ref{subsec:filter-absolute} and \ref{subsec:filter-relative}, the stability constant is determined by the form of the perturbations (Definition \ref{def:abso-perturb} or Definition \ref{def:rela-perturb}) as well as the spectrum separation achieved by the specific manifold filters (Definition \ref{def:alpha-filter} or Definition \ref{def:frt-filter}) with corresponding Lipschitz conditions (Definition \ref{def:lipschitz} or Definition \ref{def:int-lipschitz}). We address the specific cases as follows.

\begin{proposition}
\label{prop:stability-nn}
With the same conditions as Theorem \ref{thm:stability_nn}, consider the following perturbation models.
\begin{enumerate}
    \item  If the perturbed LB operator $\ccalL'$ is an absolute perturbation, i.e., $\ccalL'=\ccalL+\bbA$ [cf. Definition \ref{def:abso-perturb}] with $\|\bbA\|=\epsilon$ and the manifold filters $\bbh(\ccalL)$ are $\alpha$-FDT [cf. Definition \ref{def:alpha-filter}] with $\alpha\gg\epsilon$ and $A_h$-Lipschitz continuous [Definition \ref{def:lipschitz}] with $\delta = \pi\epsilon/(2\alpha)$, we have
    \begin{equation}
        C_{per} = \frac{\pi N}{\alpha} + A_h,
    \end{equation}
    where $N$ is the size of the $\alpha$-separated spectrum partition [cf. Definition \ref{def:alpha-spectrum}].
    \item If the perturbed LB operator $\ccalL'$ is a relative perturbation, i.e. $\ccalL'=\ccalL+\bbE\ccalL$ [cf. Definition \ref{def:rela-perturb}] with $\|\bbE\|=\epsilon$, and the manifold filters $\bbh(\ccalL)$ are $\gamma$-FRT [cf. Definition \ref{def:frt-filter}] with $\gamma/(1-\gamma)\gg\epsilon$ and $B_h$-integral Lipschitz continuous [Definition \ref{def:int-lipschitz}] with $\delta = \pi\epsilon/(2\gamma)$, we have
    \begin{equation}
        C_{per} = \frac{\pi M}{\gamma} + B_h,
    \end{equation}
    where $M$ is the size of the $\gamma$-separated spectrum partition [cf. Definition \ref{def:frt-spectrum}].
\end{enumerate}

\end{proposition}
\begin{proof}
The conclusions follow directly from Theorem \ref{thm:stability_nn} combined with Theorem \ref{thm:stability_abs_filter} or Theorem \ref{thm:stability_rela_filter} under the corresponding assumptions.
\end{proof}

Combining Theorem \ref{thm:stability_nn} with Proposition \ref{prop:stability-nn}, we observe that $\alpha$-FDT manifold filters with Lipschitz continuity can be composed to construct MNNs which are stable to absolute perturbations; while $\gamma$-FRT manifold filters with integral Lipschitz continuity can be composed to construct MNNs which are stable relative perturbations of the LB operator.
Explicitly, by inserting the stability constant $C_{per}$ in \eqref{eqn:stability_nn}, we see that other than the perturbation size $\epsilon$, there are three terms that determine the stability of MNNs. The first term is $LF^{L-1}$, which, as we have already discussed, is decided by the number of layers and filters in the MNN architecture. This term arises due to the propagation of the underlying operator perturbations across all the manifold filters in all layers of the MNN. The second term is $\pi N/\alpha$ or $\pi M/\gamma$, which results from the deviations of the eigenfunctions as well as from the frequency response variations within the same eigenvalue partition. Finally, the third term, $A_h$ or $B_h$, is given by the Lipschitz or integral Lipschitz constants which are decided during the filter design or the training process. It is important to note that the stability constant $C_{per}$ brings along the trade-off between stability and discriminability. However, unlike manifold filters, MNNs can be both stable and discriminative. This arises from the effects of nonlinear activation functions, as we discuss in further detail in Section \ref{subsec:discussion}.

% MNN stability bound is equal to the $\alpha$-FDT filter stability bound scaled by the depth and width of the MNN. More specifically, the bound scales linearly with the number of layers $L$ and exponentially with the number of features per layer $F$. While this can be expected to mean that wider and deeper architectures are less stable, in practice the filters of the MNN are learned, which means that $\alpha$ and $A_h$ vary significantly across architectures. Moreover, the nonlinear activation function plays an important role in the stability of MNNs that is not captured by Theorem \ref{thm:stability_nn}, which we will discuss in more detail in Section \ref{sec:discussion}.

% If the convolutional filters in the layers of the MNN are $\alpha$-FDT and Lipschitz [cf. Definition \red{X} and \red{Y}], the MNN inherits their stability to absolute perturbations of the LB operator. %The architecture of MNNs also has a substantial influence on the stability bound.

%%%%%%%%%%%%%%%%%%%%%%%%%%%%%%%%%%%%%%%%%%%%%%%%
%%%%%%%%%%%%%%%%%% THEOREM %%%%%%%%%%%%%%%%%%%%% 
%%%%%%%%%%%%%%%%%%%%%%%%%%%%%%%%%%%%%%%%%%%%%%%%

% \begin{theorem}[MNN stability to absolute perturbations]\label{thm:stability_nn}
% Consider a manifold $\ccalM$ with LB operator $\ccalL$. Let $\bm\Phi(\bbH,\ccalL,f)$ be an $L$-layer MNN on $\ccalM$ \eqref{eqn:mnn} with $F_0=F_L=1$ input and output features and $F_l=F,l=1,2,\hdots,L-1$ features per layer. The filters $\bbh(\ccalL)$ are $\alpha$-FDT filters [cf. Definition \ref{def:alpha-filter}] and $A_h$-Lipschitz [cf. Definition \ref{def:lipschitz}] with $\delta=\pi\epsilon/(2\alpha-2\epsilon)$. 
% Consider an absolute perturbation $\ccalL'=\ccalL + \bbA$ of the LB operator $\ccalL$ [cf. Definition \ref{def:abso-perturb}] where $\|\bbA\| = \epsilon < \alpha$. Then, under Assumptions \ref{ass:filter_function} and \ref{ass:activation} it holds that
% \begin{align}\label{eqn:stability_nn}
%   \nonumber \|\bm\Phi(\bbH,\ccalL,f)-\bm\Phi&(\bbH,\ccalL',f)\|_{L^2(\ccalM)} \leq \\&LF^{L-1}\left(\frac{\pi N\epsilon}{\alpha-\epsilon}+A_h\epsilon\right) \|f\|_{L^2(\ccalM)}.
%  \end{align}
% {where $N$ is the size of the $\alpha$-separated spectrum partition [cf. Definition \ref{def:alpha-spectrum}]. }
% \end{theorem}
% \begin{proof}
% See Appendix \ref{app:stability_nn}.
% \end{proof}

% From \eqref{eqn:stability_nn}, we observe that the MNN stability bound is equal to the $\alpha$-FDT filter stability bound scaled by the depth and width of the MNN. More specifically, the bound scales linearly with the number of layers $L$ and exponentially with the number of features per layer $F$. While this can be expected to mean that wider and deeper architectures are less stable, in practice the filters of the MNN are learned, which means that $\alpha$ and $A_h$ vary significantly across architectures. Moreover, the nonlinear activation function plays an important role in the stability of MNNs that is not captured by Theorem \ref{thm:stability_nn}, which we will discuss in more detail in Section \ref{sec:discussion}.
%However, the stability bound of $\alpha$-FDT filter is also inherited which indicates the perturbation also results from the differences of eigenfunctions and the disturbance of eigenvalues. Furthermore, the stability bound still scales linear with the size of the absolute perturbation.

 
%%%%%%%%%%%%%%%%%%%%%%%%%%%%%%%%%%%%%%%%%%%%%%%%
%%%%%%%%%%%%%%%%%% SUBSECTION %%%%%%%%%%%%%%%%%% 
%%%%%%%%%%%%%%%%%%%%%%%%%%%%%%%%%%%%%%%%%%%%%%%%
% \subsection{Stability of MNNs to relative perturbations}
% \label{subsec:stability_mnn_rel}
% \red{L: Your subsection titles are not consistent. Some have capital letters for every word, some only for the first word.}

% \red{L: This subsection is short and a little repetitive. So I think it might be a good idea to merge at least subsections A and B, state one theorem after the other, and then have a single paragraph to discuss both of them. Then we can either keep or merge subsection C. What do you think?}

% Under the same assumptions on the filter amplitude and on the activation function (Assumptions \ref{ass:filter_function} and \ref{ass:activation}), MNNs with integral Lipschitz $\gamma$-FRT filters are stable to relative perturbations as stated in Theorem \ref{thm:stability_nn_rela}. 

%%%%%%%%%%%%%%%%%%%%%%%%%%%%%%%%%%%%%%%%%%%%%%%%
%%%%%%%%%%%%%%%%%% THEOREM %%%%%%%%%%%%%%%%%%%%% 
%%%%%%%%%%%%%%%%%%%%%%%%%%%%%%%%%%%%%%%%%%%%%%%%
% \begin{theorem}[MNN stability to relative perturbations]\label{thm:stability_nn_rela}
% Consider a manifold $\ccalM$ with LB operator $\ccalL$. Let $\bm\Phi(\bbH,\ccalL,f)$ be an $L$-layer MNN on $\ccalM$ \eqref{eqn:mnn} with $F_0=F_L=1$ input and output features and $F_l=F,i=1,2,\hdots,L-1$ features per layer. The filters $\bbh(\ccalL)$ are $\gamma$-FRT filters [cf. Definition \ref{def:frt-filter}] and $B_h$-integral Lipschitz [cf. Definition \ref{def:int-lipschitz}] with $\delta=\pi\epsilon/(2\gamma-2\epsilon+2\gamma\epsilon)$. 
% Consider a relative perturbation $\ccalL'=\ccalL + \bbE\ccalL$ of the LB operator $\ccalL$ [cf. Definition \ref{def:rela-perturb}] where $\|\bbE\| = \epsilon < \gamma$. 
% Then, under Assumptions \ref{ass:filter_function} and \ref{ass:activation} it holds that
%  \begin{align}\label{eqn:stability_nn_rela}
%  \nonumber\|\bm\Phi(\bbH,\ccalL,f)-&\bm\Phi(\bbH,{\ccalL'},f)\|_{L^2(\ccalM)}  \\
%  \leq & LF^{L-1}\left(\frac{\pi M\epsilon}{\gamma-\epsilon+\gamma\epsilon}+ \frac{2{B_h}\epsilon}{2-\epsilon} \right) \|f\|_{L^2(\ccalM)} ,
%  \end{align}
% where $M$ is the size of the $\gamma$-separated spectrum partition [cf. Definition \ref{def:frt-spectrum}].
% \end{theorem}
% \begin{proof}
% See Appendix \ref{app:stability_nn_rela}.
% \end{proof}

% Similar to Theorem \ref{thm:stability_nn}, the stability bound of MNNs to relative perturbations of the LB operator also scales with $L$ and $F$. Therefore, the same conclusions hold. 

%%%%%%%%%%%%%%%%%%%%%%%%%%%%%%%%%%%%%%%%%%%%%%%%
%%%%%%%%%%%%%%%%%% SUBSECTION %%%%%%%%%%%%%%%%%% 
%%%%%%%%%%%%%%%%%%%%%%%%%%%%%%%%%%%%%%%%%%%%%%%%
\subsection{Stability of MNNs to Manifold Deformations} \label{subsec:stability_mnn_def}

In Theorem \ref{thm:stability_nn} and Proposition \ref{prop:stability-nn}, we established the conditions under which MNNs are stable to either absolute or relative perturbations of the LB operator as defined in Definitions \ref{def:abso-perturb} and \ref{def:rela-perturb}. 
Since a manifold deformation $\tau(x):\ccalM\rightarrow \ccalM$, with $\text{dist}(x,\tau(x))=\epsilon$ and $J(\tau_*)=I+\Delta$, translates into both an absolute and a relative perturbation of the Laplace-Beltrami operator, in order for the MNN to be stable to this deformation we need to meet all of these conditions in items 1 and 2 of Proposition \ref{prop:stability-nn}.
I.e., the manifold filters need to be both $\alpha$-FDT and $\gamma$-FRT, and both Lipschitz continuous and integral Lipschitz continuous. The spectrum can be made to be both $\alpha$-separated and $\gamma$-separated by making sure the eigenvalues in different partitions satisfy both \eqref{eqn:alpha-spectrum} and \eqref{eqn:frt-spectrum}. Assuming that all of these conditions are met, we can state our main result---that MNNs are stable to deformations of the manifold---as follows.
% we can analyze the stability of MNNs to general deformations of the underlying manifold.
% Theorem \ref{thm:perturb} , i.e., $\ccalL-\ccalL'=\bbE\ccalL+\bbA$ with $\|\bbE\|=\epsilon$ and $\|\bbA\|=\epsilon$. This is a combination of the perturbation terms defined in Definition \ref{def:abso-perturb} and \ref{def:rela-perturb}.

% \red{(L: Expand the introduction of this theorem. This has to be done well because the theorem below is your most important result. First, you have to explain that since a deformation has the effect of an absolute and a relative perturbation of the LB operator, we need our filters to be both FDT and FRT. Explain this first, and then explain how we can set $\gamma$ such that the filter is both FDT and FRT.)}
% B

%%%%%%%%%%%%%%%%%%%%%%%%%%%%%%%%%%%%%%%%%%%%%%%%
%%%%%%%%%%%%%%%%%% THEOREM %%%%%%%%%%%%%%%%%%%%% 
%%%%%%%%%%%%%%%%%%%%%%%%%%%%%%%%%%%%%%%%%%%%%%%%
\begin{theorem}
Let $\ccalM$ be a compact embedded manifold with LB operator $\ccalL$ and $f$ be manifold signal. We construct $\bm\Phi(\bbH,\ccalL,f)$ as a MNN on $\ccalM$ \eqref{eqn:mnn} where the filters $\bbh(\ccalL)$ are $\alpha$-FDT [cf. Definition \ref{def:alpha-filter}], $\alpha/\lambda_1$-FRT [cf. Definition \ref{def:frt-filter}], $A_h$-Lipschitz [cf. Definition \ref{def:lipschitz}] and $B_h$-integral Lipschitz [cf. Definition \ref{def:int-lipschitz}]. Consider a deformation on $\ccalM$ as $\tau(x):\ccalM\rightarrow \ccalM$ where $\text{dist}(x,\tau(x))=\epsilon$ and $J(\tau_*)=I+\Delta$ with $\|\Delta\|_F=\epsilon$ and $\epsilon\ll \min(\alpha/\lambda_1,\alpha,2)$. Then under Assumptions \ref{ass:filter_function} and \ref{ass:activation} it holds that
\begin{align}
    \|\bm\Phi(\bbH,\ccalL,f)-\bm\Phi(\bbH,\ccalL',f)\|_{L^2(\ccalM)}=O(\epsilon)\|f\|_{L^2(\ccalM)}.
\end{align}
\end{theorem}

Together, Theorem \ref{thm:perturb} and Theorem \ref{thm:stability_nn} imply that MNNs are stable to the manifold deformations $\upsilon$ introduced in the beginning of this section. This is because these deformations spawn a perturbation of the LB operator that consists of both an absolute and a relative perturbation. For stability to hold, the filters that make up the layers of the MNN need to be $\alpha$-FDT [cf. Definition \ref{def:alpha-filter}], $\gamma$-FRT [cf. Definition \ref{def:frt-filter}], Lipschitz [cf. Definition \ref{def:lipschitz}] and integral Lipschitz [cf. Definition \ref{def:int-lipschitz}]. We can propose an easier special case to relate $\alpha$ and $\gamma$ by utilizing the spectrum property of LB operator. By setting the $\alpha$-FDT filter with $\alpha = \gamma\lambda_1$, eigenvalues $\lambda_i,\lambda_{i+1}\in\Lambda_k(\alpha)$ would lead to $\lambda_i,\lambda_{i+1}\in\Lambda_l(\gamma)$ due to the fact that
\begin{align}
    \lambda_{i+1}-\lambda_i \leq \alpha=\gamma\lambda_1\leq \gamma \lambda_i,
\end{align}
with $\lambda_1$ indexed as the smallest eigenvalue in the spectrum. The requirement that the filter be $\alpha$-FDT can be removed as long as $\lambda_1>0$ and $\alpha=\gamma \lambda_1$, since a $\gamma$-FRT filter is always $\gamma\lambda_1$-FDT, i.e. $\alpha$-FDT.  

