%!TEX root = stability_manifold_TSP.tex

On the manifold $\ccalM$, we define a deformation as function $\tau(x): \ccalM \to \ccalM$, where $x \in \ccalM$ is a point on the manifold and $\text{dist}(x,\tau(x))$ is upper bounded, i.e., $\tau(x)$ is a displaced point in the neighborhood of $x$. The deformation $\tau$ has a corresponding tangent map $\tau_*: T_x\ccalM\rightarrow T_{\tau(x)}\ccalM$ and a Jacobian matrix $J(\tau_*)$. When $\text{dist}(x,\tau(x))$ is bounded, the Frobenius norm of $J(\tau_*)-I$ can also be upper bounded, and these bounds are used to measure the size of the deformation $\tau(x)$. %In this paper we are interested in how diffeomorphisms affect the performance of MNNs. Thus, we need to understand the effect of diffeomorphisms on data $f: \ccalM \to \reals$, as well as on the Laplace-Beltrami operator $\ccalL$ which is the building block of the manifold convolution \eqref{eqn:convolution-general}.

Let $f: \ccalM \to \reals$ be a manifold signal. Because $\ccalM$ is the codomain of $\tau(x)$, $g = f \circ \tau$ maps points $\tau(x) \in \ccalM$ to $f(\tau(x)) \in \reals$, so that
%$g$ is still a scalar function on $\ccalM$. In other words, we can think of 
the effect of a manifold deformation on the signal $f$ is a signal perturbation leading to a new signal $g$ supported on the same manifold. To understand the effect of this deformation on the LB operator, let $p = \ccalL g$. Since $p$ is also a signal on $\ccalM$, we may define an operator $\ccalL'$ mapping $f$ directly into $p$,
\begin{align}
\label{eqn:deform}
   p(x) = \ccalL' f(x) = \ccalL g(x) = \ccalL f(\tau(x)).%= -  \nabla \cdot  \nabla  f(\tau(x)).
\end{align}
The operator $\ccalL'$ is the perturbed LB operator, which is effectively the new LB operator resulting from the deformation $\tau$. Assuming that the gradient field is smooth, the difference between $\ccalL'$ and $\ccalL$ is given by the following theorem. The proof is deferred to Appendix \ref{app:perturb}.
%\begin{assumption}[Smoothness of the manifold]
% \label{ass:smooth}
% The gradient operator $\nabla$ of manifold $\ccalM$ satisfies $\|\nabla_y-\nabla_x\|\leq \|y-x\|$ and $\|\nabla\|\leq 1$.
% \end{assumption}
%for $x \in \ccalM$ $\upsilon(x)$ is always a point of the original manifold\footnote{In particular, if $\upsilon$ is surjective the diffeomorphism does not change the manifold (but it may still changes any functions supported on it).}, which means that the data $f$ can still be defined. However, the diffeomorphism changes how this function is evaluated as on points $x \in \ccalM$ it will now be given by $f(\upsilon(x))$. 

%As the Laplace-Beltrami operator $\ccalL$ is established based on local coordinates, the unobserved deformed points are still evaluated on the original local operator $\ccalL$. In this way, the deformation of the underlying manifold leads to the perturbation of the Laplace-Beltrami operator. As we have described, the operation carried out to the deformed manifold data $f$ can be written as:

%%%%%%%%%%%%%%%%%%%%%%%%%%%%%%%%%%%%%%%%%%%%%%%%
%%%%%%%%%%%%%%%%%% THEOREM %%%%%%%%%%%%%%%%%%%%% 
%%%%%%%%%%%%%%%%%%%%%%%%%%%%%%%%%%%%%%%%%%%%%%%%
\begin{theorem} \label{thm:perturb}
{Let $\ccalL$ be the LB operator of the manifold $\ccalM$.
Let $\tau(x):\ccalM\rightarrow \ccalM$ be a manifold perturbation such that $\text{dist}(x,\tau(x))= \epsilon$ and $J(\tau_*)= I+\Delta$ with $\|\Delta\|_F=\epsilon$. If the gradient field is smooth, it holds that}
\begin{equation}
    \label{eqn:perturb-operator}
    \ccalL-\ccalL' = \ccalE \ccalL + \bbA ,
\end{equation}
where $\ccalE$ and $\bbA$ satisfy $\|\ccalE\|=O(\epsilon)$ and $\|\bbA\|_{op}= O(\epsilon)$. 
\end{theorem}

Therefore, the perturbation of the LB operator incurred by a manifold deformation $\tau$ is a combination of an absolute perturbation $\bbA$ [cf. Definition \ref{def:abso-perturb}] and a relative perturbation $\ccalE\ccalL$ [cf. Definition \ref{def:rela-perturb}]. This largely simplifies our analysis of stability. Since manifold filters are parametric on $\ccalL$ [cf. Proposition \ref{prop:filter-spectral}], it is sufficient to characterize their stability to deformations of the manifold by analyzing their behavior in the presence of absolute and relative LB perturbations. This is what we do in Sections \ref{subsec:filter-absolute} and \ref{subsec:filter-relative}.