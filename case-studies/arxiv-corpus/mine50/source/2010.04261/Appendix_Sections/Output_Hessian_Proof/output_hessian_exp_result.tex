\subsection{Heuristic Approximating of the Top Eigenspace of Output Hessians}
\label{sec:app_outhessian_exp}

As briefly mentioned in \cref{sec:theoretical}, the closed form approximating of $S_1$ in \cref{thm:main-out} can be heuristically extended to the case with multiple layers, that the top eigenspace of the output Hessian of the $k$-layer would be approximately $\gR(\mS^{(k)})\setminus\{\textbf{1}^\T\mS^{(k)}\}$
where $\mS^{(k)} = \mW^{(n)}\mW^{(n-1)}\cdots\mW^{(k+1)}$ and $\gR(\mS^{(k)})$ is the row space of $\mS^{(k)}$.

Though our result was only proven for random initialization and random data, we observe that this subspace also has high overlap with the top eigenspace of output Hessian at the minima of models trained with real datasets. In \cref{tab:approx-m}  we show the overlap of $ \gR(\mS^{(k)})\setminus\{\textbf{1}^\T \mS^{(k)}\}$ and the top $c-1$ dimension eigenspace of $\E[\mM^{(k)}]$ of different layers at minima.


\begin{table}[H]
\vskip -0.15in
\caption{Overlap of $ \gR(\mS^{(k)})\setminus\{\textbf{1}^\T \mS^{(k)}\}$ and the top $c-1$ dimension eigenspace of $\E[\mM^{(k)}]$ of different layers.}
\vskip 0.1in
\begin{center}
\begin{small}
% \begin{sc}
\begin{tabular}{ccccccccc}
\toprule
Dataset & \multicolumn{2}{c}{MNIST} & \multicolumn{2}{c}{MNIST-R} & \multicolumn{2}{c}{CIFAR10} & \multicolumn{2}{c}{CIFAR10-R} \\
Network & F-$1500^3$    & LeNet5    & F-$1500^3$     & LeNet5     & F-$1500^3$     & LeNet5     & F-$1500^3$      & LeNet5     \\ \midrule
fc1     & 0.602         & 0.890     & 0.235          & 0.518      & 0.880          & 0.951      & 0.903           & 0.213       \\
fc2     & 0.967         & 0.931     & 0.801          & 0.912      & 0.943          & 0.972      & 0.931           & 0.701       \\
fc3     & 0.982         & 0.999     & 0.998          & 0.999      & 0.993          & 0.999      & 0.996           & 0.999     \\ \bottomrule
\end{tabular}
% \end{sc}
\end{small}
\end{center}
\label{tab:approx-m}
\vskip -0.15in
\end{table}
% \znote{This table can be compressed}
Note that the overlap can be low for random-label datasets which do not have a clear eigengap (as in \cref{fig:UTAU_H_spec}). Understanding how the data could change the behavior of the Hessian is an interesting open problem. Other papers including \citet{papyan2019measurements} have given alternative explanations which are not directly comparable to ours, however ours is the only one that gives a closed-form formula for top eigenspace. In \cref{sec:appendix_M_struct} we will discuss the other explanations in more details.