\sectiontinyvert{Camera Parameters} \label{sec:cam_param_technical}
We next formulate the notion of camera parameters.
Consider a pinhole camera model. Such a model possesses two types of parameters, extrinsic and intrinsic. \emph{Extrinsic} parameters correspond to 
\begin{itemize}
    \item A rotation matrix $R$: a matrix of size $3\times3$ characterizing the rotation from 3D real world axes into 3D camera axes.
    \item A translation vector $T$: a vector of size  $3$ representing the translational offset of the camera in the 3D scene.
\end{itemize} 
\emph{Intrinsic} parameters, stored in a $3\times3$ matrix $K$, are specific to a camera. $K$ consists of the focal length $f_x , f_y$, the camera optical center $c_x ,c_y$ and a skew coefficient $s_k$:
\begin{equation} \label{eq:intrinsics}
    K = \begin{bmatrix}
f_x & s_k & c_x\\
0 & f_y & c_y \\
0 & 0 & 1
\end{bmatrix}.
\end{equation}
We denote the mapping from 3D world coordinates into a 2D image plane by a $3\times4$ matrix $P$. $P$ is sometimes called  \emph{camera matrix} or\emph{ projection matrix}.
To calculate $P$, both camera extrinsic and intrinsic parameters are used:
\begin{equation}
\label{P matrix}
    P = K \times \begin{bmatrix}
R \kern2pt | \kern2pt T
\end{bmatrix} .
\end{equation}
