\sectiontinyvert{Conclusions and limitations}

We have presented FLEX, a multi-view method for motion reconstruction. 
%Our model is the only one that is able to reconstruct motion and pose in a multi-view setting when camera parameters are unknown.
It relies on a key understanding that 3D rotation angles and bone lengths are invariant to camera view, and their direct reconstruction spares the need for camera parameters. 
%
On a technical viewpoint, we presented a novel fusion layer with a multi-view convolutional layer and a multi-head attention mechanism that attends views.

%\paragraphtinyvert{Limitations.}
One limitation of our approach is the dependency on 3D joint location ground-truth, and in particular, the requirement that it is given at the axis system of the train cameras. 
%This limitation leads us to pursuing self-supervised ways to achieve our goal. 
Another limitation is the dependency on the 2D backbone quality, and on the accuracy value associated with each joint.
% 
Lastly, being ep-free, the output 3D joint positions are only relative to the camera, lacking the transformation with respect to a global axis system.  
%A straight forward mitigation would be to define an artificial global axis system (e.g., the axes of one of the cameras), and present all results relative to these axes. 


In summary, FLEX is unique in fusing multi-view information to reconstruct motion and pose in dynamic photography environments. %
%In settings where the relative rotations among the cameras are unknown, 
%our system \sr{is unaffected and can still }maintain a high level of accuracy.
It is unaffected by settings in which the relative rotations between the cameras are unknown, 
and can maintain a high level of accuracy regardless.
%
FLEX offers a simpler setting, where the correspondence and compatibility among the different views are rather lean, and thus more resilient to input errors and innate inaccuracies.
%Beyond offering a simpler setting, FLEX is an end-to-end network, and the correspondence and compatibility among the different views are rather leaned, and thus more resilient to input errors and innate inaccuracies. 

