\section{Probability Lemma and Concentration Bounds} \label{apx:prob}
\begin{lemma}\label{lemma:chernoff}
    (Chernoff bounds \cite{mitzenmacher2017probability}). 
    Suppose $X_1$, ... , $X_n$ are independent binary random variables such that 
    $\prob{X_i = 1} = p_i$. Let $\mu = \sum_{i=1}^n p_i$, and 
    $X = \sum_{i=1}^n X_i$. Then for any $\delta \geq 0$, we have
    \begin{align}
        \prob{X \ge (1+\delta)\mu} \le e^{-\frac{\delta^2 \mu}{2+\delta}}.
    \end{align}
    Moreover, for any $0 \leq \delta \leq 1$, we have
    \begin{align}
        \prob{X \le (1-\delta)\mu} \le e^{-\frac{\delta^2 \mu}{2}}.
    \end{align}
\end{lemma}
\begin{lemma} \label{lemma:indep}
    \cite{chen2021best}.
    Suppose there is a sequence of $n$ Bernoulli trials:
    $X_1, X_2, \ldots, X_n,$
    where the success probability of $X_i$
    depends on the results of
    the preceding trials $X_1, \ldots, X_{i-1}$.
    Suppose it holds that $$\prob{X_i = 1 | X_1 = x_1, X_2 = x_2, \ldots, X_{i-1} = x_{i-1} } \ge \eta,$$ where $\eta > 0$ is a constant and $x_1,\ldots,x_{i-1}$ are arbitrary.
  
    Then, if $Y_1,\ldots, Y_n$ are independent Bernoulli trials, each with probability $\eta$ of
    success, then $$\prob {\sum_{i = 1}^n X_i \le b } \le \prob{\sum_{i = 1}^n Y_i \le b }, $$
    where $b$ is an arbitrary integer.
  
    Moreover, let $A$ be the first occurrence of success in sequence $X_i$.
    Then, $$\ex{A} \le 1/\eta.$$
\end{lemma}
\begin{lemma} \label{lemma:indep2}
    \cite{chen2021best}.
    Suppose there is a sequence of $n+1$ Bernoulli trials:
    $X_1, X_2, \ldots,X_{n+1},$
    where the success probability of $X_i$
    depends on the results of
    the preceding trials $X_1, \ldots, X_{i-1}$,
    and it decreases from 1 to 0.
    Let $t$ be a random variable based on the $n+1$ Bernoulli trials.
    Suppose it holds that 
    $$\prob{X_i = 1 | X_1 = x_1, X_2 = x_2, \ldots, X_{i-1} = x_{i-1}, i\le t } \ge \eta,$$ 
    where $x_1,\ldots,x_{i-1}$ are arbitrary and $0 < \eta < 1$ is a constant.
    Then, if $Y_1,\ldots, Y_{n+1}$ are independent Bernoulli trials, each with probability $\eta$ of
    success, then 
    $$\prob {\sum_{i = 1}^t X_i \le bt } \le \prob{\sum_{i = 1}^t Y_i \le bt }, $$
    where $b$ is an arbitrary integer.
\end{lemma}
% \begin{proof}
%   Let $Z_j=\sum_{i=1}^j X_i \cdot 1_{\{i\le t\}} + \sum_{i=j+1}^{n+1} Y_i \cdot 1_{\{i\le t\}}$,
%   where $$Z_0=\sum_{i=1}^{n+1} Y_i \cdot 1_{\{i\le t\}} = \sum_{i = 1}^t Y_i$$
%   and $$Z_{n+1}=\sum_{i=1}^{n+1} X_i \cdot 1_{\{i\le t\}} = \sum_{i = 1}^t X_i.$$
%   If for any $1\le j \le n+1$, 
%   \begin{equation} \label{ineq:seq}
%     \prob{Z_j \le bt} \le 
%   \prob{Z_{j-1} \le bt},
%   \end{equation}
%   then,
%   $$\prob{Z_{n+1} \le bt} \le
%   \prob{Z_0 \le bt}.$$

%   In the following, we will prove Inequality~\ref{ineq:seq}.
%   \begin{align*}
%     &\prob{Z_j \le bt}\\
%     &= \prob{X_j\cdot 1_{\{j\le t\}}=0, Z_j-X_j\cdot 1_{\{j\le t\}} \le bt-1}\\
%     & \quad +\prob{X_j\cdot 1_{\{j\le t\}}=1, Z_j-X_j\cdot 1_{\{j\le t\}} \le bt-1}\\
%     & \quad +\prob{X_j\cdot 1_{\{j\le t\}}=0, Z_j-X_j\cdot 1_{\{j\le t\}} = bt}\\
%     &= \prob{Z_j-X_j\cdot 1_{\{j\le t\}} \le bt-1}\\
%     &\quad + \prob{1_{\{j\le t\}}=0,Z_j-X_j\cdot 1_{\{j\le t\}} = bt}\\
%     &\quad + \prob{X_j=0,1_{\{j\le t\}}=1,Z_j-X_j\cdot 1_{\{j\le t\}} = bt}\\
%     &= \prob{Z_j-X_j\cdot 1_{\{j\le t\}} \le bt-1}\\
%     &\quad + \prob{1_{\{j\le t\}}=0,Z_j-X_j\cdot 1_{\{j\le t\}} = bt}\\
%     &\quad + \sum_{Z_j-X_j\cdot 1_{\{j\le t\}} = bt,j\le t}
%     \prob{X_j=0 | X_1,\cdots, X_{j-1},Y_{j+1},\cdots, Y_{n+1}, j\le t} \\
%     &\quad \cdot \prob{X_1,\cdots, X_{j-1},Y_{j+1},\cdots, Y_{n+1}, j\le t}\\
%     &\le \prob{Z_j-X_j\cdot 1_{\{j\le t\}} \le bt-1}\\
%     &\quad + \prob{1_{\{j\le t\}}=0,Z_j-X_j\cdot 1_{\{j\le t\}} = bt}\\
%     &\quad + \sum_{Z_j-X_j\cdot 1_{\{j\le t\}} = bt,j\le t}
%     \prob{Y_j=0}\cdot \prob{X_1,\cdots, X_{j-1},Y_{j+1},\cdots, Y_{n+1}, j\le t}\\
%     &= \prob{Z_j-X_j\cdot 1_{\{j\le t\}} \le bt-1}\\
%     &\quad + \prob{1_{\{j\le t\}}=0,Z_j-X_j\cdot 1_{\{j\le t\}} = bt}\\
%     &\quad + \prob{Y_j=0,1_{\{j \le t\}}=1,Z_j-X_j\cdot 1_{\{j\le t\}} = bt}\\
%     &= \prob{Z_j-X_j\cdot 1_{\{j\le t\}} \le bt-1}\\
%     &\quad + \prob{Y_j\cdot 1_{\{j\le t\}}=0,Z_j-X_j\cdot 1_{\{j\le t\}} = bt}\\
%     &= \prob{Z_{j-1}\le bt}.
%   \end{align*}
% \end{proof}
% \begin{lemma}\label{lemma:comb}
%     Given $n=2m$ and $t\le n$, it holds that,
%     \[\sum_{j=\max\{0, t-m\}}^{\lfloor t/2\rfloor}
%     (t-2j)\cdot C_m^i \cdot C_{m-i}^{t-2i}\cdot 2^{t-2i}
%     =n\cdot C_{n-2}^{t-1}.\]
% \end{lemma}
% \begin{proof}
%     We construct a function $F(x,y)=(x^2+2xy+1)^m$ and
%     expand it by the power of $x$ as follows,
%     \[F(x,y)=(x^2+2xy+1)^m = \sum_{t=0}^{n}\left(\sum_{j=\max\{0, t-m\}}^{\lfloor t/2\rfloor}
%     C_m^i \cdot C_{m-i}^{t-2i} \cdot (2y)^{t-2i}\right)x^t.\]
%     Then we calculate the partial derivative of $F$ with respect to $y$
%     based on the above two variants of $F$,
%     \[\frac{\partial F(x,y)}{\partial y}=nx(x^2+2xy+1)^{m-1}
%     =\sum_{t=0}^{n}\left(\sum_{j=\max\{0, t-m\}}^{\lfloor t/2\rfloor}
%     C_m^i \cdot C_{m-i}^{t-2i}\cdot 2^{t-2i}\cdot (t-2i)\cdot y^{t-2i-1}\right)x^t.\]
%     Set $y=1$,
%     \[\left. \frac{\partial F(x,y)}{\partial y}\right|_{y=1}=nx(x+1)^{n-2}
%     =\sum_{t=0}^{n}\left(\sum_{j=\max\{0, t-m\}}^{\lfloor t/2\rfloor}
%     C_m^i \cdot C_{m-i}^{t-2i}\cdot 2^{t-2i}\cdot (t-2i)\right)x^t.\]
%     Then, with $(x+1)^{n-2}=\sum_{t=0}^{n-2}C_{n-2}^t \cdot x^t$, it holds that,
%     \[\sum_{j=\max\{0, t-m\}}^{\lfloor t/2\rfloor}
%     C_m^i \cdot C_{m-i}^{t-2i}\cdot 2^{t-2i}\cdot (t-2i)
%     =n \cdot C_{n-2}^{t-1}.\]
% \end{proof}