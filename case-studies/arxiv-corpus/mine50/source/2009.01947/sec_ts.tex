\section{The \threseq Algorithm} \label{sec:ts}
\begin{algorithm}[t]
	\caption{A parallelizable threshold algorithm for threshold $\tau$}
	\label{alg:threshold}
	\begin{algorithmic}[1]
	\Procedure{\threseq}{$f, \mathcal N, k, \delta, \epsi, \tau$}
	\State \textbf{Input:} evaluation oracle $f:2^{\mathcal N} \to \reals$, constraint $k$, accuracy $\delta$, error $\epsi$, threshold $\tau$
	\State Initialize $A \gets \emptyset$, $A' \gets \emptyset$, $V \gets \mathcal N$, 
	$\ell = \lceil 4\left(\frac{2}{\epsi}\log(n)+\log\left(\frac{n}{\delta}\right)\right) \rceil$ 
	\For{ $j \gets 1$ to $\ell$}  \Comment{Sequential \textbf{for} loop}
		\State Update $V \gets \{ x \in V : \marge{x}{A} \ge \tau \}$ \label{line:threshold-filtering} \Comment{Filtering step w.r.t. $A$}
		\If{ $|V| = 0$ } 
			\State \textbf{return} $A,A'$
		\EndIf
		\State $V \gets$ \textbf{random-permutation}$(V)$ \label{line:threshold-permute}
		\State $s \gets \min \{k-|A|, |V|\}$
		\State $B[1:s] \gets [\textbf{none},\cdots,\textbf{none}]$
		\For{$i \gets 1 $ to $s$ in parallel} \Comment{Parallel gain computation}
			\State $T_{i-1} \gets \{v_1, v_2, \ldots, v_{i-1}\}$ 
			\State \textbf{if} $ \marge{v_i}{A\cup T_{i-1}} \geq  \tau $
				\textbf{then} $B[i] \gets \textbf{true}$ \label{line:threshold-if}
			\State \textbf{elif} $ \marge{v_i}{A\cup T_{i-1}} <  0$
				\textbf{then} $B[i] \gets \textbf{false}$
		\EndFor
		\State $i^* \gets \max\{i:\# \text{\textbf{true}s in } B[1:i] \ge (1-\epsi)i\}$ \Comment{Detection of good filtering next iteration} \label{line:threshold-select_of_istar}
		\State $A \gets A \cup T_{i^*}$ \Comment{$A$ gets all elements}
		\State $A' \gets A' \cup T_{i^*}[\text{where } B \not = \textbf{false}]$ \label{line:threshold-Aprime} \Comment{$A'$ only gets nonnegative-gain elements}
		\If{ $|A| = k$ }
			\State \textbf{return} $A,A'$ 
		\EndIf
	\EndFor
	\State \textbf{return} \textit{failure}
	\EndProcedure
\end{algorithmic}
\end{algorithm}
In this section, we introduce the linear and highly parallelizable
threshold sampling algorithm
\threseq (Alg.~\ref{alg:threshold}).
%This is a modified non-monotone version of \threseq proposed in \shortciteS{chen2021best}.
This algorithm has logarithmic adaptive rounds and linear query calls
with high probability.
Rather than directly solving \tp, 
it returns two sets $A' \subseteq A$ such
that the average marginal gain of elements of $A'$
is exactly larger than the threshold with a small error rate,
and $\marge{x}{A} < \tau$ for any $x \not \in A$. 

\textbf{Overview of Algorithm.}
To obtain large sequences of elements with gains above
$\tau$, the machinery of existing monotone algorithms
\cite{Kazemi2019,chen2021best} is adopted.
These algorithms work by adaptively adding sequences of elements
to a set $A$, where the sequence has been checked in parallel
to have at most an $\epsi$ fraction of the sequence
failing the marginal gain condition.
A uniformly 
random permutation of elements is considered,
where the average marginal
gain being below $\tau$ is detected by a high proportion
of failures in the sequence, which leads to a large number
of elements being filtered out at the next iteration. 


The intuitive reason why this does not directly work for non-monotone
functions (\ie $A$ is not
a solution to \tp) is the same reason 
why \thresam of \shortciteS{Fahrbach2018,Fahrbach2018a} fails:
if one of the elements added fails the marginal gain
condition, it may do so arbitrarily badly and have a large
negative marginal gain.
Moreover, one cannot
simply exclude such elements from consideration, because they
are needed to ensure the filtering step at the next iteration will
discard a large enough fraction of elements. Our solution is to
keep these elements in the set $A$ which is used for filtering,
but only include those elements
with a nonnegative marginal gain in the candidate solution set
$A'$. The membership of $A'$ is known since the gain of every element
was computed in parallel. Moreover, $|A'| \ge (1 - \epsi)|A|$, which
gives the needed relationship on the average marginal gain of each element
of $A'$. 

% \threseq has two nested for loops.
% The outer for loop updates the candidate set $V$ by 
% filtering out elements with small marginal gains,
% randomly shuffles the elements in $V$ before selection,
% and selects a subset in sequence from $V$ to 
% the solution candidate set $A$,
% while the inner for loop figures out the prefix of the subset.
% At last, the solution set $A'$ is selected from $A$.

% In detail, \threseq works as follows.
% First, we filter out the elements in the candidate set $V$
% with marginal gain 
% less than $\tau$ on current $A$.
% Then, a random permutation of $V$ is returned.
% After that, we calculate the marginal gain of element in $V$ 
% based on the current $A$ and the elements before it.
% We call there is a \textbf{true}
% if its marginal gain is larger than the threshold $\tau$,
% or \textbf{false} if it is below 0, or \textbf{none} neither.
% Next, we select the prefix $i^*$ which is the largest $i$ follows that the
% number of \textbf{true}s in the first $i$ elements is more than $(1-\epsi)i$.
% At last, we add the subset with prefix $i^*$ to $A$,
% and delete the \textbf{false} ones as $A'$.
% With the random permutation step and the prefix selection step,
% we can ensure that the subset returned has more than $(1-\epsi)i^*$
% \textbf{true}s and
% we can filter out a constant fraction of elements in candidate set $V$
% with a constant probability.
% By deleting the \textbf{false} ones, the average marginal
% gain of the solution set $A'$ is always larger than $(1-\epsi)\tau$.

We prove the following theorem concerning the performance of \threseq.
\begin{theorem} \label{thm:threshold}
Let $(f,k)$ be an instance of \sm . For any constant $\epsi$, 
the algorithm \threseq outputs $A'\subseteq A \subseteq \mathcal{N}$ such that the following properties hold:
1) The algorithm succeeds with probability at least $1 - \delta/n$.
2) There are $\oh{n/\epsi}$ oracle queries in expectation and $\oh{\log (n/\delta)/\epsi}$ adaptive rounds.
3) It holds that $f(A') \ge (1-\epsi)\tau |A|$.
If $|A| < k$, then $\marge{x}{A} < \tau$ for all $x\in \mathcal{N}$.
4) It also holds that $f(A')\ge f(A)$ and $|A'|\ge (1-\epsi)|A|$
\end{theorem}
A downside of this bifurcated approach is that a downstream algorithm
receives two sets $A, A'$ instead of one from \threseq and must be able to handle
the fact that the gain of an element to the solution $A'$ may be greater
than $\tau$. Fortunately, our approximation algorithms below can easily
handle this restriction. 

\textbf{Overview of Proof.}
The proof of this theorem mainly focuses on two questions: 
1) if a constant fraction of elements can be filtered out
at any iteration with a high probability;
2) if the two sets returned solve \tp indirectly.
In Lemma~\ref{lemma:ThresholdFilterSet},
it is certified that the number of elements being deleted
in the next iteration monotonously increases from 0 to $|V|$ 
as the size of the selected set increases.
Then, by probability lemma and concentration bounds,
Lemma~\ref{lemma:ThresholdProb} answers the first question.
Furthermore, with enough iterations, the candidate set $V$ becomes
empty at some point with a high probability.
Also, since the size of the candidate set $|V|$ exponentially decreases,
intuitively, with logarithmic iterations, the total queries is linear. 
As for \tp, it is obvious that the second property holds with set $A$;
and, by discarding the elements with negative gains in $A$,
the gains of the rest elements increase and follow the first property of \tp.

\begin{proof}[Proof of Theorem~\ref{thm:threshold}]
	\textbf{Success Probability.}
	The algorithm succeeds if $|V|=0$ or $|A|=k$ at termination.
	If we can filter out a constant fraction of $V$ or select
	a subset with $k-|A|$ elements at any iteration with a 
	constant probability, then, with enough iterations,
	the algorithm successfully terminates with a high probability.
	The proofs of lemmas in this section are given in Appendix~\ref{apx:threseq}.
	\begin{restatable}{lemma}{ThresholdFilterSet}
		\label{lemma:ThresholdFilterSet}
        After \textbf{random-permutation} on Line~\ref{line:threshold-permute},
        let $S_i=\{x \in V: \marge{x}{A\cup T_i} < \tau\}$.
        It holds that $|S_0|=0$, $|S_{|V|}|=|V|$, and $|S_{i-1}| \le |S_i|$.
	\end{restatable}
	From Lemma~\ref{lemma:ThresholdFilterSet},
	there exists a point $t$ such that
	$t = \min \{i: |S_i| \ge \epsi |V|/2\}$,
	where the next iteration filters out more than
	$\epsi/2$-fraction of elements if $i^* \ge t$.
	% Next, we prove that, at each iteration, 
	% there is more than 1/2 probability
	% either an $\epsi/2$-fraction of $V$ is filtered out,
	% or the algorithm terminates at this iteration.
	% While the number of elements been deleted in the next
	% iteration increases with the size of the selected set increasing,
	% the portion of trues in $B$ decreases.
	Intuitively, when $i \le t$, there is a high
	probability that
	the portion of \textbf{true}s in $B[1:i]$
	exceeds $1-\epsi$.
	The following lemma is provided.
	\begin{restatable}{lemma}{ThresholdProb} 
		\label{lemma:ThresholdProb}
		It holds that $\prob{i^*<\min\{s, t\}} \le 1/2$.
	\end{restatable}
	% \textbf{Overview of proof.}
	% If $i^*<\min\{s, t\}$, it always holds that,
	% for $i' = \min\{s, t\}$, there are more than $\epsi i'$
	% bad elements in $T_{i'}$.
	% Since $i' \le t$, the probability that an element in $T_{i'}$
	% is bad is less than $\epsi/2$.
	% By Lemma~\ref{lemma:indep2}, Law of Total Probability, 
	% and Markov's Inequality, the probability that $T_{i'}$
	% has more than $\epsi i'$ bad elements is less than 1/2.
	% Thus, the probability of $i^*<\min\{s, t\}$ is less than 1/2.

	% Alg.~\ref{alg:threshold} successfully terminates once
	% $|V| = 0$ or $|A| = k$.
	% If either case happens with a high probability,
	% Alg.~\ref{alg:threshold} succeeds with a high probability.
	Suppose the algorithm does not stop when $|A| = 0$.
	If so, in the following iterations,
	it always holds that $s = 0$ and $T_{i^*}=\emptyset$.
	Lemma~\ref{lemma:ThresholdProb} still holds in this case.
	If there are at least $m=\lceil\log_{1-\epsi/2}(1/n) \rceil$ 
	iterations that $i^* \ge \min\{s, t\}$, the algorithm terminate successfully.
	Define such iteration as a successful iteration.
	Then, the number of successful iterations is a sum of dependent
	Bernoulli random variables. 
	With probability lemma and Chernoff bounds,
	the algorithms is proven to be succeed with probability at least
	$1-\delta/n$ in Appendix~\ref{apx:threseq}.

	\textbf{Objective Values and Marginal Gains.}
	If $|A| < k$, it holds that algorithm terminates with $|V| = 0$.
	So, for any $x \in \mathcal{N}$,
	there exists an iteration $j_{(x)}+1$ such that
	$x$ is filtered out at iteration $j_{(x)}+1$.
	Then, due to submodularity, 
	it holds that $\marge{x}{A} \le \marge{x}{A_{j_{(x)}}}<\tau$.
	% By calculating the gain for each element in the candidate set $V$,
	% the two sets returned by \threseq have the following good properties
	\begin{restatable}{lemma}{ThresholdGood}
		\label{lemma:ThresholdGood}
		Say an element added to the solution set good if its gain is greater than $\tau$.
		$A$ and $A'$ returned by Algorithm~\ref{alg:threshold} hold the following properties: 
		1) There are at least $(1-\epsi)$-fraction of $A$ that is good.
		2) A good element in $A$ is always a good element in $A'$. 
		3) And, any element in $A'$ has non-negative marginal gain when added.
	\end{restatable}
	Lemma~\ref{lemma:ThresholdGood} shows the properties of any single element 
	in $A$ and $A'$.
	Since $A'$ is a subset of $A$ with all the positive gain elements,
	% by Property (1) and (2) in Lemma~\ref{lemma:ThresholdGood},
	it holds that $|A'|\ge (1-\epsi)|A|$.
	By deleting an element in a set of sequence,
	the marginal gains of the other elements is nondecreasing due to 
	the diminishing property of submodular function.
	% By Property (1) and (2), at least $(1-\epsi)|A|$ elements in $A'$ are good,
	% and the rest of $A'$ has non-negative marginal gains by Property (3).
	% Therefore, $f(A')\ge (1-\epsi)\tau |A|$.
	For any $x \in A$, let $A_{(x)}$ be a subsequence of $A$ before
	$x$ is added into $A$. Define $A_{(x)}'$ analogously.
	By Lemma~\ref{lemma:ThresholdGood}, it holds that
	\[f(A') = \sum_{x \in A'}\marge{x}{A_{(x)}'}
	\ge \sum_{x \in A'}\marge{x}{A_{(x)}}
	+\sum_{x \in A\backslash A'}\marge{x}{A_{(x)}}
	\ge f(A),\]
	\[f(A')= \sum_{x \in A'}\marge{x}{A_{(x)}'}
	\ge \sum_{x \in A', x \text{ is good}}\marge{x}{A_{(x)}'}
	% \ge \sum_{x \in A', x \text{ is good}}\marge{x}{A_{(x)}}
	\ge (1-\epsi)\tau |A|.\]

	\textbf{Adaptivity and Query Complexity.}
	In Alg.~\ref{alg:threshold}, the oracle queries occur on 
	Line~\ref{line:threshold-filtering} and~\ref{line:threshold-if}.
	Since filtering and inner \textbf{for} loop can be done in parallel, 
	there are constant adaptive rounds in an iteration.
	Therefore, the adaptivity is $\oh{\ell} = \oh{\log(n/\delta)/\epsi}$.
	
	As for the query complexity, 
	let $V_j$ be the set $V$ after filtering on Line~\ref{line:threshold-filtering}
	in iteration $j$.
	There are $|V_{j-1}|+1$ and $|V_{j}|+1$ query calls on 
	Line~\ref{line:threshold-filtering} and~\ref{line:threshold-if},
	respectively.
	Suppose the number of
	iterations that successfully filter out more than 
	$\epsi/2$-fraction of $V$ equals $i$
	before current iteration $j$.
	The size of $|V_j|$ can be bounded by $n{(1-\epsi/2)}^{i}$.
	We show that the expected total queries are $\oh{n/\epsi}$ 
	in Appendix~\ref{apx:threseq}.
\end{proof}

%%% Local Variables:
%%% mode: latex
%%% TeX-master: "main"
%%% End: