
\documentclass[accepted,nohyperref]{article}

\usepackage{fullpage}
\usepackage{parskip}


\usepackage{authblk}


\usepackage{microtype}
\usepackage{graphicx}
\usepackage{subcaption}
\usepackage{booktabs} %

\usepackage{hyperref}


\usepackage{xcolor}
\usepackage{algorithm}
\usepackage[noend]{algpseudocode}



\usepackage{amsmath}
\usepackage{amssymb}
\usepackage{mathtools}
\usepackage{amsthm}

\usepackage[capitalize,noabbrev]{cleveref}

\theoremstyle{plain}
\newtheorem{theorem}{Theorem}[section]
\newtheorem{proposition}[theorem]{Proposition}
\newtheorem{lemma}[theorem]{Lemma}
\newtheorem{corollary}[theorem]{Corollary}
\theoremstyle{definition}
\newtheorem{definition}[theorem]{Definition}
\newtheorem{assumption}[theorem]{Assumption}
\newtheorem{remark}[theorem]{Remark}


\definecolor{linkcolor}{RGB}{83,83,182}

\hypersetup{
    colorlinks=true,
    citecolor=linkcolor,
    linkcolor=linkcolor,
    urlcolor=linkcolor
}

\usepackage[backend=biber,natbib=true,style=authoryear,doi=false,isbn=false,url=false,eprint=false,giveninits=true,maxbibnames=200,maxcitenames=2,mincitenames=1,uniquename=false,uniquelist=false,dashed=false]{biblatex}

\newcommand{\noopsort}[1]{#1}


\addbibresource{references.bib}%

\DeclareSourcemap{
  \maps[datatype=bibtex]{
    \map{
      \step[fieldsource=url, match=\regexp{http://(dx.doi.org/|dl.acm.org/)}, final]
      \step[fieldset=url, null]
      \step[fieldset=urldate, null]
    }
  }
}
\setlength\bibitemsep{1ex}


\title{Differentially Private Coordinate Descent \\
  for Composite Empirical Risk Minimization}

\date{}

\author[1]{Paul Mangold}
\author[1]{Aurélien Bellet}
\author[2,3]{Joseph Salmon}
\author[4]{Marc Tommasi}

\affil[1]{Univ. Lille, Inria,  CNRS, Centrale Lille, UMR 9189 - CRIStAL, F-59000 Lille, France}
\affil[2]{IMAG, Univ Montpellier, CNRS, Montpellier, France}
\affil[3]{Institut Universitaire de France (IUF)}
\affil[4]{Univ. Lille, CNRS, Inria, Centrale Lille,  UMR 9189 - CRIStAL, F-59000 Lille, France}

%
% this file defines packages to be used and new commands
%
\usepackage{comment}
\usepackage{caption}
\usepackage{stackengine} % required for 'ifdef. this package is already used by tog but not by iccv

% all \ifxxx commands should be false. the true values should be defined in the main files.

\ifundef{\ifanonymous}
{\newif\ifanonymous \anonymousfalse}
{}

\ifundef{\ifdraft}
{\newif\ifdraft \draftfalse}
{}

\ifundef{\ifeccv}
{\newif\ifeccv \eccvfalse}
{}

\ifundef{\ifcvpr}
{\newif\ifcvpr \cvprfalse}
{}

\ifundef{\ifcvprreview}
{\newif\ifcvprreview \cvprreviewfalse}
{}

% \ifcvprreview
% {\global\anonymoustrue} % must be global. maybe because of the makeletter env?
% \else
% {\global\anonymousfalse}
% \fi

% \ifappendix tells us whether the textual sup mat is in the appendix or in a separate doc 
\ifundef{\ifappendix}
{\newif\ifappendix \appendixfalse}
{}

\ifundef{\ifarxiv}
{\newif\ifarxiv \arxivfalse}
{}

\ifundef{\ificcv}
{\newif\ificcv \iccvfalse}
{}

\ifundef{\iftog}
{\newif\iftog \togfalse}
{}

\ifundef{\ificcvfinal}
{\newif\ificcvfinal \iccvfinalfalse}
{}

\ifarxiv
\appendixtrue
\fi

\iftog
\appendixtrue
\fi

\ifanonymous
{\global\iccvfinalfalse}
\else
{\global\iccvfinaltrue}
\fi

\ifcvpr{
\newcommand{\citet}{\cite}
\newcommand{\keywords}{\null}
}\fi

\ifeccv
\else
\usepackage{times}
\fi
% \usepackage[numbers,sort]{natbib}
\usepackage{ifthen}
\usepackage{cancel}
\usepackage{epsfig}
\usepackage{graphicx}
\usepackage{amsmath}
\iftog {} \else \usepackage{amssymb} \fi % fails for TOG style

\usepackage{xcolor}
\usepackage{enumitem}
\usepackage{wrapfig}
\usepackage{float}
\usepackage{xspace} % needed for \etal etc.
% \usepackage{hyperref}
\usepackage{orcidlink}

\newcommand{\bK}{{\bf K}}
\newcommand{\bQ}{{\bf Q}}
\newcommand{\bW}{{\bf W}}
\newcommand{\bV}{{\bf V}}
\newcommand{\bJ}{{\bf J}}
\newcommand{\bP}{{\bf P}}
\newcommand{\bL}{{\bf L}}
\newcommand{\bT}{{\bf T}}
\newcommand{\bD}{{\bf D}}
\newcommand{\bF}{{\bf F}}
\newcommand{\bG}{{\bf G}}
\newcommand{\bI}{{\bf I}}
\newcommand{\bc}{{\bf c}}
\newcommand{\bff}{{\bf f}}
\newcommand{\bp}{{\bf p}}
\newcommand{\bq}{{\bf q}}
\newcommand{\br}{{\bf r}}
\newcommand{\bss}{{\bf s}}
\newcommand{\bv}{{\bf v}}
\newcommand{\bk}{{\bf k}}
\newcommand{\Loss}{\mathcal{L}}
\newcommand{\mm}{\mathcal{M}}
\newcommand{\mms}{\mathcal{S}}
\newcommand{\Dp}{\bbd_{\text{p}}}
\newcommand{\Ds}{\bbd_{\text{s}}}
\newcommand{\cc}{\mathcal{C}}
\newcommand{\nn}{\mathcal{N}}
\newcommand{\pp}{\mathcal{P}}
\newcommand{\qq}{\mathcal{Q}}
\newcommand{\bbe}{\mathbb{E}}
\newcommand{\bbr}{\mathbb{R}}

\iftog 
\else
\newcommand{\shortcite}{\cite}
\fi

%%%%%%% or's imports
\usepackage{url}
\usepackage{graphics}
% \usepackage{layouts} % causes two warnings, not sure this package is needed
\usepackage[normalem]{ulem} % ulem defines \sout (strikethrough). normalem disables the change of emph from italic to underline.
\usepackage{multirow}
\usepackage{tabu,stackengine}
\usepackage{wrapfig}
% \usepackage{floatrow}
\usepackage{booktabs}
\usepackage{soul}
% Table float box with bottom caption, box width adjusted to content
% \newfloatcommand{capbtabbox}{table}[][\FBwidth]
% \usepackage{caption}
% \usepackage{subcaption}
\usepackage{makecell}
\ifcvpr
\else
\usepackage{cleveref}
\crefname{section}{Sec.}{Secs.}
\Crefname{section}{Section}{Sections}
\Crefname{table}{Table}{Tables}
\crefname{table}{Tab.}{Tabs.}
\fi

\newcommand{\todo}[1]{{\color{red}#1}}
\newcommand{\tbd}{{\color{red}xxx}}
\newcommand{\bluebold}[1]{{\textbf{\color{blue} #1}}}
\newcommand{\graybluebold}[1]{{\textbf{\color{blue!50} #1}}} % 50% blue and 50% white, to be used when cam params are perturbed

\definecolor{applegreen}{rgb}{0.55, 0.71, 0.0}
\definecolor{burgundy}{rgb}{0.5, 0.0, 0.13}
\definecolor{calpolypomonagreen}{rgb}{0.12, 0.3, 0.17}

\ifdraft
\newcommand{\setcolor}[1]{\color{#1}}
\newcommand{\sr}[1]{{\color{violet} #1}}
\newcommand{\sroo}[1]{{\color{calpolypomonagreen} #1}}
\newcommand{\srrep}[2]{\sout{#1} \sr{#2}}
\newcommand{\bg}[1]{{\color{orange} #1}}
\else
\newcommand{\setcolor}[1]{}
\newcommand{\sr}[1]{{#1}}
\newcommand{\sroo}[1]{#1}
\newcommand{\srrep}[2]{#2}
\newcommand{\bg}[1]{#1}
\fi
\newcommand{\srr}[1]{{\color{red}\textbf{SR:} #1}}
\newcommand{\srm}[1]{{\color{violet}\textbf{SR:} #1}}
\newcommand{\sro}[1]{{\color{calpolypomonagreen}\textbf{SR:} #1}}

\newcommand{\dcc}[1]{{\color{red}\textbf{DC:} #1}}
\newcommand{\dc}[1]{{\color{red} #1}}

\newcommand{\rgc}[1]{{\color{cyan}\textbf{RG} #1}}
\newcommand{\rg}[1]{{\color{cyan} #1}}

\newcommand{\brr}[1]{{\color{red}\textbf{BG:} #1}}
\newcommand{\brrep}[2]{\sout{#1} \bro{#2}}
%\newcommand{\brchange}[2]{\sout{~\mbox{#1}} \bro{#2}}
\newcommand{\bro}[1]{{\color{orange}\textbf{BG:} #1}}
\newcommand{\brchange}[2]{\sout{#1} \bro{#2}}

\newcommand{\gum}[1]{{\color{violet}\textbf{GA:} #1}}
\newcommand{\gur}[1]{{\color{red}\textbf{GA:} #1}}
\newcommand{\gurep}[2]{\sout{#1} \bro{#2}}

\newcommand{\algoname}{FLEX}
\newcommand{\projectpage}{\url{https://briang13.github.io/FLEX}}

\setlength{\abovedisplayshortskip}{-30pt}
\setlength{\belowdisplayshortskip}{-30pt}
\setlength{\abovedisplayskip}{-30pt}
\setlength{\belowdisplayskip}{-30pt}


\makeatletter
\iftog
\newcommand{\sectiontinyvert}{\section}
\newcommand{\paragraphtinyvert}{\paragraph}
\newcommand{\paragraphnovert}{\paragraph}
\newcommand{\subparagraphnovert}{\subparagraph}
\else

% \renewcommand\section{\@startsection{section}{1}{\z@}%
%                       {-18\p@ \@plus -4\p@ \@minus -4\p@}%
%                       {12\p@ \@plus 4\p@ \@minus 4\p@}%
%                       {\normalfont\large\bfseries\boldmath
%                         \rightskip=\z@ \@plus 8em\pretolerance=10000 }}
% \renewcommand\subsection{\@startsection{subsection}{2}{\z@}%
%                       {-18\p@ \@plus -4\p@ \@minus -4\p@}%
%                       {8\p@ \@plus 4\p@ \@minus 4\p@}%
%                       {\normalfont\normalsize\bfseries\boldmath
%                         \rightskip=\z@ \@plus 8em\pretolerance=10000 }}

% \renewenvironment{abstract}{%
%       \list{}{\advance\topsep by0.35cm\relax\small
%       \leftmargin=1cm
%       \labelwidth=\z@
%       \listparindent=\z@
%       \itemindent\listparindent
%       \rightmargin\leftmargin}\item[\hskip\labelsep
%                                     \bfseries\abstractname]}
%     {\endlist}

% \renewenvironment{equation*}{%
%   \mathdisplay@push
%   \st@rredtrue \global\@eqnswfalse
%   \color{black}
%   \mathdisplay{equation*}%
% }{%
%   \endmathdisplay{equation*}%
%   \mathdisplay@pop
%   \ignorespacesafterend
% }


\newenvironment{myequation}{%
% \list{}{
%\topsep by-70pt 
% \smallskip
% \vspace{-40\p@}
  \incr@eqnum
  \mathdisplay@push
  \st@rredfalse \global\@eqnswtrue
  \mathdisplay{equation}%
% }
}{%
  \endmathdisplay{equation}%
  \mathdisplay@pop
  \ignorespacesafterend
%   \endlist
}

% \newenvironment{equation}{%
%   \mathdisplay@push
%   \st@rredfalse \global\@eqnswtrue
%   \mathdisplay{equation}%
%   \incr@eqnum\mathopen{}%
% }{%
%   \endmathdisplay{equation}%
%   \mathdisplay@pop
%   \ignorespacesafterend
% }


\newenvironment{myequation*}{%
\list{}{\topsep by-7pt
\mathdisplay@push
  \st@rredtrue \global\@eqnswfalse
  \mathdisplay{equation}%
}}{%
  \endmathdisplay{equation}%
  \mathdisplay@pop
  \ignorespacesafterend
  \endlist
}

\newenvironment{myabstract}{%
      \list{}{\advance\topsep by-7pt\relax\small
      \leftmargin=0.95cm
      \labelwidth=\z@
      \listparindent=\z@
      \itemindent\listparindent
      \rightmargin\leftmargin}\item[\hskip\labelsep
                                    \bfseries\abstractname]}
    {\endlist}

\newcommand{\sectiontinyvert}{
  \@startsection{section}               % name 
  {1}                                   % level
  {\z@}                                 % indent
  {10pt \@plus 3pt \@minus 0pt}         % before skip
  {5pt \@plus 2pt \@minus 0pt}          % after skip
  {\normalfont\large\bfseries\boldmath\rightskip=\z@ \@plus 8em\pretolerance=10000}          % style
}

\newcommand{\subsectiontinyvert}{
  \@startsection{subsection}            % name 
  {2}                                   % level
  {\z@}                                 % indent
  {10pt \@plus 3pt \@minus 0pt}         % before skip
  {5pt \@plus 2pt \@minus 0pt}          % after skip
  {\normalfont\normalsize\bfseries\boldmath\rightskip=\z@ \@plus 8em\pretolerance=10000}     % style
}
\newcommand\subsubsectiontinyvert{
    \@startsection{subsubsection}
    {3}
    {\z@}
    {10pt \@plus 3pt \@minus 0pt}%
    {-0.5em \@plus -0.22em \@minus -0.1em}%
    {\normalfont\normalsize\bfseries\boldmath}}
    
% \renewcommand\paragraph{\@startsection{paragraph}{4}{\z@}%
%                       {-12\p@ \@plus -4\p@ \@minus -4\p@}%
%                       {-0.5em \@plus -0.22em \@minus -0.1em}%
%                       {\normalfont\normalsize\itshape}}
\newcommand{\paragraphtinyvert}{%
  \@startsection{paragraph}{4}%
  {\z@}{1ex \@plus 0.0ex \@minus 0.2ex}{-1em}% was "plus 0.2 ex"
  {\normalfont\normalsize\bfseries\boldmath} % like subsubsection in llncs
}
\newcommand{\paragraphnovert}{%
  \@startsection{paragraph}{4}%
  {\z@}{0ex \@plus 0ex \@minus 0ex}{-0.5 em}%
  {\normalfont\normalsize\bfseries}%
}
\newcommand{\subparagraphnovert}{%
  \@startsection{subparagraph}{5}%
  {3ex}{0ex \@plus 0ex \@minus 0ex}{-1em}%
  {\normalfont\normalsize\bfseries}%
}

% If the paper title is too long for the running head, you can set
% an abbreviated paper title here
% \newcommand{\orcid}[1]{\href{https://orcid.org/#1} {\protect\includegraphics[width=8pt]{./images/ORCID-iD_icon-128x128.png}}}
% \newcommand{\orcid}[1]{\orcidID{#1}}
\newcommand{\orcid}[1]{\orcidlink{#1}}

\fi

\ificcv
\else
% the following are copied from iccv.sty
% Add a period to the end of an abbreviation unless there's one
% already, then \xspace.
\DeclareRobustCommand\onedot{\futurelet\@let@token\@onedot}
\def\@onedot{\ifx\@let@token.\else.\null\fi\xspace}

\def\eg{\emph{e.g}\onedot} \def\Eg{\emph{E.g}\onedot}
\def\ie{\emph{i.e}\onedot} \def\Ie{\emph{I.e}\onedot}
\def\cf{\emph{c.f}\onedot} \def\Cf{\emph{C.f}\onedot}
\def\etc{\emph{etc}\onedot} \def\vs{\emph{vs}\onedot}
\def\wrt{w.r.t\onedot} \def\dof{d.o.f\onedot}
\def\etal{\emph{et al}\onedot}
\fi

\makeatother


\begin{document}

\maketitle



\begin{abstract}
  Machine learning models can leak information about the data used to
  train them. To mitigate this issue, Differentially Private (DP) variants of
  optimization
  algorithms like Stochastic Gradient Descent (DP-SGD) have been
  designed to trade-off utility for privacy in Empirical Risk
  Minimization (ERM) problems. In this paper, we propose
  Differentially Private proximal Coordinate Descent (DP-CD), a new
  method to solve composite DP-ERM problems. We derive utility
  guarantees through a novel theoretical analysis of inexact
  coordinate descent. Our results show that, thanks to larger step
  sizes, DP-CD can exploit imbalance in gradient coordinates to
  outperform DP-SGD. We also prove new lower bounds for composite
  DP-ERM under coordinate-wise regularity assumptions, that are nearly
  matched by DP-CD. For practical implementations, we propose to clip
  gradients using coordinate-wise thresholds that emerge
  from our theory, avoiding costly hyperparameter tuning. Experiments
  on real and synthetic data support our results, and show that DP-CD
  compares favorably with DP-SGD.
\end{abstract}

version https://git-lfs.github.com/spec/v1
oid sha256:f7f279fa0f93cb2842457a52f2e0361e29261eb78732a2e3ff4c6aebddfefb19
size 6684


\section{Preliminaries}
\label{sec:preliminaries}


In this section, we introduce important technical notions that will be used
throughout the paper.

\paragraph{Norms.}
We start by defining two conjugate norms that will be crucial in our analysis,
for they allow to keep track of coordinate-wise quantities.
Let $\scalar{u}{v} = \sum_{j=1}^p u_i v_i$ be the Euclidean dot product, let $M = \diag(M_1, \dots, M_p)$ with $M_1, \dots, M_p > 0$, and
\begin{align*}
  \norm{w}_M = \sqrt{\scalar{Mw}{w}}\enspace,\quad\quad\quad
  \norm{w}_{M^{-1}} = \sqrt{\scalar{M^{-1}w}{w}} \enspace.
\end{align*}
When $M$ is the identity matrix $I$, the $I$-norm $\norm{\cdot}_I$ is the standard $\ell_2$-norm $\norm{\cdot}_2$.

\paragraph{Regularity assumptions.}
We recall classical regularity assumptions along with ones
specific to the coordinate-wise setting.
We denote by $\nabla f$ the gradient of
a differentiable function $f$, and by $\nabla_j f$ its $j$-th coordinate.
We denote by $e_j$ the $j$-th vector of $\RR^p$'s canonical basis.

\textit{Convexity:} a differentiable function $f : \RR^p
  \rightarrow \RR$ is convex if
for all $v, w \in \RR^p$,
$f(w) \ge f(v) + \scalar{\nabla f(v)}{w - v}$.

\textit{Strong convexity:} a differentiable function $f : \RR^p \rightarrow
  \RR$ is
$\mu_M$-strongly-convex \wrt the norm $\smash{\norm{\cdot}_M}$ if
for all $v, w \in \RR^p$,
$f(w) \ge f(v) + \scalar{\nabla f(v)}{w - v} + \frac{\mu_M}{2}\norm{w - v}_M^2$.
The case $M_1=\cdots=M_p=1$ recovers standard $\mu_I$-strong convexity \wrt
the $\ell_2$-norm.

\textit{Component Lipschitzness:} a function $f : \RR^p \rightarrow \RR$
is
$L$-component-Lipschitz for $L = (L_1,\dots,L_p)$ with $L_1,\dots,L_p > 0$ if
for all $w \in \RR^p$, $t \in \RR$ and $j \in [p]$,
$\abs{f(w + t e_j) - f(w)} \le L_j \abs{t}$.
It is $\Lambda$-Lipschitz if for all $v, w \in \RR^p$,
$\abs{f(v) - f(w)} \le \Lambda \norm{v - w}_2$.


\textit{Component smoothness:} a differentiable function $f : \RR^p
  \rightarrow \RR$ is
$M$-component-smooth for $M_1,\dots,M_p > 0$ if
for all $v, w \in \RR^p$,
$f(w) \le f(v) + \scalar{\nabla f(v)}{w - v} + \frac{1}{2}\norm{w - v}_{M}^2$.
When $M_1=\dots=M_p=\beta$, $f$ is said to be $\beta$-smooth.

The above component-wise regularity hypotheses are not restrictive:
$\Lambda$-Lipschitzness
implies $(\Lambda, \dots, \Lambda)$-component-Lipschitzness and
$\beta$-smoothness implies $(\beta, \dots, \beta)$-component-smoothness.
Yet, the actual component-wise constants of a function can be much
lower than what can be deduced from their global counterparts.
This will be crucial for our analysis and in the performance of DP-CD.

\begin{remark}
  \label{rmq:constrained-regularity-assumptions}
  When $\psi$ is the characteristic function of a convex set (with separable
  components), the regularity assumptions only need to hold on this
  set. This allows considering problem~\eqref{eq:dp-erm} with a smooth
  objective under box-constraints.
\end{remark}




\paragraph{Differential privacy (DP).}

Let $\cD$ be a set of datasets and $\cF$ a set of possible outcomes.
Two datasets $D, D' \in \cD$ are said \textit{neighboring}
(denoted by $D \sim D'$) if they differ on at most one element.

\begin{definition}[Differential Privacy, \citealt{dwork2006Differential}]
  A randomized algorithm
  $\cA : \mathcal D
    \rightarrow \mathcal F$ is $(\epsilon, \delta)$-differentially private if,
  for all neighboring datasets $D, D' \in \mathcal D$ and all
  $S \subseteq \mathcal F$ in the range of $\cA$:
  \begin{align*}
    \prob{\cA(D) \in S} \le \exp(\epsilon) \prob{\cA(D') \in S} + \delta \enspace.
  \end{align*}
\end{definition}
The value of a function $h: \mathcal D \rightarrow \mathbb R^p$ can be privately
released using the Gaussian mechanism, which adds centered Gaussian noise
to $h(D)$ before releasing it \citep{dwork2013Algorithmic}.
The scale of the noise is calibrated to the sensitivity $
  \Delta(h)
  = \sup_{D \sim D'} \norm{h(D) - h(D')}_2$ of $h$.
In our setting, we will perturb coordinate-wise gradients: we denote by
$\Delta(\nabla_j \ell)$ the sensitivity of the $j$\nobreakdash-th coordinate
of gradient of the loss function $\ell$ with respect to the data.
When $\ell(\cdot;d)$ is $L$-component-Lipschitz for all $d\in\mathcal{X}$, upper
bounds on these sensitivities are readily available: we have
$\Delta(\nabla_j\ell) \le 2L_j$ for any $j\in[p]$ (see \Cref{sec:lemma-sensitivity}).
The following quantity, relating the coordinate-wise sensitivities of gradients
to coordinate-wise smoothness is central in our analysis:
\begin{align}
  \label{eq:delta-lipschitz-norm}
  \Delta_{M^{-1}}(\nabla \ell)
  = \Big(\sum_{j=1}^p \frac{1}{M_j} \Delta (\nabla_j\ell)^2\Big)^{\frac{1}{2}}
  \leq \! 2 \norm{L}_{M^{-1}}\enspace.
\end{align}
In this paper, we consider the classic central model of DP, where a trusted
curator has access to the raw dataset and releases a model trained on this
dataset\footnote{In fact, our privacy guarantees hold even if all
  intermediate iterates are released (not just the final model).}.





\section{Differentially Private Coordinate Descent}
\label{sec:diff-priv-coord}

In this section, we introduce the
Differentially Private proximal Coordinate Descent (DP-CD) algorithm to
solve problem~\eqref{eq:dp-erm} under $(\epsilon,\delta)$-DP constraints.
We first describe our algorithm, show how to parameterize it to
satisfy the desired privacy constraint, and prove corresponding utility
results.
Finally, we compare these utility guarantees with DP-SGD.

\subsection{Private Proximal Coordinate Descent}

Let $D = \{d_1, \dots, d_n\} \in \cX^n$ be a dataset.
We denote by $f(w) = \frac{1}{n}\sum_{i=1}^n \ell(w; d_i)$ the $M$-component-smooth
part of \eqref{eq:dp-erm},
by $\psi(w) = \sum_{j=1}^p \psi_j(w_j)$ its separable part,
and let $F(w) = f(w) + \psi(w)$.
Proximal coordinate descent methods \cite{richtarik2014Iteration}
solve problem~\eqref{eq:dp-erm} through iterative proximal gradient
steps along each coordinate of~$F$.
Formally, given $w \in \RR^p$ and $j \in [p]$, the $j$-th coordinate of
$w$ is updated as follows:
\begin{align}
  \label{eq:proximal-update-nonoise}
  w_j^+ = \prox_{\gamma_j\psi_j} \big(w_j - \gamma_j
  \nabla_j f(w_t)\big)\enspace,
\end{align}
where $\gamma_j>0$ is the step size and $\prox_{\gamma_j\psi_j}
(w)= \smash{\argmin_{v\in\RR^p} \big
\{ \frac{1}{2} \norm{v - w}_2^2 + \gamma_j\psi_j(v) \big\}}$ is the proximal
operator associated with $\psi_j$ \citep{parikh2014Proximal}.

\begin{algorithm*}[t]
  \caption{Differentially Private Proximal Coordinate Descent Algorithm
    (DP-CD).}
  \label{algo:dp-cd}
  \textbf{Input}:
  noise scales $\sigma = (\sigma_1, \dots, \sigma_p)$ for $\sigma_1,\dots,\sigma_p > 0$;
  step sizes $\gamma_1,\dots,\gamma_p > 0$;
  initial point $\bar w^0 \in \mathbb{R}^p$;
  iteration budgets $T, K > 0$.
  \begin{algorithmic}[1]
    \For{$t = 0, \dots, T-1$}
    \State Set $\theta^0 = \bar w^t$
    \For{$k = 0, \dots, K-1$}
    \State Pick $j$ from $\{1, \dots, p\}$ uniformly at random
    \State Draw $\eta_j \sim \mathcal N(0, \sigma_j^2)$
    \label{algo-line:noise-generation}
    \State Set $\theta^{k+1} = \theta^k$
    \State Set $\theta^{k+1}_{j} = \prox_{\gamma_{j}\psi_{j}} (\theta^{k}_{j} -
      \gamma_{j} (\nabla_{j} f(\theta^k) + \eta_j))$
    \label{algo-line:coordinate-minimization-update}
    \vspace*{-.02cm}
    \EndFor
    \State Set $\bar w_{t+1} = \frac 1K \sum_{k=1}^K \theta^k$
    \label{algo-line:periodic_avg}
    \EndFor
    \State \Return $ w_{priv} = \bar w_T$
  \end{algorithmic}
\end{algorithm*}
Update \eqref{eq:proximal-update-nonoise} only requires the computation of the
$j$-th entry of the gradient. To satisfy differential privacy, we perturb this
gradient entry with additive Gaussian noise of variance $\sigma_j^2$.
The complete DP-CD procedure is shown in \Cref{algo:dp-cd}.
At each iteration, we pick a coordinate uniformly at random
and update according to~\eqref{eq:proximal-update-nonoise}, albeit with noise
addition (see line \ref{algo-line:coordinate-minimization-update}).
For technical reasons related to our analysis, we use a
periodic averaging scheme (line~\ref{algo-line:periodic_avg}).
This scheme is similar to DP-SVRG \citep{johnson2013Accelerating}, although
no variance reduction is required since DP-CD computes coordinate gradients
over the whole dataset.



\subsection{Privacy Guarantees}
\label{sec:privacy-guarantees}

For \Cref{algo:dp-cd} to satisfy $(\epsilon,\delta)$-DP, the noise scales
$\sigma=(\sigma_1,\dots,\sigma_p)$ can be calibrated as given in \Cref
{thm:dp-cd-privacy}. %

\begin{theorem}
  \label{thm:dp-cd-privacy}
  Assume $\ell(\cdot;d)$ is $L$-component-Lipschitz $\forall d\in\cX$.
  Let $\epsilon \leq 1$ and $\delta < 1/3$.
  If $\sigma_j^2 = \frac{12L_j^2 TK \log(1/\delta)}{n^2\epsilon^2}$
  for all $j \in [p]$,
  then \Cref{algo:dp-cd} satisfies $(\epsilon, \delta)$-DP.
\end{theorem}
\begin{sketchproof}(Complete proof in \Cref{sec:proof-privacy}).
  We track the privacy loss using Rényi differential privacy (RDP),
  which gives better guarantees than $(\epsilon,\delta)$-DP for the
  composition
  of Gaussian mechanisms \citep{mironov2017Renyi}.  The
  $j$\nobreakdash-th entry of $\nabla f$ has sensitivity
  $\Delta(\nabla_j f) = {\Delta(\nabla_j \ell)}/{n} \le {2L_j}/{n}$.
  By the Gaussian mechanism each iteration of DP-CD is
  $(\alpha, \frac{2 L_j^2 \alpha}{n^2\sigma_j^2})$-RDP for all
  $\alpha > 1$. The composition theorem for RDP gives a global RDP
  guarantee for DP-CD, that we convert to $(\epsilon,\delta)$-DP using
  Proposition~3 of \citet{mironov2017Renyi}. Choosing $\alpha$
  carefully finally proves the result.
\end{sketchproof}


The dependence of the noise scales on $\epsilon$, $\delta$, $n$ and
$TK$ (the number of updates) in \Cref{thm:dp-cd-privacy} is standard
in DP-ERM. However, the noise is calibrated to the loss function's
\emph{component}-Lipschitz constants. These can be much lower their
global counterpart, the latter being used to calibrate the noise
in DP-SGD algorithms. This will be crucial for DP-CD to achieve better utility
than DP-SGD in some regimes.
We also note that, unlike DP-SGD, DP-CD does not rely on privacy
amplification by subsampling \citep{Balle_subsampling,mironov2019Enyi}, and
thereby avoids the approximations required by these
schemes to bound the privacy loss.

\begin{remark}
  Theorem~\ref{thm:dp-cd-privacy} assumes $\epsilon \in (0,1]$ to give
  a simple closed form for the noise scales. In practice we compute
  tighter values numerically using Rényi DP formulas directly (see
  Eq.~\ref{eq:full_privacy_formula} in \Cref{sec:proof-privacy}),
  removing the need for this assumption.
\end{remark}



\subsection{Utility Guarantees}
\label{sec:utiltity-analysis-cd}

We now state our central result on the utility of DP-CD
for the composite DP-ERM problem.
As done in previous work, we use the asymptotic notation $\widetilde O$ to hide
non-significant logarithmic factors. Non-asymptotic utility bounds can be
found in \Cref {sec-app:proof-utility}.
\begin{theorem}
  \label{thm:cd-utility}
  Let $\ell(\cdot; d)$ be a convex and $L$-component-Lipschitz loss
  function for all $d \in \cX$, and $f$ be convex and
  $M$-component-smooth. Let
  $\psi : \RR^p \rightarrow \RR$ be a convex and separable function.
  Let $\epsilon \leq 1, \delta < 1/3$ be the privacy budget.  Let $w^*$
  be a minimizer of $F$ and $F^* = F(w^*)$.
  Let $w_{priv}\in\mathbb{R}^p$ be the output of \Cref{algo:dp-cd} with step
  sizes $\gamma_j = {1}/{M_j}$, and noise
  scales $\sigma_1,\dots,\sigma_p$ set as in Theorem~\ref{thm:dp-cd-privacy} (with $T$ and $K$ chosen below) to ensure
  $(\epsilon,\delta)$-DP.
  Then, the following holds:
  \begin{enumerate}[leftmargin=12pt]
    \item For $F$ convex, $K=O\left( \frac{R_M \sqrt{p} n \epsilon}{\norm{L}_{M^{-1}}} \right)$, and $T = 1$, then:
          \begin{align*}
            \expec{}{F(w_{priv}) - F^*}
            = \widetilde O\bigg(\frac{\sqrt{p \log(1/\delta)}}{n\epsilon}
            \norm{L}_{M^{-1}} R_M\bigg)\enspace,
          \end{align*}
          where $R_M = \max(\sqrt{F(w^0) - F(w^*)}, \norm{w^0 - w^*}_M)$
          and more simply $R_M = \norm{w^0 - w^*}_M$ when $\psi = 0$.
    \item For $F$ $\mu_M$-strongly convex w.r.t. $\smash{\norm{\cdot}_M}$,
          $K = O\left(p/\mu_M\right)$, and
          $T = O\left( \log(n\epsilon \mu_M/p \norm{L}_{M^{-1}}) \right)$, then:
          \begin{align*}
            \expec{}{F(w_{priv}) - F^*}
            = \widetilde O\bigg(\frac{p\log(1/\delta)}{n^2 \epsilon^2}
            \frac{\norm{L}_{M^{-1}}^2}{\mu_M}
            \bigg)\enspace.
          \end{align*}
  \end{enumerate}%
  Expectations are over the randomness of the algorithm.
\end{theorem}
\begin{sketchproof}(Complete proof in \Cref{sec-app:proof-utility}).
  Existing analyses of CD fail to track the noise tightly
  across coordinates when adapted to the private setting. Contrary to
  these classical analyses, we prove a recursion on
  $\mathbb{E}\norm{\theta^k - w^*}_M^2$, rather than on
  $\expec{}{F(\theta^{k}) - F(w^*)}$. Our key technical result is a descent
  lemma
  (\Cref{lemma:descent-lemma}) allowing us to obtain
  \begin{align}
    \label{eq:sketch-proof:cd-utility:first-ineq}
     & \expec{}{F(\theta^{k+1}) - F^*} - \frac{p - 1}{p} \expec{}{F(\theta^k) - F^*}
       \le \mathbb{E}\norm{\theta^k - w^*}_M^2 - \mathbb{E}\norm{\theta^{k+1} - w^*}_M^2
       + \frac{\norm{\sigma}_M^2}{p}
       \enspace.
  \end{align}
  The above inequality shows that coordinate-wise updates leave a fraction
  $\frac{p-1}{p}$ of the function ``unchanged'', while the remaining
  part decreases (up to additive noise). Importantly, all quantities are
  measured
  in $M$-norm. When summing
  \eqref{eq:sketch-proof:cd-utility:first-ineq} for $k=0,\dots,K-1$,
  its left hand side simplifies and its right hand side is simplified
  as a telescoping sum:
  \begin{align}
    \label{eq:sketch-proof:cd-utility:second-ineq}
     & \frac{1}{p}\sum_{k=1}^{K} \expec{}{F(\theta^{k}) - F^*}
      \le \expec{}{F(\bar w^t) - F^*} + \mathbb{E}\norm{\bar w^t - w^*}_M^2
    + \frac{K}{p}\norm{\sigma}_{M^{-1}}^2\enspace,
  \end{align}
  where $\bar w^t$ comes from $\theta^0 = \bar w^t$. As
  $\bar w^{t+1} = \sum_{k=1}^K \frac{\theta^k}{K}$ and $F$ is convex,
  we have
  $F(\bar w^{t+1}) - F^* \le \frac{1}{K} \sum_{k=1}^K F(\theta^k) -
    F^*$. This proves the sub-linear convergence (up to an additive
  noise term) of the inner loop. The result in the convex case follows
  directly (since $T=1$, only one inner loop is run).  For strongly
  convex $F$, it further holds that
  $\mathbb{E}\norm{\bar w^t - w^*}_M^2 \le
    \frac{2}{\mu_M}\expec{}{F(\bar w^t) - F(w^*)}$.  Replacing in
  \eqref{eq:sketch-proof:cd-utility:second-ineq} with large enough $K$
  gives
  $\expec{}{F(\bar w^{t+1}) - F^*} \le \tfrac{1}{2} \expec{}{F(\bar
      w^{t}) - F^*} + \norm{\sigma}_{M^{-1}}^2,$ and linear convergence
  (up to an additive noise term) follows. Finally, $K$ and $T$ are chosen to
  balance the ``optimization'' and the ``privacy'' errors.
\end{sketchproof}

\begin{remark}
  \label{rmq:improvement-inexact-coordinate-descent}
  Our novel convergence proof of CD is also useful in the non-private
  setting. In particular, we improve upon known convergence rates for
  inexact CD methods with additive error \citep{tappenden2016Inexact},
  under the hypothesis that gradients are noisy and unbiased. In their
  formalism, we have $\alpha = 0$ and
  $\beta = \norm{\sigma}_{M^{-1}}^2/p$. With our analysis, the
  algorithm requires $2pR_M^2 / (\xi - p\beta)$ (resp.
  $4p/\mu_M \log((F(w^0) - F^*) / (\xi - p\beta))$) iterations to
  achieve expected precision $\xi > p\beta$ when $F$ is convex
  (resp. $\mu_M$-strongly-convex \wrt $\norm{\cdot}_M$), improving
  upon \citet{tappenden2016Inexact}'s results by a factor
  $\sqrt{p\beta / 2R_M^2}$ (resp. $\mu_M/2$). See \Cref
  {sec:proof-remark-1} for details.  Moreover, unlike this prior work,
  our analysis does not require the objective to decrease at each
  iteration, which is essential to guarantee DP.
\end{remark}

Our utility guarantees stated in \Cref{thm:cd-utility} directly depend on
precise coordinate-wise regularity measures of the objective function.
In particular, the initial distance to optimal, the strong convexity parameter
and the overall sensitivity of the loss function are measured in the norms
$\smash{\norm{\cdot}_M}$ and $\smash{\norm{\cdot}_{M^{-1}}}$
(\ie weighted by coordinate-wise smoothness constants or their inverse).
In the remainder of this section, we thoroughly compare our utility results
with existing ones for DP-SGD.
We will show the optimality of our utility guarantees in
Section~\ref{sec:utility-lower-bounds}.



\begin{table*}[t]
  \centering
  \caption{
    Utility guarantees for DP-CD, DP-SGD, and DP-SVRG for $L$-component-Lipschitz, $\Lambda$-Lipschitz loss.
  } \label{table:utility-cd-sgd}
    \begin{tabular*}{\textwidth}{c @{\extracolsep{\fill}} c c c}
      \toprule
      & Convex
      & Strongly-convex \\
      \midrule
      DP-CD (this paper)
      & $\displaystyle \widetilde O\left(\frac{\sqrt{p \log(1/\delta)}}{n\epsilon}\norm{L}_{M^{-1}} R_{ M}\right)$
      & $\displaystyle \widetilde O\left(\frac{p \log(1/\delta) }{n^2 \epsilon^2}\frac{\norm{L}_{M^{-1}}^2}{\mu_{ M}} \right)$\\
      \midrule
      \makecell{DP-SGD \citep{bassily2014Private} \\ DP-SVRG \citep{wang2017Differentially}}
      & $\displaystyle \widetilde O\left(\frac{\sqrt{p\log(1/\delta)}}{n\epsilon}\Lambda R_I\right)$
      & $\displaystyle \widetilde O\left(\frac{p \log(1/\delta)}{n^2 \epsilon^2} \frac{ \Lambda^2}{\mu_I}\right)$ \\
      \bottomrule
    \end{tabular*}
  \vskip -0.1in
\end{table*}







\subsection{Comparison with DP-SGD and DP-SVRG}
\label{sec:comparison-with-dp-sgd}

We now compare DP-CD with DP-SGD and DP-SVRG, for which
\citet{bassily2014Private} and \citet{wang2017Differentially} proved utility
guarantees.
In this section, we assume that the loss function $\ell$ satisfies the hypotheses
of \Cref{thm:cd-utility}, and is $\Lambda$-Lipschitz.
We denote by $\mu_I$ the strong convexity parameter
of $\ell(\cdot, d)$ \wrt $\norm{\cdot}_2$ and $R_I$ the equivalent of $R_M$ when
$M$ is the identity matrix $I$.
As can be seen from \Cref{table:utility-cd-sgd}, comparing
DP-CD and DP-SGD boils down to comparing
$\smash{\norm{L}_{M^{-1}} R_{M}}$ with $\Lambda R_I$ for
convex functions and ${\norm{L}_{M^{-1}}^2 }/\mu_M$
with ${\Lambda^2 }/{\mu_I}$ for strongly-convex functions.
We compare these terms in two scenarios, depending on the distribution of
coordinate-wise smoothness constants.
To ease the comparison, we assume that $\smash{R_M = \norm{w^0 - w^*}_M}$ and $\smash{R_I = \norm{w^0 - w^*}_I}$ (which is notably the case when $\psi = 0$), and that $F$ has a unique minimizer $w^*$.

\paragraph{Balanced.}
When the smoothness constants $M$ are all equal,
$\norm{L}_{M^{-1}} R_{ M} = \norm{L}_{2} R_I$ and
${\norm{L}_{M^{-1}}^2 }/{\mu_M} = {\norm{L}_{2}^2 }/{\mu_I}$.  This
boils down to comparing $\norm{L}_{2}$ to $\Lambda$. As
$\Lambda \le \norm{L}_{2} \le \sqrt{p}\Lambda$, DP-CD can be up to $p$
times worse than DP-SGD. This can only happen when features are extremely
correlated, which is generally not the case in machine learning.  We
show empirically in \Cref{sec:stand-datas} that, even in balanced regimes,
DP-CD can still significantly outperform DP-SGD.
\paul{is it really p times worse?}



\paragraph{Unbalanced.}
More favorable regimes exist when smoothness constants are imbalanced.
To illustrate this, consider the case where the first
coordinate of the loss function $\ell$ dominates others.
There, $M_{\max} \!=\! M_1 \!\gg\! M_{\min} \!=\! M_j$ and
$L_{\max} \!=\! L_1 \!\gg\! L_{\min}\!=\! L_j $ for all $j\neq 1$, so that
$L_1^2/M_1$ dominates the other terms of $\norm{L}_{M^{-1}}^2$.  This
yields
$\norm{L}_{M^{-1}}^2 \approx L_1^2 / M_1 \approx \Lambda / M_{\max}$,
and $\mu_M = \mu_I M_{\min}$.
Moreover, if the first coordinate of $w^*$ is already well estimated
by $w^0$ (which is common for sparse models), then
$R_M \approx M_{\min}
  R_I$. %
We obtain that
$\norm{L}_{M^{-1}} R_M \approx \sqrt{{M_{\min}}/{M_{\max}}} \Lambda
  R_I$ for convex losses and
$\frac{\norm{L}_{M^{-1}}}{\mu_M} \approx
  \frac{M_{\min}}{M_{\max}}\frac{\Lambda^2}{\mu_I}$ for strongly-convex
ones.
In both cases, DP-CD can perform arbitrarily better than DP-SGD,
depending on the ratio between the smallest and largest
coordinate-wise smoothness constants of the loss function.  This is
due to the inability of DP-SGD to adapt its step size to each coordinate.
DP-CD thus converges quicker than DP-SGD on coordinates with
smaller-scale gradients, requiring fewer accesses to the dataset, and
in turn less noise addition. We give more details on this comparison
in \Cref{sec-app:comparison-with-dp}, and complement it with an empirical
evaluation on synthetic and real-world data in
Section~\ref{sec:numerical-experiments}.




















version https://git-lfs.github.com/spec/v1
oid sha256:c7f0b993298967d632932cb8fcb2128c51d7dc4f0d2c488198713193b11f8b1a
size 3470



\section{DP-CD in Practice}
\label{sec:dp-cd-practice}

We now discuss practical questions related to DP-CD. First, we show how
to implement coordinate-wise gradient clipping using a single hyperparameter.
Second, we explain how to privately estimate the smoothness constants.
Finally, we
discuss the possibility of standardizing the features and how this relates
to estimating smoothness constants for the important problem of fitting
generalized linear models.


\subsection{Coordinate-wise Gradient Clipping}
\label{sub:clipping}

To bound the sensitivity of coordinate-wise gradients, our analysis of
Section~\ref{sec:diff-priv-coord} relies on the knowledge of Lipschitz
constants for the loss function $\ell(\cdot;d)$ that must hold for all
possible data points
$d\in \cX$, see
inequality \eqref{eq:delta-lipschitz-norm} and the discussion above
it.
This is classic in the analysis of DP optimization algorithms \citep[see
  e.g.,][]
{bassily2014Private,wang2017Differentially}. In practice however, these
Lipschitz constants can be difficult to bound tightly and often
give largely pessimistic estimates of sensitivities, thereby making gradients
overly noisy. To overcome this problem, the common practice in concrete
deployments of DP-SGD algorithms is to \emph{clip per-sample gradients}
so that their norm does not exceed a fixed threshold parameter $C > 0$
\citep{abadi2016Deep}:
\begin{align}
  \label{eq:standard_clipping_rule}
  \clip(\nabla \ell(w), C)
  = \min\Big(1, \frac{C}{\norm{\nabla \ell(w)}}_2\Big)  \nabla \ell(w)\enspace.
\end{align}
This effectively ensures that the sensitivity $\Delta(\clip(\nabla \ell, C))$
of the clipped gradient is bounded by $2C$.

In DP-CD, gradients are released one coordinate at a time and should
thus be clipped in a coordinate-wise fashion. Using the same threshold for
each coordinate would ruin the ability of DP-CD to account for imbalance
across gradient coordinates, whereas tuning coordinate-wise thresholds as $p$
individual
hyperparameters $\{C_j\}_{j=1}^p$
is impractical.

Instead, we leverage the results of \Cref{thm:cd-utility} to adapt them from a
single hyperparameter.
We first remark that our utility guarantees are
invariant to the scale of the matrix $M$. After rescaling
$M$ to
$\widetilde M = \frac{p}{\tr(M)} M$ so that
$\tr(\widetilde M) = \tr(I) = p$, as proposed by
\citet{richtarik2014Iteration}, the key quantity
$\Delta_{\widetilde M^{-1}}(\nabla\ell)$ as defined in \eqref{eq:delta-lipschitz-norm} appears in
our utility
bounds instead of $\norm{L}_{M^{-1}}$. This suggests to parameterize the
$j$-th threshold as $C_j = \sqrt{{M_j}/{\tr(M)}} C$ for some $C > 0$,
ensuring that $\Delta_{\widetilde M^{-1}}(\{\clip(\nabla_j\ell, C_j)\}_{j=1}^p)
  \leq 2C$.
The parameter $C$ thus controls the overall sensitivity, allowing clipped
DP-CD to perform $p$ iterations for the same privacy budget as one iteration
of clipped DP-SGD.

\subsection{Private Smoothness Constants}
\label{sec:priv-smoothn}

DP-CD requires the knowledge of the coordinate-wise smoothness
constants $M_1,\dots,M_p$ of $f$ to set appropriate step sizes (see
\Cref{thm:cd-utility}) and clipping thresholds (see
above).\footnote{In fact, only $\smash{M_j/\sum_{j'} M_{j'}}$ is needed, as we
  tune the clipping threshold and scaling factor for the step sizes.
  See \Cref{sec:numerical-experiments}.}  In most problems, the
$M_j$'s depend on the dataset $D$ and must thus be estimated privately
using a fraction of the overall privacy budget.  Since $f$ is an
average of loss terms, its coordinate-wise smoothness constants are
the average of those of $\ell(\cdot, d)$ over $d\in D$.  These per-sample
quantities are easy to get for typical losses (see
\Cref{sec:data-standardization} for the case of linear models).
Privately estimating $M_1,\dots,M_p$ thus reduces to a classic private
mean estimation problem for which many methods exist.  For instance,
assuming that the practitioner knows a crude upper bound on per-sample
smoothness constants, he/she can compute the smoothness constants of
the $\ell(\cdot, d)$'s, clip them to the pre-defined upper bounds, and
privately estimate their mean using the Laplace mechanism
(see \Cref{sec:priv-estim-smoothn} for
details).  We show numerically in
\Cref{sec:numerical-experiments} that dedicating $10\%$ of the total
budget $\epsilon$ to this strategy allows DP-CD to effectively exploit
the imbalance across gradients' coordinates.



\subsection{Feature Standardization}
\label{sec:data-standardization}

CD algorithms are very popular to solve generalized linear
models \citep{friedman2010Regularization} and their regularized version (\eg
LASSO, logistic regression).
For these problems, the coordinate-wise smoothness constants are
$M_j \propto \frac 1n \norm{X_{:,j}}_2^2$, where $X_{:,j} \in \RR^{n}$
is the vector containing the value of the $j$-th feature. Therefore, standardizing the features to have zero mean and
unit variance (a standard preprocessing step) makes coordinate-wise smoothness
constants equal. However, this requires to compute the
mean and variance of each feature in $D$, which is more
costly than the smoothness
constants to estimate privately.\footnote{We note that the privacy cost of
standardization is rarely accounted for in practical evaluations.}
Moreover, while our theory suggests that DP-CD may not be superior to DP-SGD
when smoothness constants are all equal (see
Section~\ref{sec:comparison-with-dp-sgd}), the
numerical results of \Cref{sec:numerical-experiments} show that DP-CD often
outperforms DP-SGD even when features are standardized.

Finally, we emphasize that standardization is not always possible.
This notably happens when solving the problem at hand is a subroutine of another algorithm.
For instance, the Iteratively Reweighted Least Squares (IRLS) algorithm
\citep{holland1977Robust} finds the maximum likelihood estimate of a
generalized linear model by solving a sequence of linear
regression problems with reweighted features, proscribing standardization.
Similar situations happen when using reweighted $\ell_1$ methods for
non-convex sparse regression \citep{Candes_Wakin_Boyd08}, relying on convex (LASSO) solvers for the inner loop.
DP-CD is thus a method of choice to serve as subroutine in private versions of these algorithms.


version https://git-lfs.github.com/spec/v1
oid sha256:1bb624c6d850fa3fd5805fa6f9c266eaac79da2d816e27d80c36752dd05e0386
size 7783

version https://git-lfs.github.com/spec/v1
oid sha256:ac4d7ee4654fc3ffca23ee016831c5e8f4efff208f631ba4b5484e8413aca25e
size 4146

version https://git-lfs.github.com/spec/v1
oid sha256:7a58af7b10360cd844b17ee8415b00f0465e45e261797af3f201bfc5e0dae5e2
size 1626


\section*{Acknowledgments}

The authors would like to thank the anonymous reviewers who provided
useful feedback on previous versions of this work, which helped to improve
the paper.

This work was supported in part by the Inria Exploratory Action FLAMED and
by the French National Research Agency (ANR) through grant ANR-20-CE23-0015
(Project PRIDE) and ANR-20-CHIA-0001-01 (Chaire IA CaMeLOt).


\printbibliography



\appendix %
\newpage


\onecolumn


version https://git-lfs.github.com/spec/v1
oid sha256:9aeafbcb62ae000999b4c641d64373d85591458f1f6f018678d25189af2226f0
size 4551




\section{Proof of \Cref{thm:dp-cd-privacy}}
\label{sec:proof-privacy}


To track the privacy loss of an adaptive composition of $K$ Gaussian
mechanisms, we use Rényi Differential Privacy
\citep[RDP]{mironov2017Renyi}. We note that similar results are
obtained with zero Concentrated Differential Privacy
\citep{bun2016Concentrated}. This flavor of differential privacy,
gives tighter privacy guarantees in that setting, as it reduces the
noise variance by a multiplicative factor of $\log(K/\delta)$ in
comparison to the usual advanced composition theorem of differential
privacy \citep{dwork2006Calibrating}.  Importantly, RDP can be
translated back to differential privacy.

In this section, we recall the definition and main properties of zCDP.
We denote by $\cD$ the set of all datasets over a universe $\cX$
and by $\cF$ the set of possible outcomes of the randomized
algorithms we consider.

\subsection{Rényi Differential Privacy}

We will use the Rényi divergence (\Cref{def:renyi-div}), which gives
a distribution-oriented vision of privacy.

\begin{definition}[Rényi divergence, \citealt{vanerven2014Renyi}]
  \label{def:renyi-div}
  For two random variables $Y$ and $Z$ with values in the same domain $\cC$,
  the Rényi divergence is, for $\alpha > 1$,
  \begin{align}
    D_\alpha(Y||Z)
    = \frac{1}{\alpha - 1} \log \int_{\cC} \prob{Y=z}^\alpha \prob{Z=z}^{1 - \alpha} dz\enspace.
  \end{align}
\end{definition}



We now define RDP in \Cref{def:rdp}. RDP provides a strong privacy
guarantee that can be converted to classical differential privacy
(\Cref{lemma:rdp-to-dp} and \Cref{cor:rdp-to-dp}).

\begin{definition}[Rényi Differential Privacy,
\citealt{mironov2017Renyi}]
  \label{def:rdp}
  A randomized algorithm $\cA : \mathcal D \rightarrow \mathcal F$ is
  $(\alpha, \epsilon)$-Rényi-differentially private (RDP) if, for all
  all datasets $D, D' \in \mathcal D$ differing on at most one
  element,
  \begin{align}
    D_\alpha(\cA(D) || \cA(D')) \le \epsilon\enspace.
  \end{align}
\end{definition}

\begin{lemma}[{\citealt[Proposition~3]{mironov2017Renyi}}]
  \label{lemma:rdp-to-dp}
  If a randomized algorithm $\cA : \mathcal D \rightarrow \mathcal F$
  is $(\alpha, \epsilon)$-RDP, then it is
  $(\epsilon + \frac{\log(1/\delta)}{\alpha - 1}, \delta)$-differentially
  private for all $0 < \delta < 1$.
\end{lemma}

\begin{remark}
  \label{rmq:rdp-to-dp}
  The above $(\alpha,\epsilon)$-RDP guarantees hold for multiple
  values of $\alpha,\epsilon$. As such, $\epsilon = \epsilon(\alpha)$
  can be seen as a function of $\alpha$, and \Cref{lemma:rdp-to-dp}
  ensures that the algorithm is $(\epsilon', \delta)$-DP for
  \begin{align}
    \epsilon' = \min_{\alpha > 1} \left\{ \epsilon(\alpha) + \frac{\log(1/\delta)}{\alpha - 1} \right\}\enspace.
  \end{align}
\end{remark}

We can now restate in \Cref{thm:rdp-composition} the composition
theorem of RDP, which is key in designing private iterative
algorithms.

\begin{theorem}[{\citealt[Proposition~1]{mironov2017Renyi}}]
  \label{thm:rdp-composition}
  Let $\cA_1, \dots, \cA_K : \cD \rightarrow \cF$ be $K > 0$
  randomized algorithms, such that for $1 \le k \le K$, $\cA_k$
  is $(\alpha, \epsilon_k(\alpha))$-RDP,
  where these algorithms can be chosen adaptively (i.e., $\cA_k$ can use
  to the output of $\cA_{k'}$ for all $k' < k$).
  Let $\cA : \mathcal D \rightarrow \mathcal F^K$ such that for $D \in \cD$,
  $\cA(D) = (\cA_1(D), \dots, \cA_K(D))$.
  Then $\cA$ is $\left(\alpha, \sum_{k=1}^K \epsilon_k(\alpha)\right)$-RDP.
\end{theorem}



Finally, we define the Gaussian mechanism (\Cref{def:gaussian-mechanism}), as used
in \Cref{algo:dp-cd}, and restate in \Cref{lemma:gaussian-rdp-private} the
privacy guarantees that it satisfies in terms of RDP.

\begin{definition}[Gaussian mechanism]
  \label{def:gaussian-mechanism}
  Let $f: \mathcal D \rightarrow \RR^p$, $\sigma > 0$, and $D \in \cD$.
  The Gaussian mechanism for answering the query $f$ is defined as:
  \begin{align}
    \cM_f^{Gauss}(D; \sigma) = f(D) + \mathcal N\left( 0, \sigma^2 I_p
    \right)\enspace.
  \end{align}
\end{definition}

\begin{lemma}[{\citealt[Corollary 3]{mironov2017Renyi}}]
  \label{lemma:gaussian-rdp-private}
  The Gaussian mechanism with noise $\sigma^2$ is
  $(\alpha, \frac{\Delta(f)^2 \alpha}{2 \sigma^2})$-RDP, where
  $\Delta(f) = \sup_{D,D'} \norm{f(D) - f(D')}_2$ (for neighboring
  $D, D'$) is the sensitivity of $f$.
\end{lemma}
\begin{proof}
  The function $h = \frac{f}{\Delta(f)}$ has sensitivity $1$, thus for
  any $s > 0$, the Gaussian mechanism $\cM_h^{Gauss}(\cdot;s)$ is
  $(\alpha, \frac{\alpha}{2\sigma^2})$-RDP
  \citep[Corollary~1]{mironov2017Renyi}. As $f = \Delta(f) \times h$,
  we have
  $\cM_f^{Gauss}(\cdot;\sigma) = \Delta(f) \times
  \cM_h^{Gauss}(\cdot;\frac{\sigma}{\Delta(f)})$.
  This mechanism is thus
  $(\alpha, \frac{\Delta(f)^2\alpha}{2\sigma^2})$-RDP.
\end{proof}


\begin{corollary}
  \label{cor:rdp-to-dp}
  Let $0 < \epsilon \le 1, 0 < \delta < \tfrac{1}{3}$. If a randomized
  algorithm $\cA : \mathcal D \rightarrow \mathcal F$ is
  $(\alpha, \frac{\gamma \alpha}{2 \sigma^2})$-RDP with $\gamma > 0$ and
  $\sigma = \frac{\sqrt{3 \gamma \log(1/\delta)}}{\epsilon}$ for all
  $\alpha > 1$, it is also $(\epsilon, \delta)$-DP.
\end{corollary}

\begin{proof}
  From \Cref{rmq:rdp-to-dp} it holds that $\cA$ is
  $(\epsilon',\delta)$-DP with
  $\epsilon' = \min_{\alpha > 1} \left\{ \frac{\gamma \alpha}{2 \sigma^2} +
    \frac{\log(1/\delta)}{\alpha - 1} \right\}.$ This minimum is
  attained when the derivative of the objective is zero, which is the
  case when
  $\frac{\gamma}{2\sigma^2} = \frac{\log(1/\delta)}{(\alpha - 1)^2}$,
  resulting in $\alpha = 1 + \sqrt{\frac{2\log(1/\delta)\sigma^2}{\gamma}}$.
  $\cA$ is thus $(\epsilon', \delta)$-DP with
  \begin{align}
  \label{eq:full_privacy_formula}
    \epsilon' = \frac{\gamma}{2\sigma^2} + \frac{\sqrt{\gamma \log(1/\delta)}}{\sqrt{2} \sigma}
    + \frac{\sqrt{\gamma \log(1/\delta)}}{\sqrt{2} \sigma}
     = \frac{\gamma}{2\sigma^2} + \frac{\sqrt{2\gamma \log(1/\delta)}}{\sigma}\enspace.
  \end{align}

  Choosing $\sigma = \frac{\sqrt{3 \gamma \log(1/\delta)}}{\epsilon}$
  now gives
  \begin{align}
    \epsilon'
    = \frac{\epsilon^2}{6 \log(1/\delta)} + \sqrt{{2}/{3}} \epsilon
    \le (1/6  + \sqrt{2/3}) \epsilon
    \le \epsilon\enspace,
  \end{align}
  where the first inequality comes from $\epsilon \leq 1$, thus $\epsilon^2
  \le \epsilon$
  and $\delta < 1/3$ thus $\frac{1}{\log(1/\delta)} \le 1$.
  The second inequality follows from $1/6  + \sqrt{2/3} \approx 0.983 < 1$.
\end{proof}


\subsection{Proof of \Cref{thm:dp-cd-privacy}}

We are now ready to prove \Cref{thm:dp-cd-privacy}.
From the privacy perspective, \Cref{algo:dp-cd} adaptively releases
and post-processes a series of gradient coordinates protected by the Gaussian
mechanism. We thus start by proving
\Cref{lemma:rdp-composition-gaussian-k}, which gives an $
(\epsilon,\delta)$-differential privacy
guarantee for the adaptive composition of $K$ Gaussian mechanisms.

\begin{lemma}
  \label{lemma:rdp-composition-gaussian-k}
  Let $0 < \epsilon \le 1$, $\delta < 1/3$, $K > 0$, $p > 0$, and
  $\{f_k : \RR^p \rightarrow \RR\}_{k=1}^{k=K}$ a family of $K$ functions.
  The adaptive composition of $K$ Gaussian mechanisms,
  with the $k$-th mechanism releasing
  $f_k$ with noise scale $\sigma_k = \frac{\Delta(f_k) \sqrt{3K \log(1/\delta)}}{\epsilon}$
  is $(\epsilon, \delta)$-differentially private.
\end{lemma}
\begin{proof}
  Let $\sigma > 0$. \Cref{lemma:gaussian-rdp-private} guarantees that
  the $k$-th Gaussian mechanism with noise scale
  $\sigma_k = \Delta(f_k) \sigma > 0$ is
  $(\alpha, \frac{\alpha}{2 \sigma^2})$-RDP.  Then, the composition of
  these $K$ mechanisms is, according to \Cref{thm:rdp-composition},
  $(\alpha, \frac{k\alpha}{2 \sigma^2})$-RDP.  This can be converted
  to $(\epsilon,\delta)$-DP via \Cref{cor:rdp-to-dp} with
  $\gamma = K$, which gives
  $\sigma_k = \frac{\Delta(f_k)\sqrt{3 k \log(1/\delta)}}{\epsilon}$
  for $k\in[K]$.
  \end{proof}

We now restate \Cref{thm:dp-cd-privacy} and prove it.

\begin{restate-theorem}{\ref{thm:dp-cd-privacy}}
  Assume $\ell(\cdot;d)$ is $L$-component-Lipschitz $\forall d\in\cX$.
  Let $\epsilon < 1$ and $\delta < 1/3$.
  If $\sigma_j^2 = \frac{12L_j^2 TK \log(1/\delta)}{n^2\epsilon^2}$
  for all $j \in [p]$,
  then \Cref{algo:dp-cd} satisfies $(\epsilon, \delta)$-DP.
\end{restate-theorem}

\begin{proof}
  For $j \in [1, p]$, $\nabla_j f$ in \Cref{algo:dp-cd} is released using the
  Gaussian mechanism with noise variance $\sigma_j^2$.
  The sensitivity of $\nabla_j f$
  is $\Delta(\nabla_j f) = \frac{\Delta (\nabla_j \ell)}{n} \le \frac{2L_j}
  {n}$. Note that $TK$ gradients are released, and
  \begin{align*}
    \sigma_j^2 = \frac{12L_j^2 TK \log(1/\delta)}{n^2\epsilon^2} \text{ for } j \in [1, p]\enspace,
  \end{align*}
  thus by \Cref{lemma:rdp-composition-gaussian-k} and the post-processing
  property of DP,
  \Cref{algo:dp-cd}
  is  $(\epsilon, \delta)$-differentially private.
\end{proof}




version https://git-lfs.github.com/spec/v1
oid sha256:f6050a5a9016d22fb6631f76a49b9bf2a1b05611211d116f5e6f0eec481a5883
size 27886


version https://git-lfs.github.com/spec/v1
oid sha256:4644b9f4d872c7ab1940054891e4cfff836554dd8d249723d2a3d043c050ffcb
size 5238


version https://git-lfs.github.com/spec/v1
oid sha256:789b9619df5452188d3ec879e430266570ad98a176e2ec0445f67a0ffb02ff4b
size 4561


version https://git-lfs.github.com/spec/v1
oid sha256:ecc2c161b81f7f31ab4b1083458a2536595aae92e450f96ca9c724e8bc74e0aa
size 24493


version https://git-lfs.github.com/spec/v1
oid sha256:6a0d67cb43a3ed4da9e40a3f59a93b16780273f07669990e31fe553b16ad797a
size 1577


version https://git-lfs.github.com/spec/v1
oid sha256:77030193464a00fc40e75712c0cdb36efcc2065436da70589040164814db80d4
size 11012





\end{document}
